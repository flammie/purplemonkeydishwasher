\documentclass[free]{flammie}

\newif\ifcameraready
\camerareadyfalse

\usepackage{polyglossia}
\usepackage{fontspec}
\usepackage{xunicode}
\usepackage{xltxtra}
\usepackage{url}
\usepackage{hyperref}
\usepackage{expex}
\usepackage[obeyDraft]{todonotes}
%\usepackage{todonotes}


\begin{document}


\title{TITLE \footnotepubrights{
    This work is licensed under a Creative Commons Attribution–NoDerivatives
    4.0 International Licence.  Licence details:
    \url{http://creativecommons.org/licenses/by-nd/4.0/}.
}}


\ifcameraready
\author{AUTHOR \\
    \url{EMAIL} \\
    AFFILIATION \\
}
\fi

\maketitle
\begin{abstract}

    ABSTRACT
\end{abstract}

\section{Introduction}\label{sec:introduction}

INTRODUCTION

The main research questions can be formulated as follows:

\section{Background}\label{sec:background}

BACKGROUND

The question has been studied before by \cite{}.

The systems \cite{} and \cite{} are similar than the work of this article, but
they use different things to solve problems as well.

\section{Methods}\label{sec:methods}

METHODS

We define \textit{Finite-State Automaton} (FSA) as n-tuple $(\Sigma, )$.

\section{Conclusions}\label{sec:conclusions}

CONCLUSIONS

In this article we demonstrated

\section*{Acknowledgements}

ACKNOWLEDGMENTS

\bibliographystyle{unsrt}
\bibliography{CONFERENCEYEAR}

\end{document}
% vim: set spell:
