\documentclass[a4paper,12pt]{article}

\usepackage{fontspec}
\usepackage{xunicode}
\usepackage{xltxtra}
\usepackage{expex}

\usepackage{url}
\usepackage{hyperref}

\usepackage[margin=3cm]{geometry}

\usepackage{natbib}

\linespread{1.3}

\defaultfontfeatures{Mapping=tex-text}
\setmainfont{Times New Roman}

\title{Omorfi---The New Old of Computational Morphological Analysis of Finnish
Language}
\iffalse
\author{Tommi A Pirinen}
\fi
\date{}
\begin{document}

\maketitle
\iffalse
\begin{abstract}

    This squib describes a contemporary system for computational analysis of
    morphology of Finnish word-forms. The system we present is at its core a
    finite-state morphology, just like other popular systems for computational
    Finnish morphology have been since 1983.  The purpose of this squib is to
    describe the key differences and new approaches improving the previous
    versions of the analysers. Firstly an extended classification was made nto
    accommodate computational semi-automatic classification of new words and
    fine-tune the inflectional variants, and to systematically go through the
    synchronic variations in morphophonology of Finnish on the grapheme level.
    Secondly we pursue some of the recent developments in the field of
    statistical computational linguistics.  We show how we have adapted
    statistical approaches into a traditional rule-based approach that was used
    with Finnish morphology. Thirdly, our system is based fully on free and
    open source systems and approaches. A practical benefit of this approach is
    shown with the use of crowd-sourced dictionary Wiktionary as one additional
    of lexical data.  To evaluate our analyser, we have analysed large text
    corpora on the system and verified that only 1~\% of the word-forms remains
    unknown to the system.  We have also made some rudimentary quality and
    speed comparisons against results of the systems currently in use to show
    that we have a faster system that still recreates the main results of other
    systems faithfully if needed.

\end{abstract}
\fi
\section{Introduction}

Computational morphological analysis is a central component for most of the
computational applications of linguistic analysis. Computational morphological
analysis of Finnish language was first described some 30 years
ago~\citep{koskenniemi1983twolevel}. The aim of this paper is to examine the
differences of our current approach to traditional systems which are used for
the task and to highlight some of the new development. The new development that
we discuss are mainly in line with the recent trends in the field of
computational linguistics. In the recent years, there has been a sharply rising
interest in the statistical methods for computational linguistics. However, a
morphologically complex language like Finnish does not lend itself easily for
statistical treatment, so we try to bring into focus the modifications to the
basic statistical approaches that we have applied to combine statistical data
with Finnish morphological analysis.

One of the main goals of this paper is to act as the central scientific
documentation of our current morphological analyser, and we have made an
attempt to highlight the long-term design goals of the system instead of
transitional and volatile features of a fast-moving computer software that is
developed by a base of open source and language enthusiasts.

In computational linguistics, the beginning of the 2000's was focused into
statistical language models and engineering. It is commonly argued that
statistical models are not as easily suitable for morphologically complex
languages like Finnish as they are for mostly isolating languages like English.
Our analyser is based on the same assumption, and it is similar rule-based
analyser as described in the earlier research of Finnish analysis. In this
squib we show how we have integrated statistical features with a traditional
rule-based morphological analyser of Finnish language. The system we use for
the development is based on one that has been developed earlier at the
University of Helsinki~\citep{pirinen2008suomen}. We furthermore demonstrate a
full-fledged version of named system with large scale dictionaries, as opposed
to the earlier which seems to be principally a proof of concept for noun and
verb inflectional morphology.

Another recent development in the computational language models is the concept
of \emph{maintainability} of these computational systems, e.g.
in~\cite{maxwell2008joint,pirinen2011modularisation}. Specifically we will show
how we use the power of \emph{crowd-sourcing} to keep up with the new words,
neologisms and other rare words missing from dictionary. In particular we study
the use of the popular online dictionary Wiktionary as a source of additional
lexical data. The crowd-sourcing as well as a few newer morphological phenomena
in the language have implications to lexicographical structure of the data as
well, so we try to describe in this article some of the newest findings based
on the word data we have added to lexical resources of our systems. An example
of such recent addition to morphophonological variations is introduction of
quantitative consonant gradation of voiced stops~\cite[§ 41]{visk}. This
required some new additions to existing classifications. Furthermore we have
re-analysed the generalisations of old lexicographical classifications in
assumption that resulting system is more favorable for computational systems as
well as human classifiers to make educated guesses when classifying and
verifying new lexical entries.

The basic framework of the computational system we present here is largely
unchanged from the one introduced in~\citet{koskenniemi1983twolevel}. 
The finite-state automata remain the state-of-the-art for morphological analysis
of morphologically complex languages. The main
technological difference is that we are now using weighted finite-state 
automata~\citep{openfst}. In practice this means that we have a capability to
express statistics or preference relations in our morphological dictionaries.
In this squib we show how we have used existing statistical methods with
our Finnish morphological analyser.

One notable practical distinction in our system is licensing policy. Omorfi
analyser is a free and open source product\footnote{All data available in
\url{http://code.google.com/p/omorfi/}}.  In contemporary computational
linguistics, freeness of systems and data is rightly seen as a cornerstone of
properly conducted science, as it fulfills the requirement of repeatability by
not setting unnecessary fences for the repetition of the scientific results.
This also means that it can be downloaded to any home computer and modified to
each end user application at will. There is a large base of recent research
supporting this, specifically for Finnish the latest is
by~\citet{koskenniemi2008build}.

On the linguistic side, one of the central design decisions in Omorfi is to be
an analyser that captures the scientific consensus of contemporary
\emph{linguistic knowledge}.  This means that the system is built to create
analyses that have been accepted by linguistic community. Specifically we try
to avoid making analysis decisions based on ad hoc needs of certain types of
computational linguistics software: ideally all analyses should be traceable to
grammar descriptions such as~\citet{visk} or scientific papers.

To summarise, this article is an overview of a Finnish morphological analyser.
It consists of the methodological advances found from past years research of
finite-state morphologies and computational linguistics, and contrasts the
resulting system to the previous state-of-the-art in morphological analysis of
Finnish.

The paper is organised as follows: In section~\ref{sec:background}, we go
through the basic concepts of Finnish morphology, modern finite-state
technology and the management of lexicographical data that is relatively
central to the new system. In section~\ref{sec:data} we describe the various
sources of lexical data that are used in building of the morphological
dictionary and demonstrate some engineering aspects that were needed to manage
vast amounts of variously annotated dictionaries. In
section~\ref{sec:evaluation} we measure the analyser's suitability for common
tasks by performing various corpus annotation tasks and comparing our results
to other systems currently in use. In section~\ref{sec:discussion} we go
through some practical implications of our system in field of computational
linguistics as well as research of Finnish linguistics by computer. Finally, in
section~\ref{sec:conclusion} we summarise the findings.

\section{Background}
\label{sec:background}

Finnish inflectional morphology has been widely documented for a long time now,
from \citet{setala1920suomen} to \citet{hakulinen1968suomen} and
\citet{karlsson1982suomen} to the most recent points in \citet{visk}. This
groundwork in morphological documentation is important basis to keep in mind,
even when considering computational analysis of morphology, since
we believe that the point of computational morphological analyser is primarily
to capture contemporary linguistic knowledge about word-forms as accurately as
possible.  Therefore we have made an attempt to follow the written research of
Finnish morphology wherever possible.

There have been numerous implementations of computational morphological
analysis of Finnish along the years. These  can be dated back to at least
1980's, including~\citet{koskenniemi1983twolevel}, which is among the more
influential works in the field of computational morphology.  More recent works
have used e.g. a fully statistical approach for approximating a morphological
segmentation of Finnish~\citep{creutz2005unsupervised}.
In~\citet{ranta2008predictable}, Ranta presents a morphological analysis of
Finnish based on \emph{smart paradigm} system. The treatment of Finnish
lexicography in the smart paradigm system is useful reference for our work,
since it contains some important notes about the paradigmatic system of Finnish
inflection that we have used.  Otherwise, the implementation of our analyser
follows the traditional works on finite-state morphology
by~\citet{koskenniemi1983twolevel,beesley2003finite}.

The practical way how finite-state machines work in morphological analysers has
some implications to the low-level output result of the system. In
figure~\ref{fig:output-example} we give an example of one specific output
formatting scheme. The example is result of analysing the beginning sentence of
the universal declaration of human rights. The same example is represented in
standard linguistic gloss in example~\ref{ex:undhr}. We have omitted the
part-of-speech classes and other lexical information in the glosses; the main
points to note are the ambiguity of the compound \emph{tasavertaisina}, which
is found in the dictionary as lexicalised word `equal' as well as productive
compound of \emph{tasa} `even' and vertainen `peer' (noun) / `equal'
(adjective), and how the systematic ambiguity of the third person possessive is
represented. The order of results in figure~\ref{fig:output-example} is
important, the analyser prefers the reading with lexicalised word-form over the
productively compounded ones.

\ex
\begingl
\gla Kaikki ihmise-t synty-vät vapa-i-na ja tasa=vertais-i-na arvo-lta-an ja oikeuks-i-lta-an//
\glb All.{\sc Sg.Nom} human-{\sc Pl.Nom} born-{\sc Ind.Pres.Sg3} free-{\sc Pl-Ess} and.{\sc Conj} equal=value-{\sc Pl-Ess} worth-{\sc Sg.Ela-Sg3/Pl3} and.{\sc Conj} right-{\sc Pl-Ela-Sg3/Pl3}//
\glft `All human beings are born free and equal in dignity and rights.' \label{ex:undhr}//
\endgl
\xe


In practice, a finite-state analyser is a system that maps a string
representing a word-form in text to a set of string comprising the potential
analyses and vice versa. In addition, contemporary systems like ours assign a
numeric value to each string pair. This can be used to rank the possible
readings.  The system Omorfi is using for implementing the ranking is based on
an algebraic structure called \emph{tropical semiring} which describes how the
values are to be interpreted. What it practically means is that we use
numerical weights that are combined together using regular addition and the
ranking is done in ascending order of values. For this reason it is commonly
called a penalty weighting scheme: the bigger the weight the more likely the
analysis is, and to reduce the ranking of an analysis in the result set, we
simply add weight to it. In a statistical system this value is directly derived
from the probability of the analysis using formula $W(wf) = -\log(P(wf)$, where
W is the weight function and P is the probability function of word-form $wf$.
The weight function can be determined separately for the word-forms and the
analyses.  If there is not enough statistical data, however, the system can be
easily extended with hand-written weighting procedures. For Finnish this means
that, when there is not available a corpus of morphological analyses, it can be
substituted with estimations of likelihoods of specific analyses. Omorfi has
one implementation of this based on \citet[§ 1227]{visk} for cases. Rest of the
analyses were scaled based on the case values and estimated on a short manually
disambiguated text, but these are subject to vary between different versions of
analyser as more data becomes available, and in practice should often be
tailored towards the use of each application.


\begin{figure}
    \begin{scriptsize}
\begin{verbatim}
Kaikki  [WORD_ID=kaikki][POS=PRONOUN][SUBCAT=QUANTOR][NUM=SG][CASE=NOM]

ihmiset [WORD_ID=ihminen][POS=NOUN][NUM=PL][CASE=NOM]

syntyvät        [WORD_ID=syntyä][POS=VERB][VOICE=ACT][DRV=VA][PCP=VA][NUM=PL][CASE=NOM]
syntyvät        [WORD_ID=syntyä][POS=VERB][VOICE=ACT][MOOD=INDV][TENSE=PRESENT][PERS=PL3]

vapaina [WORD_ID=vapaa][POS=ADJECTIVE][NUM=PL][CASE=ESS]
vapaina [WORD_ID=vapaa][POS=NOUN][NUM=PL][CASE=ESS]

ja      [WORD_ID=ja][POS=PARTICLE][SUBCAT=CONJUNCTION]

tasavertaisina  [WORD_ID=tasavertainen][POS=ADJECTIVE][NUM=PL][CASE=ESS]
tasavertaisina  [WORD_ID=tasa][POS=NOUN][NUM=SG][CASE=NOM][GUESS=COMPOUND]
    [WORD_ID=vertainen][POS=ADJECTIVE][NUM=PL][CASE=ESS]
tasavertaisina  [WORD_ID=tasa][POS=NOUN][NUM=SG][CASE=NOM][GUESS=COMPOUND]
    [WORD_ID=vertainen][POS=NOUN][NUM=PL][CASE=ESS]

arvoltaan       [WORD_ID=arvo][POS=NOUN][NUM=SG][CASE=ABL][POSS=PL3]
arvoltaan       [WORD_ID=arvo][POS=NOUN][NUM=SG][CASE=ABL][POSS=SG3]

ja      [WORD_ID=ja][POS=PARTICLE][SUBCAT=CONJUNCTION]

oikeuksiltaan   [WORD_ID=oikeus][POS=NOUN][NUM=PL][CASE=ABL][POSS=PL3]
oikeuksiltaan   [WORD_ID=oikeus][POS=NOUN][NUM=PL][CASE=ABL][POSS=SG3]

.       [WORD_ID=.][SUBCAT=PUNCTUATION][BOUNDARY=SENTENCE]
\end{verbatim}
    \end{scriptsize}
    \caption{Example of Omorfi output in analysis mode with the default
        long analysis style \label{fig:output-example}}
\end{figure}

One of the main building blocks fora  rule based analysis of morphology is the
lexicographical data. In order to correctly inflect a word, a root and some
classification to determine inflection is needed. Without the classification,
it is only possible to somewhat guess the inflectional patterns of the word.
For most of the Finnish words, guessing a correct pattern is quite reliable,
but for purposes where high precision is required, such as spell-checking, a
manually verified lexicon is necessary.  Also, the task of classification is
only needed once per word, and does not necessarily require expert skills
beyond native-like language understanding.  Therefore, the main analyser
provided for use only uses known words and forms, and the guessers are used
to extend the lexicon and classifying new words.

The requirement for large amounts of lexicographical data is very apparent in
applications like spell-checking. In spell-checking and correction it is
required to know how suitable a word-form is, for standard written Finnish.
For example for compound words---a potentially infinite class of words in
Finnish---a compound that is known to exist and has been attested in a number
of texts is a better word than a compound that can be created by morphological
combination of nominals' forms.

\section{Data}
\label{sec:data}

There are a few freely available open resources for lexicographical data of
Finnish. The first one  we used is based on a dictionary maintained by Research
institute of languages in Finland. The lexicographical data of the dictionary
has been available under free software licence since 2007 as \emph{Nykysuomen
sanalista}\footnote{\url{http://kaino.kotus.fi/sanat/nykysuomi}}).  This data
consists of some 90,000 words.  The classification is detailed in their web
site and in the published dictionary.  It uses 80 numeric classes, which freely
combine with 14 alphabetic classes for the stem consonant gradation.  This
means theoretically 1,120 classes, though not all combinations are possible or
used. 

The lexical data Nykysuomen sanalista is conservative and does not
update often, so it has been necessary to seek for other sources of lexical
data. This is required to overcome the lack of number of classes of words that
were missing but are important in practical applications. This includes large
classes like neologisms, jargon, proper nouns and abbreviations.

The second source of lexical data we acquired from the internet is a free, open
source database named
Joukahainen\footnote{\url{http://joukahainen.puimula.org}}. In this database,
the words are classified with a different classifying scheme.

For another source of lexical data we are using the power of crowd-sourcing. In
this case, we selected the popular Wiktionary project.  Wiktionary is a
dictionary that is built in the Wikipedia style, where everyone can freely add
and edit its data. This has some implications to the quality and reliability of
the data. It needs to be verified more carefully before adding to the live
analyser. The classification of Wiktionary words varies between the official
dictionary classification to none to custom ad hoc classes that Wiktionary
contributors have come up with.

Finally, we have used data that has been produced for various projects at
university of Helsinki, including other lexical projects, such as FinnWordNet,
as well as projects and experiments to collect and classify proper nouns, named
entities and so forth. 

All the new sets of lexical data have been further classified using
semi-automatic computational methods. For the needs of this classifications, we
have further developed a new classification, as well as computational methods
to help the workers to classify and verify the data.

The full lexical data used in the current version of our analyser consists of
396,673 classified lexemes, the classification is roughly summarised by word's
classes and their origin in table~\ref{table:lexical}. The columns are the
lexical sources used: Nykysuomen sanalista in column titled \emph{Kotus}
(RILF), \emph{Joukahainen}, \emph{Wiktionary}, terms harvested in various
research project in University of Helsinki in column titled \emph{UH}, and a
total sum in the last column. The rows represent coarse morphological
classification, with proper nouns separated into their own row. On the
\emph{others} row are the adverbs, adpositions and such words with defective
paradigms. These words in current dictionaries have one entry per inflectional
ending.

\begin{table}
  \centering
    \begin{tabular}{|l|r|r|r|r||r|}
        \hline
        \bf Database: & Kotus & Joukahainen & Wiktionary & UH & \bf Total \\
        \bf Class   & & & & & \\
        \hline
        Adjectives     & 10,537 & 652 & 59 & 6,368 & 17,616 \\
        Nouns          & 67,608 & 2,945 & 1,133 & 1,589 & 73,275\\
        (Proper nouns) & 20 & 5,394 & 70 & 262,330 & 267,814\\
        Verbs          & 9,685 & 476 & 12 & 499 & 10,672\\
        Others         & 6620 & 5 & 12 & 25,536 & 32,173 \\
        \hline
        \bf Total      & 94,290 & 9,472 & 1,286 & 296,322 & 396,637 \\
        \hline
    \end{tabular}
  \caption{Lexical data used in the analyser
  \label{table:lexical}}
\end{table}

\section{Methods}

The system presented uses established techniques, such as finite state
morphology~\cite{beesley2003finite}. In this section we describe some specific
details of our finite-state implementation and the lexical data classification
in the analyser briefly. For further engineering details it is suggested to
read the up-to-date documentation and source code in the project's web
site~\footnote{\url{http://code.google.com/p/omorfi/}}.

\subsection{Implementation of the Analyser}

The core computational methodology for the implementation of the analyser is
finite-state technology, as introduced
in~\cite{koskenniemi1983twolevel,beesley2003finite}. On top of that we
have applied recent extensions from the research of finite-state morphology,
such as weighted finite-state methods~\cite{openfst,hfst2012}. This brings the
traditional rule-based language analyser towards the statistical language
analysers that are widely popular in handling of morphologically less
complex languages.

A naive way to apply statistical methods for morphological analysis is
straight-forward; to get the likelihood of word-form $P(w)$ as a surface form
$w$, we calculate the amount of word-forms $f(w)$ in a corpus and divide it by
the number of word-forms in the whole corpus $CS$. This is shown in formula:

$$
P(w) = \frac{f(w)}{CS}
$$

For a morphologically complex language with relatively little corpora, the
frequency of most of the word-forms is $0$. To handle this gracefully, we need
to estimate the probability of unseen words and offset a part of the
probability mass among them. A basic schoolbook approach to this is
\emph{additive discounting}, where we simply assume that all unseen words have
appeared in the corpus $\alpha$ times, and we increase the corpus size by
$\alpha$ per each type $T$ in the corpus. This is shown in formula:

$$
P_{ADD\alpha}(w) = \frac{f(w) + \alpha}{CS + T \times \alpha}
$$

Another consideration that we used from recent research findings on training
compounding language models~\citep{pirinen2009weighting} is the estimation of
the compound words by their parts. This means that given unseen compound
\emph{banaaniovi}, the estimated likelihood is $P(\mathrm{banaani}) \times
P(\mathrm{ovi})$, rather than $P_{ADD\alpha}(\mathrm{\hat banaaniovi})$. One of
the non-apparent results of such training is that in case of ambiguous
compounds, the analyser will systematically prefer the ones with least words.

\subsection{Classification and Its Use}

The lexical data that is used in the analyser comes from variety of sources,
with different quality, classification schemes (or lack of thereof).
To maintain the data we have rethought the classification scheme by
systematically analysing the word-forms of each paradigm on character level such
that for each stem variation and each suffix, the word is considered to be in
the same class iff each character of the each stem's corresponding varying part
and each character in each suffix allomorph variant is the same. 

To contrast our classification with the existing classification we get, for
example, two separate classes for each class by the difference in vowel
harmony: this is evident because for inessive suffix \emph{ssä} and \emph{ssa}
are not the same on the character level. Similarly we get four classes instead
of one for the non-alternating nominal stems since the singular illatives
\emph{on}, \emph{un}, \emph{yn}, \emph{ön} are not the same on character level.
Since there is a number of minor details in Finnish morphophonology that
results combinatorically in quite many classes, we provide the whole table with
mappings to existing other systems in our wiki
page\footnote{\url{https://code.google.com/p/omorfi/wiki/InflectionClassTables}}.

This classification scheme can have some implications also to the
implementation of the resulting grammar. With the classification it is
essentially possible to build the analyser using naive concatenative morphology
systems without any additional variation schemes. For example, the stem
variation present in nominals ending in \emph{-nen}, can be built using the
stem without the \emph{-nen} part, by selecting one of \emph{nen}, \emph{se} or
\emph{s} to create the nominative stem, the singular vowel stem and the plural
stem respectively.

To help the collection of new lexical data, we have used some naive
automatic classification methods, such as longest common suffix search from the
morphological dictionary.  The way it works is that given a new word-form, it
can go through the dictionary to find a longest suffix it has in common with
some existing word-form. For example  given a new word-form \emph{googlella}
`with google', the system can guess that it might be like \emph{naḷ̣lella} `with
a teddy bear', or \emph{puolella} `with a half', and using the classification
then suggest the end user or corpus harvester to study forms like:
\emph{googlen, googlea, googleen, googlet, googlejen, \ldots} versus
\emph{googlen, googlta, googleen, googlet, googlten, \ldots}
to decide on the classification of the new word.

\section{Evaluation}
\label{sec:evaluation}

In this section we evaluate our analyser using the traditional approaches to
evaluating a morphological analyser. Firstly, we measure a naive coverage by
trying to analyse large corpora and see the proportion of word-forms that can
be analysed at all. Then we measure the quality of analysis as compared to
reference corpora of morphologically analysed and disambiguated texts. Finally
we measure the speed of our system in terms of throughput of the words per
second to give an impression of the tasks that the system might be suitable
for.

\subsection{Dictionary Coverage}

First we measure how many of the word-forms in the material are
out-of-vocabulary items. This gives us naive coverage, formally defined as
$\mathrm{Coverage} = \frac{\mathrm{Analysed}}{\mathrm{Corpus size}}$. For
coverage measurements we use three freely available corpora, the ebooks of
project Gutenberg, the data of Finnish Wikipedia, and the Europarl corpus. For
downloading and preprocessing the corpora we used scripts freely available from
an internet site~\footnote{\url{https://github.com/flammie/bash-corpora}}. The
results are presented in table~\ref{table:coverage}. 

\begin{table}
    \centering
    \begin{tabular}{|l|r|r|}
        \hline
        \bf System: & Omorfi & FINTWOL \\
        \hline
        \bf Corpus & & \\
        \hline
        Gutenberg & 98.3~\% & 93.0~\%   \\
        Wikipedia & 98.4~\% & 92.7~\% \\
        Europarl  & 99.3~\% & 93.6~\% \\
        \hline
    \end{tabular}
    \caption{Naive coverages when analysing common corpora
    \label{table:coverage}}
\end{table}

\subsection{Analysis Recall Compared to Old Systems}

The correctness of analysis was measured using the analyses of Finnish text
collection corpora~\citep{ftc}. From there we picked the \emph{Helsingin
sanomat} corpus from the year 1995. In context of morphological analysis,
recall is measured as a proportion of correct results in all results, or
formally $\mathrm{Recall} = \frac{\mathrm{Correct}}{\mathrm{Correct} +
\mathrm{Missing}}$. In table~\ref{table:quality} we show the results for our
system with random selection of the results, and then with the selection based
on unigram probabilities and estimated analysis weights. The reference corpus
is based on other system's output so we assume its results are around 100~\%
and have not shown it in the table.

\begin{table}
    \centering
    \begin{tabular}{|l|r|r|r|}
        \hline
        \bf Measure & Recall at 1 & Recall at 1---5 & Recall 1---$\infty$  \\
        \hline
        \bf Syatem & & & \\
        \hline
        Baseline & 26.3~\% & 76.7~\% & 90.5~\%\\
        With weights & 27.7~\% & 79.8~\% & 90.5~\%\\
        \hline
    \end{tabular}
    \caption{Quality of the analyser measured as equality to results of
        another analyser \label{table:quality}}
\end{table}

\subsection{Processing Speed}

To measure the processing speed we simply measured analysis time of 10,000,000
first word-forms of Finnish Wikipedia averaged over five runs. The results in
table~\ref{table:speed} are given in full word-forms per second rounded down.

\begin{table}
    \centering
    \begin{tabular}{|l|r|r|r|r|}
        \hline
        \bf System & Omorfi & fintwol & textmorfo & fifdg \\
        \hline
        \bf Corpus &        &         &           & \\
        \hline
        Wikipedia & 56,179  & 14,705  & 2,010 & 2,584 \\
        \hline
    \end{tabular}
    \caption{Quality of the analyser measured as equality to results of
        another analyser \label{table:speed}}
\end{table}

\section{Discussion}
\label{sec:discussion}

Morphological analysis of Finnish with computational methods and finite-state
automata is a 30 year old topic. In this paper we attempted to study the
similarities and differences that our current analyser has to those developed
in the years between. One feature that is new compared to most of the systems,
is the possibility of statistical and similar values. The resulting analyser is
currently capable of basic baseline quality morphological disambiguation, which
can be usable for information extraction and similar applications where the
accuracy requirement for top results is not so crucial. This should also be
usable as baseline for systems that perform morphological disambiguation based
on full syntactic analysis, like constraint or dependency
grammars~\citep{karlsson1995constraint} For future research, it should be
interesting to see more advanced statistical methods applied as well as
combinations with other rule-based systems..

To get the lexical data organised and the analyser implemented we have extended
the lexical classification using very strict, systematic and naive approach to
synchronic morphophonology. There are a couple of reasons for this meticulous
re-classification of paradigms. One is that by systematically resorting to
synchronic character level knowledge, we could get the most simple naive
morphological description that lends itself to automatic classification of out
of vocabulary words. For example, the longest suffix classification works quite
well in guessing the paradigms, however, if you are using the old
classification scheme, trying to find evidence that word belongs to class 1 by
the fact that we try to find illatives of form \emph{Vn} or inessives of form
\emph{ssA} is slightly harder task than finding the wordforms with suffix
exactly e.g.  \emph{on} and \emph{ssa}. To that aspect, the system is not
useful only for guessing, as the lexical data comes from various sources,
including freely editable dictionary, we can also use the automatic
classification schemes to ensure that the word data coming from such sources is
classified to our expectations.

One of the implications of the systematic classification schemes is that it
allows for implementations based on poorer computational morphology
implementations to be made. This means that the lexical data could be used in
systems without morphophonology support, such as
hunspell~\citep{tron2005hunmorph} or apertium~\citep{apertium}, which are
popular enough to be considered the de facto standard systems for computational
linguistics in fields of spell checking and machine translation respectively. 

The coverage of the analyser is systematically around 99~\%, this is virtually
at the upper limits of reasonable results with the given corpora. This can be
noticed by analysing the errors or the out-of-vocabulary word-forms left in the
current corpora. For Wikipedia, we get word-forms like: \emph{Lä}, \emph{of},
\emph{The}, \emph{amp}, \emph{the}, \emph{HMS}, \emph{New},
\emph{and}, \emph{jpg}, that is, mainly code notations, foreign words and
abbreviations. In the Gutenberg corpus, we get, among some missing proper
nouns, forms like: \emph{nämät}, \emph{kauvan}, \emph{sitte} and other
dialectal, spoken or otherwise non-standard forms. Of course these are all
plausible additions for morphological analyser as well.

The measurements of recall we used in the evaluation section are based on a
``gold standard'' which is merely a result of another automatic analyser. This
is not necessarily a representative of correctly analysed Finnish. For this
effect, it is better to read the precision and recall scores as
\emph{faithfulness} scores of our system being a re-implementation of the
reference system; for this reason, higher recall measures have not been
considered as a crucial design goal for Omorfi analyser.

The speed of our analyser is at the range where it is not of practical concern
for most end applications anymore, we can basically analyse quite large corpora
in few minutes of batch processing, and online processing is possible for
virtually all practical interactive applications.  It is also noteworthy that
the speed of our analyser is not directly comparable to \texttt{textmorfo} and
\texttt{fi-fdg}, which perform full morphosyntactic analysis, whereas Omorfi
only does isolated morphological analysis. It is possible to extend Omorfi with
some syntactic analysis capabilities using Karlsson's Finnish CG available as
free/open source implementation in University of Tromsa's open source
computational linguistics
repository~\footnote{\url{http://giellatekno.uit.no}}. The modifications
required to match the analyses with the system's expectations are minimal and
quite a bit of the work has already been done in the referred repository.


\section{Conclusion}
\label{sec:conclusion}

In this article we set out to show some features of the newest edition of our
Finnish morphological analyser. We have showed that it gives reasonably good
results for basic analysis tasks with much higher coverage, precision and recall
than other systems currently in use. We studied the usefulness of systematic
re-classification of Finnish inflectional paradigms based on synchronic
orthographical evidence, and found that it provides some benefits for tasks
like automatic classification and verification of new lexical data.

% apalike with underscores???
\bibliographystyle{apalike}
\bibliography{omorfi2013}

\end{document}
% vim: set spell:
