\documentclass{beamer}

\usepackage{fontspec}
\usepackage{polyglossia}

\usepackage{graphicx}
\usepackage{color}
\usepackage{url}

\usepackage{pifont}
\newcommand{\hand}{\ding{43}}

\title{Digital Humanities View on Machine Translation\\
\scriptsize{ESU-DH in Leipzig, 2018-07}}
\author{Tommi A Pirinen \scriptsize \guilsinglleft
tommi.antero.pirinen@uni-hamburg.de
\guilsinglright}
\institute{Hamburger Zentrum für Sprachkorpora, CLARIN-D}
\date{\today}

\begin{document}

\selectlanguage{english}

\maketitle

\begin{frame}{Introduction}
    Covered in this session:
    \begin{itemize}
        \item Who are we?
        \item What is (machine) translation
        \item Different forms of machine translation
        \item Relations between MT and digital humanities
    \end{itemize}
\end{frame}

\begin{frame}{Roll-call}
    \begin{itemize}
        \item Tommi A Pirinen, computational linguist, University of Hamburg
            working on language technology infra (CLARIN-D), and Uralic
            languages (Finnish, Hungarian, Estonian, Karelian, \ldots)
        \item you?
    \end{itemize}
\end{frame}

\begin{frame}{What is Translation}
    Prototypically:
    \begin{itemize}
        \item source text: 
            \textit{Hamburg\only<2->{$_1$} hat\only<2->{$_2$} die\only<2->{$_3$}
            höchste\only<2->{$_4$} Grundsteuer\only<2->{$_5$} in\only<2->{$_6$}
            Deutschland\only<2->{$_7$}} ---Hamburger Abendblatt 2018-07-09
        \item target text: 
            \textit{Hamburg\only<2>{$_1$} has\only<2>{$_2$} the\only<2>{$_3$}
            highest\only<2>{$_4$} property\only<2>{$_{5.1?}$} 
            tax\only<2>{$_{5.2?}$} in\only<2>{$_6$} Germany\only<2>{$_7$}}
        \item (or:
            \textit{Hampurin\only<3>{$_1$} kiinteistövero\only<3>{$_5$}
            on\only<3>{$_2$} Saksan\only<3>{$_7$} korkein\only<3>{$_4$}
            \only<3>{$0_3 0_6$}})
    \end{itemize}
\end{frame}

\begin{frame}{What is a Machine Translation}
    \begin{itemize}
        \item Computerised or computer-aided process of producing a
            (number of) translations from a source language (SL)
            text into a target language (TL)
    \end{itemize}
\end{frame}

\begin{frame}{What is MT used for (at the moment)}
The two main purposes:
    \begin{itemize}
        \item \emph{Assimilation}: to understand a text when you are not
            very competent in the source language
        \item \emph{Dissemination}: to create a rough draft for translators
            or editors
    \end{itemize}
\end{frame}

\begin{frame}{Wikipedia ContentTranslate}
      \includegraphics[width=1.0\textwidth]{captura-mediawiki.png}

\end{frame}

\begin{frame}{News}
    \begin{onlyenv}<1>
      \begin{center}
        \includegraphics[width=0.8\textwidth]{yle-uutiset-orig.png}
      \end{center}
    \end{onlyenv}
    \begin{onlyenv}<2>
      \begin{center}
        \includegraphics[width=0.8\textwidth]{yle-uutiset-apertium.png}
      \end{center}
    \end{onlyenv}

\end{frame}

\begin{frame}{DH Research data?}
    Not there yet, but it would be neat, see here:
    \includegraphics[width=1\textwidth]{flexpic.png}
\end{frame}

\begin{frame}{Dissemination vs. Assimilation}
    \centering
    \begin{tabular}{|l|p{.33\linewidth}|p{.33\linewidth}|}
        \hline
        \textbackslash & \bf Necessary & \bf Unnecessary \\
        \hline
        \bf Assimilation & 
        Understandability, fast translation & correct syntax, word-choices,
        predictable errors, happy translators\\
        \hline
        \bf Dissemination & reasonable syntax, predictable errors, high
        accuracy, happy translators & Fast translation, understandability \\
    \end{tabular}
    \only<1-2>{
    \alert<1>{With the binoculars the hat-having man sees the squirrel.} \\
    \alert<2>{The man wearing a hat sees the squirrel with the binoculars.}
}\only<3-4>{
    \alert<3>{The migration gave a great deal of criticism when it spoke out.}\\
    \alert<4>{The organisation received a great deal of criticism when it spoke out.}
}
\end{frame}

\begin{frame}{Types of Machine Translation Systems}
\begin{columns}[t]
  \begin{column}{0.5\textwidth}
   Rule-based\\
   \hrule
   dictionaries and rules\\
   \includegraphics[width=0.8\textwidth]{rbmt-overview.png}
  \end{column}
  \begin{column}{0.5\textwidth}
      Corpus-based (Statistical, Neural...)\\
   \hrule
    existing translations \\
    of sentences \\
   \includegraphics[width=0.8\textwidth]{smt-overview.png}
  \end{column}
\end{columns}


\end{frame}

\begin{frame}
  \frametitle{Rule-based machine translation}

\centering
\begin{tabular}{l|l}
\textbf{Strengths} & \textbf{Weaknesses}\\
\hline
+ Predictable output & - Lack of fluency \\
+ Predictable errors! & - Lack of idiomaticness \\
+ Incremental improvements & - ``Mechanical'' output \\ 
+ Translation errors traceable & - Development can be \\
+ Terminology control easy & \hfill time consuming \\
+ No need for large quantity & \\
\hfill of existing translations& \\
    + Creates dictionaries & \\
\hline
\end{tabular}


\end{frame}

\begin{frame}
  \frametitle{Corpus-based machine translation}

\centering
\begin{tabular}{l|l}
\textbf{Strengths} & \textbf{Weaknesses}\\
\hline
+ Fluent output & - Unpredictable \\
+ Idiomatic output & - Incremental improvements\\
                       & \hfill are hard \\
+ No need for linguistic & - Development can be \\
\hfill resources: & \hfill time consuming\\
\hfill - dictionaries & \\
\hfill- grammars & \\
\hfill- linguists  & \\
\hline
\end{tabular}

\end{frame}

\begin{frame}
  \frametitle{Corpus-based machine translation works best when...}

\begin{itemize}

  \item You have a big corpus of pre-translated and aligned sentences from one language to another --- or programmers
     who don't mind doing the alignment \\

~\\
  \item The language to be translated into is not morphologically complex --- and the language to be 
         translated from is more morphologically complex. \\
~\\
  \item The domain you want to translate is the same or similar as the one of your corpus. \\
~\\
  \item You lack linguists who are interested and motivated. \\
\end{itemize}

\end{frame}


\begin{frame}
  \frametitle{Rule-based machine translation works best when...} 

\begin{itemize}
  \item You don't have any pre-aligned corpora, or the pre-aligned corpora you have are bad. \\
~\\
  \item The languages to be translated are typologically similar. \\
~\\
  \item You are translating formal language. \\
~\\
  \item You have interested and motivated linguists. \texttt{:D} \\

\end{itemize}

\end{frame}

\begin{frame}
  \frametitle{Why do we work on rule-based machine translation ?}
Machine translation conferences are full of papers about corpus-based MT, so why  
work on rule-based MT ? \\
~\\ 
\begin{itemize} 
  \item Sometimes there are no corpora, or only rubbish corpora
  \item When we codify translation rules, it tells us something about language(s) and translation
  \item We can produce useful systems! -- really!
  \item Languages are interesting
  \item It's really fun!
\end{itemize}

\end{frame}


\begin{frame}
  \frametitle{Intermediate representation}

Another view on MT is with regards to internal representations; what
the source language text is turned into on the way to target language:

\begin{itemize}
  \item Direct translation: No intermediate representation
  \item Shallow: Morphology, some word-combinations
  \item Syntax transfer: Intermediate representation is a parse tree, or graph,
      or similar
  \item Semantic transfer: Intermediate representation are predicates with
      semantic rôles.
  \item Interlingua: As with semantic transfer, only the same 
     intermediate representation is shared by all languages / language pairs
\end{itemize} 


\end{frame}



\begin{frame}
  \frametitle{Vauquois}

\centering
\includegraphics[width=0.85\textwidth]{pyramid}

\end{frame}

\begin{frame}{Direct translation}
    Example here
    \begin{itemize}
        \item Typically word-to-word (or phrase to phrase or even letter
            to letter)
        \item No grammatical compnonent means no surprising translations
        \item good for glossing
        \item good for statistics and neural nets(!?), basically what is
            most used there
    \end{itemize}
\end{frame}

\begin{frame}{Shallow transfer}
    Example
    \begin{itemize}
        \item \textbf{main topic of this workshop}
        \item uses a bit of grammar, not too much (school-level)
        \item blah
    \end{itemize}
\end{frame}

\begin{frame}[fragile]
  \frametitle{Syntactic transfer}
\begin{center}
\end{center}
\begin{onlyenv}<1>
\includegraphics[width=0.75\textwidth]{dep-primer-0.png}
\end{onlyenv}
\begin{onlyenv}<2>
\includegraphics[width=0.75\textwidth]{dep-primer-1.png}
\end{onlyenv}
\begin{onlyenv}<3>
\includegraphics[width=0.75\textwidth]{dep-primer-2.png}
\end{onlyenv}
\begin{onlyenv}<4>
\includegraphics[width=0.75\textwidth]{dep-primer-3.png}
\end{onlyenv}
\begin{onlyenv}<5>
\includegraphics[width=0.75\textwidth]{dep-primer-4.png}
\end{onlyenv}
\begin{onlyenv}<6>
\includegraphics[width=0.75\textwidth]{dep-primer-5.png}
\end{onlyenv}
\begin{onlyenv}<7->
\includegraphics[width=0.75\textwidth]{dep-primer-6.png}
\end{onlyenv}
\begin{onlyenv}<7-8>
~\\
~\\

\begin{verbatim}
Heinrich antoi lihapalan koiralleen.
\end{verbatim}
\end{onlyenv}
\begin{onlyenv}<9>
\begin{verbatim}
Heinrich antoi lihapalan koiralleen.
\end{verbatim}
\begin{itemize}
  \item Performance depends on quality of analysis
  \item Bad analysis can lead to surprising translations
\end{itemize}
\end{onlyenv}



\end{frame}



\begin{frame}[fragile]
  \frametitle{Semantic translation}
``Itziar likes flowers a lot.''
\begin{verbatim}
Itziar pitää paljon kukista.
\end{verbatim}

\begin{onlyenv}<1>
\begin{verbatim}
pitää[                      magen[
  subject=[Itziar],            subject=[Itziar],
  adv=[kukista],               object=[Blumen],
  adv=[paljon]                 adv=[sehr gern]
]                            ]
\end{verbatim}
\end{onlyenv}
\begin{onlyenv}<2->
\begin{small}
\begin{verbatim}
pitää[                      magen[
  experiencer=[Itziar],        experiencer=[Itziar],
  theme=[kukista],             theme=[Blumen],
  quantification=[paljon]      quantification=[sehr gern]
]                            ]
\end{verbatim}
\end{small}
\end{onlyenv}


\begin{onlyenv}<3->
\begin{verbatim}
Itziar mag Blumen sehr gern.
\end{verbatim}
\end{onlyenv}

\begin{onlyenv}<4->
\begin{itemize}
  \item Obtaining a reliable semantic parse automatically is hard
  \item Very data intensive
\end{itemize}
\end{onlyenv}

\end{frame}


\begin{frame}
  \frametitle{Interlingua}

Interlingual MT works like semantic transfer, the difference being 
that the semantic representation
is the same for all languages.

\begin{itemize}
  \item This only really works for limited domains
  \item Designing an intermediate representation that is valid for 
all the languages of the world --
    or even for a subset is a hard task
  \item But, if your domain is small, and you need high quality in 
many languages, then it can
    still be a feasible option
\end{itemize}

\begin{onlyenv}<2>
  The only wide-coverage interlingual system I know of (GF) has bac
k-off to 
   syntactic and direct translation.
\end{onlyenv}


\end{frame}

\begin{frame}
  \frametitle{Some systems that you may have heard of\ldots}

\begin{center}
\includegraphics[width=0.76\textwidth]{pyramid.png}
\end{center}

\end{frame}

\begin{frame}{Reality}
    \begin{itemize}
        \item Systems aren't so clear-cut Corpus-based---rule-based or
            direct---shallow.
        \item 
    \end{itemize}
\end{frame}

\begin{frame} %% framesection
 \begin{center}
 {\Large {\bf Problems in rule-based machine translation}}
 \end{center}
\end{frame}

\begin{frame}
  \frametitle{Analysis}

\begin{center}
{\em Form does not entirely determine content.}
\end{center}
Many sentences in natural language can have more than one interpretation, and these 
interpretations may be translated differently in different languages.

\begin{itemize}
  \item {\em ¿Tuovatko he uutisia Kreikasta?} -- {\sc theme} `about' or {\sc source} `from'?
  \begin{itemize}
    \item Are they bringing news from Greece?
    \item Are they bringing news about Greece?
  \end{itemize}
  \item however, if you translate from Finnish to another Uralic  languages,
      chances are the ambiguity carries over
\end{itemize}

The machine only knows as much as you can explain to it.

\end{frame}

\begin{frame}
  \frametitle{Synthesis}

\begin{center}
{\em Content does not entirely determine form.}
\end{center}

A single meaning can be expressed in more than one way. A
single sentence may have many adequate equivalents.

\begin{itemize}
\item Hampurin kinteistövero on Saksan korkein
\item Hampurissa on Saksan korkein kiinteistövero
\item Hampurilla on korkein kiinteistövero Saksassa
\end{itemize}

\end{frame}
\begin{frame}
  \frametitle{Transfer}

\begin{center}
{\em The same content is represented differently in different languages.}

\end{center}

Languages differ how they express a particular meaning. Some
languages encode facets of meaning which are not encoded by
others.

\begin{itemize}
  \item Definiteness
  \item Gender
  \item Direction and class of movement
  \item ...
\end{itemize}

\begin{itemize}
\item {\em hän on lääkäri.}
\begin{itemize}
  \item hän = he/she/they, on = is, lääkäri = doctor
\item {\em he / she /they  is / are a doctor.}
\item {\em er / sie ist ein Doktor/in.}
\end{itemize}
\end{itemize}
\begin{itemize}
\item I like it
    \begin{itemize}
        \item jag gillar det
        \item es gefällt mir
        \item ne meeldivat mind
        \item minä pidän siitä
    \end{itemize}
\end{itemize}


\end{frame}

\begin{frame}{Next session}
Look at apertium specifically and install environments / hands-on.
Check moodle for instructions. Or wiki.apertium.org.
\end{frame}

\end{document}
