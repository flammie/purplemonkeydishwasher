\documentclass[10pt,xetex]{beamer} %[compress, blue]
\mode<presentation>

\usepackage{helsingfors}

\date{2nd--13th November, 2015}
\title{Session 6: Chunk transfer}
\begin{document}

\begin{frame}
        \titlepage
\MyLogoBottomCentred
\end{frame}

\logo{\includegraphics[height=1.6cm]{../logo/logo}}

\begin{frame}
\frametitle{Chunk transfer} % Advanced transfer

Chunk transfer: 3 transfer submodules:
\newline


%\includegraphics[scale=0.32]{dibuix-transfer3.png}
\begin{center}
\includegraphics[scale=0.45]{dibuix-transfer-moduls-1.pdf}
\end{center}

\begin{tabular}{lccc}
~~~~~~~~~~~~~~~~~~~~~~~~~  &t1x & t2x & ~~ t3x \\
~~~~~~~~~~~~~~~~~~~~~~~~~  &LRLM & LRLM & ~~ \\
~~~~~~~~~~~~~~~~~~~~~~~~~  &patterns & patterns of & ~~ chunks \\
~~~~~~~~~~~~~~~~~~~~~~~~~ & ~~ & chunks & ~~ chunks

\end{tabular}
\newline
\newline
\newline
\newline
Chunker and interchunk have the same \alert{file structure}.
Chunker is like the transfer module in the single level transfer.
Main difference: 
\begin{itemize}
\item Transfer outputs \alert{lexical units}
\item Chunker outputs \alert{chunks}, to be processed by the interchunk
\end{itemize}
\end{frame}

\begin{frame}
  \frametitle{Chunks}

Chunks = group of words that correspond roughly to a phrase. \\
~\\

\begin{small}
\begin{tabular}{llll}
{\bf Input pattern} & 	{\bf Example} & 	{\bf Output chunk} &	{\bf Example} \\
\hline
nom & 	{\em ҫурт} &	SN$\{$nom$\}$ &	дом \\
adj nom & 	{\em хитре ҫурт} & 	SN$\{$nom adj$\}$ &	красивый дом \\
det nom &	{\em манăн ҫурт} &	SN$\{$det nom$\}$ &	мой дом\\
num nom &	{\em икĕ ҫурт} &	SN$\{$num nom$\}$ &	два дома\\
adj nom &	{\em хитре ҫуртсем} & 	SN$\{$adj nom$\}$ &	красивые домы\\
adv adj nom &	{\em питĕ хитре ҫурт} & 	SN$\{$adv adj nom$\}$ &	очень красивый дом\\
%num adv adj nom &	{\em пилĕ питĕ хитре ҫурт} &	SN$\{$num adv adj nom$\}$ &	пять очень красивых домов \\
\hline
verb &	{\em вулать} 	& V$\{$verb$\}$  &	читает \\
verb &	{\em вуламасть} &	V$\{$neg\_{}adv verb$\}$  &	не читает  \\
adv verb &	{\em ан вула !} & 	V$\{$adv verb$\}$ &	не читай ! \\
ger verb &	{\em вулама пуçлать} & 	V$\{$verb inf$\}$ 	&начинает читать \\
\hline
\end{tabular}
\end{small}

\end{frame}

\begin{frame}
The creation of chunks allows for a more powerful treatment of syntactic/structural divergences:

\begin{exampleblock}{Example} % Example
\begin{tabular}{l|l|l}
interchunk rule: & NP & V\\
\hline
possible chunks: & noun & verb\\
& det noun & aux verb\\
& det adj noun & adv verb\\
& adj noun & aux adv verb \\
& adj adj noun & aux aux verb\\
& adj conj adj noun & ...\\
& ... &\\


\end{tabular}
\end{exampleblock}

\end{frame}


\begin{frame}{The chunker}

The \textbf{chunker} is like the \textbf{transfer} module of the single-level transfer, being the main difference:
\begin{itemize}
\item The transfer outputs \alert{lexical units}
\item The chunker outputs \alert{chunks}
\end{itemize}
Structure of a chunk:
\begin{itemize}
\item Chunk = chunkname<tags>\{chunk\_content\}


\item In the chunker, lexical units are translated into the target language


\end{itemize}
Example input (from ``chunker''):
\begin{exampleblock}{}
\emph{``Анна -- мугалим'' (ky $\rightarrow$ ru)} 
\begin{small}
\texttt{\^{}n<SN><DEF>{\color{blue}{<f>}}<aa>{\color{red}{<sg>}}{\color{green}{<nom>}}\{\^{}Анна<np><ant>{\color{blue}{<3>}}<aa>{\color{red}{<5>}{\color{green}{<6>}}}\$\}\$} \\
\texttt{\^{}cop<Vcop><guio>\{\^{}-<guio>\$\}\$} \\
\texttt{\^{}n<SN><IND>{\color{blue}{<GD>}}<aa>{\color{red}{<ND>}}{\color{green}{<nom>}}\{\^{}преподаватель<n>{\color{blue}{<3>}}<aa>{\color{red}{<5>}{\color{green}{<6>}}}\$\}\$}
\end{small}

\end{exampleblock}


%\end{itemize}

\end{frame}


\begin{frame}
\frametitle{The interchunk module}


The interchunk module:
\begin{itemize}
\item detects \alert{patterns of chunks} in the same way that the chunker detects patterns of words
\item applies the necessary transformations to these patterns: agreement, reorderings, syntactic or morphological changes
\\[5pt]
\end{itemize}
Example output:
\begin{exampleblock}{}
\begin{small}
\emph{``Анна -- мугалим'' (ky $\rightarrow$ ru)} 
\begin{small}
\texttt{\^{}n<SN><DEF>{\color{blue}{<f>}}<aa>{\color{red}{<sg>}}{\color{green}{<nom>}}\{\^{}Анна<np><ant>{\color{blue}{<3>}}<aa>{\color{red}{<5>}{\color{green}{<6>}}}\$\}\$} \\
\texttt{\^{}cop<Vcop><guio>\{\^{}-<guio>\$\}\$} \\
\texttt{\^{}n<SN><IND>{\color{blue}{<f>}}<aa>{\color{red}{<sg>}}{\color{green}{<nom>}}\{\^{}преподаватель<n>{\color{blue}{<3>}}<aa>{\color{red}{<5>}{\color{green}{<6>}}}\$\}\$}
\end{small}

\end{small}

\end{exampleblock}
The main difference with the chunker rules: the <clip> element in the rules does not have a \emph{side} attribute:
\begin{itemize}
\item making chunks (chunker): \texttt{<clip pos="2" side="tl" part="lem"/>}\\
\item working between chunks (interchunk): \texttt{<clip pos="2" part="lem"/>}\\

\end{itemize}

\end{frame}

\begin{frame}
%\frametitle{Interchunk}

\begin{exampleblock}{Examples of interchunk operations (adapted from \texttt{en-es})}
{\smallersans
\begin{itemize}
  \item Agreement SN-SA:
  \begin{itemize}
    \item \emph{Идея жаман эмес} $\rightarrow$ \emph{Идея не плохая}
    \item Rule: SN SA POSTADV, SN ADV SA...
  \end{itemize}
  \item Agreement SN-SN:
  \begin{itemize}
    \item \emph{Анна -- мугалим} $\rightarrow$ \emph{Анна -- преподавательница}
    \item Rule: SN Vcop SN, SN Vcop ADV SN...
   \end{itemize}
   \item Agreement SN-SV:
   \begin{itemize}
     \item \emph{Адам келди} $\rightarrow$ \emph{Мужчинa пришёл}
     \item \emph{Адамдар келди} $\rightarrow$ \emph{Мужчины пришли}
     \item Rule: SN SV, SN ADV SV, ...
    \end{itemize}
   \item Deletion of subject pronouns:
   \begin{itemize}
      \item \emph{Мы не пошли} $\rightarrow$ \emph{No fuimos}
    \end{itemize}
    \item Syntactic changes:
    \begin{itemize}
       \item \emph{Им нравятся торты.} $\rightarrow$ \emph{Алар тортту жакшы көрүшөт.}
    \end{itemize}
    \item Generation of articles:
    \begin{itemize}
      \item \emph{Кече балдар\textbf{ды} көрдүм.} $\rightarrow$ \emph{Ahir vaig veure \textbf{els} nens.}
      \item ~~~~~~~~~~~~~~~~~~~~~~~~~~~~~~~~~~~~~~~~~~ $\rightarrow$ \emph{Hier j'ai vu \textbf{les} enfants.}
    \end{itemize}
\end{itemize}
}
%\texttt{\^{}Nom<SN><PDET><f><sg>\{\^{}Maria<np><ant><3><4>\$\}\$\\
%\^{}be<Vcop><vbser><pri><p3><sg>\{\^{}ser<vbser><3><4><5>\$\}\$ \\
%\^{}det\_nom<SN><DET>{\color{red}{<f><sg>}}\{\^{}el<det><def><3><4>\$ \^{}profesor<n><3><4>\$\}\$}


\end{exampleblock}

\end{frame}

\begin{frame}
\frametitle{Взаимодействие между модулями /3}

\begin{exampleblock}{}
\begin{small}
  <rule comment="REGLA: SN -- SN">\\
  ~~  <pattern>\\
  ~~~~	<pattern-item n="SN"/>\\
  ~~~~	<pattern-item n="vbcop"/>\\
  ~~~~	<pattern-item n="SN"/>\\
  ~~ </pattern>\\
  ~~ <action>\\
  ~~ (...)\\
  ~~~	<out>\\
  ~~~~	  <chunk>\\
  ~~~~~	    <clip pos="1" part="whole"/>\\
  ~~~~	  </chunk>\\
  ~~~~	  <b pos="1"/>\\
  ~~~~	  <chunk>\\
  ~~~~~	    <clip pos="2" part="whole"/>\\
  ~~~~	  </chunk>\\
(...)
\end{small}
\end{exampleblock}
\end{frame}


\begin{frame}
\frametitle{The postchunk module}

The postchunk module:
\begin{itemize}
\item substitutes numbers in tags by the referenced values

\item removes the chunk lemma and tags
\item outputs the sequence of lexical units
\item can operate only on a single chunk at a time
\item can perform some final operations on the chunk content
\newline

\end{itemize}

% Example output:
\begin{exampleblock}{}
\emph{``Анна -- мугалим'' (ky $\rightarrow$ ru)}\\
\texttt{\^{}Анна<np><ant>{\color{blue}{<f>}}<aa>{\color{red}{<sg>}}{\color{green}{<nom>}}\$} \\
\texttt{\^{}-<guio>\$} \\
\texttt{\^{}преподаватель<n>{\color{blue}{<f>}}<aa>{\color{red}{<sg>}}{\color{green}{<nom>}}\$} \\

\end{exampleblock}

Output of generation: {\em Анна -- преподавательница}
\end{frame}


\begin{frame}
\frametitle{Looking at three-stage transfer}

\textbf{Input text}: \emph{``Жөргөмүштөр сулуу.''}
\\[9pt]

\textbf{apertium -d . kir-rus-pretransfer}\\
\texttt{\^{}Жөргөмүш<n><pl><nom>\$ \^{}сулуу<adj>\$}
\newline

\textbf{apertium -d . kir-rus-chunker}\\
\texttt{\^{}Nom<SN>{\color{blue}{<m>}}<nn>{\color{red}{<pl>}}{\color{green}{<nom>}}\{\^{}паук<n>{\color{blue}{<2>}}<nn>{\color{red}{<4>}}{\color{green}{<5>}}\$\}\$\\
 \^{}adj<SA>{\color{blue}{<GD>}}{\color{red}{<ND>}}{\color{green}{<CD>}}\{\^{}красивый<adj>{\color{magenta}{<short>}}{\color{blue}{<2>}}{\color{red}{<3>}{\color{green}{<4>}}}\$\}\$}
\newline

\textbf{apertium -d . kir-rus-interchunk}\\
\texttt{\^{}Nom<SN>{\color{blue}{<m>}}<nn>{\color{red}{<pl>}}{\color{green}{<nom>}}\{\^{}паук<n>{\color{blue}{<2>}}<nn>{\color{red}{<4>}}{\color{green}{<5>}}\$\}\$\\
 \^{}adj<SA>{\color{blue}{<mfn>}}{\color{red}{<pl>}}{\color{green}{<nom>}}\{\^{}красивый<adj><short>{\color{blue}{<2>}}{\color{red}{<3>}{\color{green}{<4>}}}\$\}\$}
\newline

\textbf{apertium -d . kir-rus-postchunk}\\
\texttt{\^{}Паук<n>{\color{red}{<m>}}<nn>{\color{red}{<pl><nom>}}\$}\\
\texttt{\^{}красивый<adj><short>{\color{red}{<mfn><pl><nom>}}\$} \\
~\\
\textbf{Output text:} ``\emph{Пауки красивы.}'' % Output text

\end{frame}


%\begin{frame}



%\end{frame}


\end{document}
