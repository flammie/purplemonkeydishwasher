\documentclass[10pt,xetex]{beamer} %[compress, blue]
\mode<presentation>

\usepackage{multicol}


\date{\today}
\title{Lexical transfer and lexical selection}

\begin{document}


\begin{frame}
  \frametitle{Introduction}

This session will be structured in two sections:

\begin{itemize}
  \item {\bf Lexical transfer}: Translating words and marking them for structural transfer
~\\
~\\
  \item {\bf Lexical selection}: Choosing a translation of a polysemous word depending on context
\end{itemize}

\end{frame}

\begin{frame}

\begin{center}
\includegraphics[width=\textwidth]{architecture-4.pdf}
\end{center}

\end{frame}

\begin{frame}

\begin{block}{Transfer}
The transfer module is where the \textbf{magic} happens: the intermediate
representation in source language (SL) is converted into an
intermediate representation in target language (TL).

\end{block}
~\\

Transfer in Apertium consists of two submodules:
~\\
\begin{itemize}
  \item {\bf Lexical transfer}
  \begin{itemize}
\item returns all the possible translations in TL for an SL word;
\item marks some lexical features which will be used by the structural transfer.
  \end{itemize}
  \item {\bf Structural transfer}
  \begin{itemize}
    \item performs syntactic operations involving groups of words (see next sessions).
  \end{itemize}
\end{itemize}

\end{frame}

\begin{frame}
  \frametitle{Lexical transfer}

\begin{block}{}

The lexical transfer module reads each SL lexical form and delivers the
corresponding TL lexical forms by looking the SL word up in a bilingual dictionary.
\end{block}
\begin{block}{Bilingual dictionary}
\begin{itemize}
  \item No surface forms in this stage: input and output are lexical forms consisting of lemma, part-of-speech and inflection information.
  \item The dictionary contains a list of equivalent lexical forms.
  \item A single bilingual dictionary is used for both directions of translation.
  \item XML syntax similar (but simpler) to monolingual dictionaries.
  \item Paradigms are usually not necessary.
\end{itemize}
\end{block}
\end{frame}

\begin{frame}[fragile]
 \frametitle{Lexical transfer}

It should be a fairly simple task!

\begin{center}
 {\bf треугольник $\leftrightarrow$ трикутник}
\end{center}
~\\

\begin{verbatim}

(ru)                          (uk)
треугольник<n><m><sg><nom> ←→ трикутник<n><m><sg><nom>
треугольник<n><m><sg><gen> ←→ трикутник<n><m><sg><gen>
треугольник<n><m><sg><dat> ←→ трикутник<n><m><sg><dat>
треугольник<n><m><sg><acc> ←→ трикутник<n><m><sg><acc>
треугольник<n><m><sg><ins> ←→ трикутник<n><m><sg><ins>
треугольник<n><m><pl><nom> ←→ трикутник<n><m><pl><nom>
...

\end{verbatim}
\end{frame}

\begin{frame}
  \frametitle{Bilingual dictionary}

So, word correspondences in the bilingual dictionary use a shorter representation

\begin{block}{}

Only lemma and part-of-speech are mandatory if the rest of tags do not change:
\begin{itemize}
 \item треугольник<n> - трикутник<n>
\end{itemize}
\end{block}
\begin{block}{XML encoding in the bilingual dictionary}
~\\
\begin{center}
\begin{small}
{\tt  <e><p><l>треугольник<s n="n"/></l><r>трикутник<s n="n"/></r></p></e>}
\end{small}
\end{center}
~\\
Note: More often, some extra information (e.g. gender) is added, regardless of if it changes
~\\

\begin{center}
\begin{small}
{\tt  <e><p><l>треугольник<s n="n"/><s n="m"/><s n="nn"/></l> \\
      ~~~~~~~~~~<r>трикутник<s n="n"/><s n="m"/><s n="nn"/></r></p></e>}
\end{small}
\end{center}

\end{block}
\end{frame}

\begin{frame}
\frametitle{Change of gender}

\begin{block}{}
Only the tags until the last change need to be indicated:

\begin{center}
государство<n><nt> - держава<n><f>
\end{center}
\end{block}
\begin{block}{XML encoding in the bilingual dictionary}
{\tt <e><p><l>государство<s n="n"/><s n="nt"/></l><r>держава<s n="n"/><s n="f"/></r></p></e>}
\end{block}

\end{frame}

\begin{frame}
  \frametitle{Lexical ambiguity}

Real life is a little bit more complex, as SL words may have more than one equivalent in TL.

\begin{block}{Homonymy}
\begin{itemize}
\item A surface form with more than one possible morphological analysis:
\item Кыргызский {\em жаз} (noun or verb) translates into русский {\em весна} (noun) or {\em писать} (verb).
\item The part-of-speech disambiguator has already selected one single alternative (recall previous session).
\end{itemize}
\end{block}
\begin{block}{Polysemy}% (Многозначность)}
A lemma and part-of-speech that have several meanings:\\ Chuvash «{\em тӳпе}» (noun) translates into \\
Russian «{\em вершина}», «{\em крыша}» and «{\em небо}».
\end{block}
\end{frame}

\begin{frame}
  \frametitle{Polysemy}% (Многозначность)}

\begin{itemize}
\item Polysemy needs to be addressed in the lexical transfer or lexical selection modules when different meanings are translated in different ways; for instance, {\em тӳпе} in the Chuvash--Russian bilingual dictionary.
\item This is called {\bf lexical selection}.
\item {\bf Free-rides} do not pose any problem: Russian {\em ручка} can be translated with {\em ручка} when referring to both дверная ручка `door handle' and шариковая ручка (writing instrument).
\end{itemize}

%PICS HERE

\end{frame}

\begin{frame}
  \frametitle{Examples of bilingual dictionary entries}%How to add bidix errors

\begin{block}{Examples of bilingual dictionary entries}
Let's add some entries to the Chuvash-Russian bilingual dictionary!
\end{block}
\end{frame}

\begin{frame}
 \frametitle{Adding entries to the bilingual dictionary}

\begin{onlyenv}<1>
\begin{center}
тӳпе $\leftrightarrow$ вершина `peak, top'
\end{center}
\end{onlyenv}
\begin{onlyenv}<2>
{\tt <e><p><l>тӳпе<s n="n"/></l><r>вершина<s n="n"/><s n="f"/><s n="nn"/></r></p></e>}

\end{onlyenv}

\end{frame}


\begin{frame}
 \frametitle{Adding entries to the bilingual dictionary}

\begin{onlyenv}<1>
\begin{center}
тăрă $\rightarrow$ вершина
\end{center}
\end{onlyenv}
\begin{onlyenv}<2>
{\tt <e><p><l>тӳпе<s n="n"/></l><r>вершина<s n="n"/><s n="f"/><s n="nn"/></r></p></e>}\\
{\tt <e><p><l>тăрă<s n="n"/></l><r>вершина<s n="n"/><s n="f"/><s n="nn"/></r></p></e>} \\

\end{onlyenv}

\end{frame}

\begin{frame}
 \frametitle{Adding entries to the bilingual dictionary}

\begin{onlyenv}<1>
\begin{center}
тӳпе $\leftarrow$ небо `sky'
\end{center}
\end{onlyenv}
\begin{onlyenv}<2>
{\tt <e><p><l>тӳпе<s n="n"/></l><r>вершина<s n="n"/><s n="f"/><s n="nn"/></r></p></e>} \\
{\tt <e><p><l>тăрă<s n="n"/></l><r>вершина<s n="n"/><s n="f"/><s n="nn"/></r></p></e>}  \\
{\tt <e><p><l>тӳпе<s n="n"/></l><r>небо<s n="n"/><s n="nt"/><s n="nn"/></r></p></e>} \\

\end{onlyenv}

\end{frame}

\begin{frame}
 \frametitle{Adding entries to the bilingual dictionary}

\begin{onlyenv}<1>
\begin{center}
тӳпе $\leftarrow$ крыша `roof'
\end{center}
\end{onlyenv}
\begin{onlyenv}<2>
{\tt <e><p><l>тӳпе<s n="n"/></l><r>вершина<s n="n"/><s n="f"/><s n="nn"/></r></p></e>} \\
{\tt <e><p><l>тăрă<s n="n"/></l><r>вершина<s n="n"/><s n="f"/><s n="nn"/></r></p></e>}  \\
{\tt <e><p><l>тӳпе<s n="n"/></l><r>небо<s n="n"/><s n="nt"/><s n="nn"/></r></p></e>}\\
{\tt <e><p><l>тӳпе<s n="n"/></l><r>крыша<s n="n"/><s n="f"/><s n="nn"/></r></p></e>} \\

\end{onlyenv}

\end{frame}


\begin{frame}
  \frametitle{Dealing with polysemy: Multiwords}

\begin{block}{}
One way we can deal with lexical selection of polysemous terms by using
multiwords:


{\tt ручка<n>  $\leftrightarrow$ ручка<n>} \\
{\tt дверная ручка<n> $\leftrightarrow$ авăр<n>}

\end{block}

\begin{block}{}
Another way is to use the lexical selection module. More on that shortly.
\end{block}
\end{frame}

\begin{frame}
  \frametitle{What not to translate}

Sometimes, we might have a word that is difficult to translate: {\em јаничарство}, {\em laufabrauð},  {\em abertzale}.

\begin{itemize}
  \item Maybe the word does not appear in the dictionary
  \item Maybe the ``translation'' in the dictionary is actually a description
  \begin{itemize}
    \item e.g., {\em laufabrauð} $\rightarrow$ ``traditional Icelandic deep-fried patterned Christmas wafer''
  \end{itemize}
\end{itemize}

%Some things you could try:

\begin{itemize}
  \item Find a more literal translation in the target language, e.g. ``laufa'' + ``brauð'' = ``leaf'' + ``bread''.
   \item Check in a search engine, or one of the image search engines
   \item  Not appearing in dictionary != bad translation.
\end{itemize}
Example: {\em Börnin bökuðu laufabrauð}

\begin{itemize}
\item The children baked leafbread.
\item The children baked laufabrauð.
\item The children baked traditional Icelandic deep-fried patterned Christmas wafer.
\end{itemize}

For assimilation, people can look up the word. For dissemination, they don't want to replace 7 words.
\end{frame}

\begin{frame}[fragile]

In the Chuvash--Turkish dictionary:

\begin{verbatim}
<e> \\
~<p> \\
~~~~<l>диссертант<s n="n"/></l>  \\
~~~~<r>bilim<b/>adamı<b/>ünvanını<b/>almak<b/>için<b/>tez<b/>yazan<b/>ve<b/>savunan<b/>kişi<s n="n"/></r> \\
~</p> \\
</e> \\
\end{verbatim}

Can anyone come up with a better translation ?

\end{frame}

\begin{frame}
  \frametitle{Marking lexical features for the structural transfer}

\begin{itemize}
\item The lexical transfer also marks some lexical features which will be used by the structural transfer.
\end{itemize}

\begin{block}{}
\begin{itemize}
\item For instance, a noun with the same surface form for its two genders.
\item Russian monolingual dictionary:
\begin{itemize}
  \item коллега → коллега<n><m><aa><sg><nom>
  \item коллеги → коллега<n><m><aa><pl><nom>
\end{itemize}
\item The structural transfer will choose the gender by looking at the surrounding context.
\item The lexical transfer simply marks this issue with the tag {\tt GD}.
\end{itemize}
\end{block}
\begin{itemize}
\item Similar things hold for number ({\tt ND}).
\end{itemize}

\end{frame}

\begin{frame}
  \frametitle{Marking lexical features for the structural transfer}
\begin{block}{}
<e r="LR"><p>
<l>колега<s n="n"/><s n="m"/></l>
<r>коллега<s n="n"/><s n="mf"/></r>
</p></e> \\
<e r="LR"><p>
<l>колега<s n="n"/><s n="f"/></l>
<r>коллега<s n="n"/><s n="mf"/></r>
</p></e> \\
<e r="RL"><p>
<l>колега<s n="n"/><s n="GD"/></l>
<r>коллега<s n="n"/><s n="mf"/></r> </p></e> \\
\end{block}
\end{frame}


\begin{frame}
  \frametitle{Some useful commands}

\begin{block}{Compiling a bilingual dictionary}
{\tt \$ lt-comp lr apertium-chv-rus.chv-rus.dix chv-rus.autobil.bin} \\
{\tt \$ lt-comp rl apertium-chv-rus.chv-rus.dix ru-cv.autobil.bin}
\end{block}
\begin{block}{Testing the lexical transfer}
{\tt \$ echo "\^{}тӳпе<n><nom>\$" | lt-proc -b chv-rus.autobil.bin }\\
{\tt \^{}тӳпе<n><nom>/небо<n><nt><nn><nom>/ крыша<n><f><nn><nom>/вершина<n><f><nn><nom>\$ }

\end{block}

\end{frame}


%%%%%%%%%%%%%%%%%%%%%%%%%%%%%%%%%%%%%%%%%%%%%%%%%%%%%%%%%%%%%%%%%%%%%%%%%%%%%%%
%%
%%%%%%%%%%%%%%%%%%%%%%%%%%%%%%%%%%%%%%%%%%%%%%%%%%%%%%%%%%%%%%%%%%%%%%%%%%%%%%%

\begin{frame}

\begin{center}
\includegraphics[width=\textwidth]{architecture-5.pdf}
\end{center}

\end{frame}

\begin{frame}
\frametitle{What is lexical selection?}

\begin{center}
  La clima és tropical amb {\bf estació humida} i {\bf estació} seca.
\end{center}

\begin{itemize}
  \item {\bf What ?}
  \begin{itemize}
    \item Choosing the most adequate target language (TL) translation
        for words with the same POS and $>$ 1 translation
    \item Why is it different from word-sense disambiguation (WSD) ?
    \begin{itemize}
      \item We don't care if we get the right sense if we get the
        right translation.
    \end{itemize}
  \end{itemize}

\end{itemize}

\end{frame}

\begin{frame}
  \frametitle{Lexical selection in Apertium}

	\begin{block}{Lexical selection in Apertium}
\begin{itemize}
\item    Currently, in most language pairs, Apertium does not deal explicitly with polysemy (no lexical selection is performed).
\begin{itemize}
  \item There is however a lexical-selection module (now two years old!)
\end{itemize}
\item    A SL lexical form can have more than one TL equivalent in the
    dictionary (more on this later) -- but only one can be selected in the final translation.
\item    We will include in the bilingual dictionary the translation of the
    most frequent meaning.
			\begin{itemize}
\item    This makes bilingual dictionaries domain-dependent.
			\end{itemize}
\end{itemize}
	\end{block}

\end{frame}



\begin{frame}
\frametitle{Rules are cool}

\begin{itemize}
  \item What are lexical selection rules ?
  \begin{itemize}
    \item A {\bf source word}: The word you want to translate
    \item A {\bf target word}: The word you want to translate it to
    \item A {\bf context}: Set of pairs of features and positions
    \item An {\bf operation}: Either to {\tt select} a word you want to translate to, or {\tt remove} a word you definitely don't want to translate to
    \begin{itemize}
      \item  Position: The relative position where the {\bf feature} should be found (free drugs != drug free)
      \item  Feature: A feature is either a surface form, a lemma, a tag (morphological, syntactic, semantic)
    \end{itemize}
  \end{itemize}
~\\

  \item Why are rules good ?
  \begin{itemize}
    \item Human-editable
    \item Capable of being iteratively developed
    \item Deterministic
    \item Fun to write
  \end{itemize}
\end{itemize}

\end{frame}

\begin{frame}

\frametitle{Rule examples}

Not all words are disambiguated with the same information:

\begin{onlyenv}<1>
	\begin{itemize}
	  \item {\bf Lexical}: {\em règim} $\rightarrow$ $\{$ 1: диета, 2: режим, \ldots $\}$
	  \begin{itemize}
	    \item el règim$_1$ $\{$ artificial·i·mèdic, vegetarià, alimentari, d'·herbes $\}$
	    \item el règim$_2$ $\{$ nazi, totalitari, feixista, franquista, militar, fiscal $\}$
	  \end{itemize}
   \end{itemize}
\end{onlyenv}


\begin{onlyenv}<2>
	\begin{itemize}
	  \item {\bf Morphological}: {\em pisu} $\rightarrow$ $\{$ 1: весь, 2: етаж, \ldots $\}$
	  \begin{itemize}
	    \item pisua$_1$ {\sc num} `Bere pisua 1 eta 2 kilo artekoa da.'
	    \item {\sc num.ord} pisua$_2$ `Hirugarren pisuan...'
	  \end{itemize}
	\end{itemize}
\end{onlyenv}

\begin{onlyenv}<3>
	\begin{itemize}
	  \item {\bf Syntactic}: {\em hil} $\rightarrow$ $\{$ 1: убить, 2: умереть, \ldots $\}$
	  \begin{itemize}
	    \item {\sc @subj} {\sc @obj} hil$_1$ `Mikelek bera hiltzea nahi du.' %% `Mikelek ni hilko du.'
	    \item {\sc @subj} hil$_2$ `Jainkoa hiltzea nahi du.' %% Jainkoa hil da. %% zu etxean hiltzea nahi dut
	  \end{itemize}
   \end{itemize}
\end{onlyenv}

\begin{onlyenv}<4->
	\begin{itemize}
	  \item {\bf Semantic}: {\em passar} $\rightarrow$ $\{$ 1: pass, 2: spend, \ldots $\}$
	  \begin{itemize}
	    \item passar$_1$ {\sc location} `He passat pel teu despatx.'
	    \item passar$_2$ {\sc time} `Vaig passar dues hores...'
	  \end{itemize}
%  \item {\bf Pragmatic}:
	\end{itemize}
\end{onlyenv}

\begin{onlyenv}<5>

Of course, depending on your level of analysis, some will be more possible than others.

\end{onlyenv}
\end{frame}

\begin{frame}[fragile]
  \frametitle{What do the rules look like ? }
% first column
\begin{multicols}{2}
\begin{small}
\begin{verbatim}
  <rule>
    <match lemma="pitää" tags="v.*">
      <select lemma="doallat" tags="v.*"/>
    </match>
  </rule>
  <rule>
    <match lemma="pitää" tags="v.*">
      <select lemma="liikot" tags="v.*"/>
    </match>
    <match tags="*.ela"/>
  </rule>
  <rule>
    <match tags="*.gen"/>
    <match lemma="pitää" tags="v.*">
      <select lemma="berret" tags="v.*"/>
    </match>
  </rule>
\end{verbatim}
\begin{verbatim}
  <rule>
    <or>
      <match lemma="vuosi"/>
      <match lemma="aika"/>
      <match lemma="kuukausi"/>
      <match lemma="tunti"/>
      <match lemma="viikko"/>
    </or>
    <match lemma="vaihde" tags="n.*">
      <select lemma="jorggáldat" tags="n.*"/>
    </match>
  </rule>
\end{verbatim}
\end{small}
\end{multicols}
\end{frame}


\begin{frame}
  \frametitle{How to get started writing rules}

Some ways you could get started writing lexical selection rules:

\begin{itemize}
  \item Think about a word and possible contexts when it can only be translated
    with another word
  \item Use a dictionary to find usage examples
  \item Use a monolingual corpus to look at a concordance (key word in context)
  \item Use a parallel corpus to find contexts where one word is used as a
    translation of another word
\end{itemize}

If you have an existing MT system, rules can also be learnt from parallel or monolingual
corpora,  but people interested in that can ask me later :)

\end{frame}


\end{document}
