\documentclass[10pt,xetex]{beamer} %[compress, blue]
\mode<presentation>

\usepackage{shupashkar}

\date{26th January 2011}
\title{{\cyrtext Раздел 6б: Противопоставления тюркских языков}}
\author{Jonathan Washington \\ \texttt{\href{mailto:jonwashi@indiana.edu}{jonwashi@indiana.edu}}}
\begin{document}

\begin{frame}
	\titlepage
	\MyLogoBottomCentred
\end{frame}

\logo{\includegraphics[height=1.6cm]{../logo/logo}}

%\begin{frame}
	%different ways of expressing e.g., present progressive
	%cognates with different meanings

%no gender рода нет
%Агглютинация
%
%
%\end{frame}
\begin{frame}
	%\hspace{0.1in}\newline
	%\hspace{-0.8in}
	\includegraphics[width=1.2\textwidth]{tklangs}
\end{frame}

\begin{frame}
	\frametitle{Местоимение 2-ого лица}
	%2nd person pronouns
	\begin{block}{ты \& вы \& ..?}
	\begin{tabular}{ll}
		\texttt{ky} & сен - силер - сиз - сиздер \\
		\texttt{kk} & сен - сендер - сіз - сіздер \\
		\texttt{tr} & sen - siz (- sizler) \\
		\texttt{tt} & син - сез \\
		\texttt{cv} & эсӗ - эсир/эсӗр \\
	\end{tabular}
	\end{block}
\end{frame}

\begin{frame}
	\frametitle{Конструкция ``нравиться''}
	%verb/construction ``like''
	\begin{block}{нравитья}
	\begin{tabular}{ll}
		\texttt{tt} & ({\color{green} мин}) {\color{blue} окув}ны {\color{red} ярат}а{\color{green} м} \\
		\texttt{ky} & ({\color{green} мен}) {\color{blue} окуу}ны {\color{red} жакшы көр}ө{\color{green} м} \\
		\texttt{ky} & {\color{blue} окуу} ({\color{green} мага}) {\color{red} жаг}а{\color{blue} т} \\
		\texttt{ky} & ({\color{green} мен}) {\color{blue} окуу}ну {\color{red} жактыр}{а\color{green} м} \\
		\texttt{kz} & ({\color{green} мен}) {\color{blue} оқу}ды {\color{red} жақсы көр}е{\color{green} мін} \\
		\texttt{kz} & {\color{blue} оқу} ({\color{green} маған}) {\color{red} ұна}й{\color{blue} ды} \\
		\texttt{tr} & ({\color{green} ben}) {\color{blue} okul}u {\color{red} sev}iyor{\color{green} um} \\
		\texttt{cv} & {\color{green} эпӗ} {\color{blue} вӗренме} {\color{red} юрат}ат{\color{green} ӑп} \\
		\texttt{ru} &  {\color{blue} учеба} ({\color{green} мне}) {\color{red} нрави}{\color{blue} т}ся
	\end{tabular}
	\end{block}
\end{frame}

\begin{frame}
	%issue of different gerunds used for same things in different languages (e.g., керек)
	%(tt) йокларга кирәк
	%(ky) укташ(ым) керек / уктоо керек
	%(kk) ұйықтау(ым) керек
	%(tr)
	\frametitle{Разные инфинитивы}
	\begin{block}{Разные инфинитивы / герундивы}
		\begin{tabular}{ll}
			\texttt{ru} & (мне) {\color{red} надо} ид{\color{blue} ти} на учебу \\
			\texttt{tt} & (миңа) укуга бар{\color{blue} ырга} {\color{red} кирәк} \\
			\texttt{kk} & (маған) оқуға бар{\color{blue} у}(ым) {\color{red} керек} \\
			\texttt{ky} & (мага) окууга бар{\color{blue} уу}(м) {\color{red} керек} \\
			\texttt{ky} & (мага) окууга бар{\color{blue} ыш}(ым) {\color{red} керек} \\
			\texttt{uz} & (минга) оқувга бор{\color{blue} иш}(им) {\color{red} керак} \\
			\texttt{tr} & (bana) okula git{\color{blue} me}(m) {\color{red} lazım} \\
			\texttt{cv} & ...?
		\end{tabular}
	\end{block}

\end{frame}

\begin{frame}
	%``gibi'' vs <simil>
	\frametitle{Аффикс (``подеж'') или послелог?}
	\begin{block}{``как''}
		\begin{tabular}{ll}
			\texttt{tr} & {\color{green} keçi} {\color{blue} gibi} şarkı söylüyor \\
			\texttt{ky} & {\color{green} эчки}{\color{blue} дей} ырдайт \\
			\texttt{kk} & {\color{green} ешкі}{\color{blue} дей} ән айтады \\
			\texttt{ru} & он(а) поет {\color{blue} как} {\color{green} козел} \\
		\end{tabular}
	\end{block}

	\begin{block}{с глаголами}
		\begin{tabular}{ll}
			\texttt{tr} & Hırvatistan başla{\color{green} dığ}ı {\color{blue} gibi} gider \\
			\texttt{ky} & Хорватия баштал{\color{green} ган}ын{\color{blue} дай} кетет \\
			\texttt{ru} & Хорватия продолжится {\color{blue} как} начи{\color{green} ла}сь \\
		\end{tabular}
	\end{block}
\end{frame}

\begin{frame}
	%abilitive
	\frametitle{Abilitative}
	\begin{block}{Abilitative}
		\begin{tabular}{ll}
			\texttt{ru} & я {\color{blue} уме}ю {\color{green} пис}ать \\
			\texttt{tr} & ben {\color{green} yaz}a{\color{blue} bil}iyorum \\
			\texttt{uz} & мен {\color{green} ёз}а {\color{blue} ол}аман \\
			\texttt{tt} & мин {\color{green} яз}а {\color{blue} ал}ам \\
			\texttt{ba} & мин {\color{green} яҙ}а {\color{blue} ал}ам \\ 
			\texttt{kk} & мен {\color{green} жаз}а {\color{blue} ал}амын \\
			\texttt{ky} & мен {\color{green} жаз}а {\color{blue} ал}ам \\
			\texttt{cv} & эпӗ {\color{green} ҫир}ма {\color{blue} пӗл}етӗп \\
		\end{tabular}
	\end{block}
\end{frame}


\begin{frame}
	%ki
	\frametitle{Относительные местоимения или слово указывующее на цитату}
	\begin{block}{Относительные местоимения или слово указывующее на цитату}
		\begin{tabular}{ll}
			\texttt{tr} & Annem {\color{green} de}rdi {\color{blue} ki} ``siz de büyüyeceksiniz adam olacaksınız.'' \\
			\texttt{ky} & Апам ``силер да чоңоёсуңар, адам болосуңар'' {\color{blue} деп} {\color{green} айт}мак. \\
			%(en) My mother would've said "you guys will grow up and will become men."
			\texttt{ru} & Моя мама бы {\color{green} сказа}ла {\color{blue} что}, вы тоже вырастете и станете мужчинами. \\
		\end{tabular}
	\end{block}
\end{frame}

\begin{frame}
	%tr: num ile num = ky: num.abl num.dat cheyin
	\frametitle{Цифры + падежи}
	\begin{block}{Цифры + падежи}
		\begin{tabular}{lll}
			\texttt{tr} & {\color{green} bir} ile {\color{blue} beş} & \texttt{bir<num> ile<postp> beş<num>} \\
			\texttt{ky} & {\color{green} бир}ден {\color{blue} беш}ке чейин & \texttt{бир<num><abl> беш<num><dat> чейин<postp>} \\
			\texttt{kk} & {\color{green} бір}ден {\color{blue} бес}ке шейин & \texttt{бір<num><abl> бес<num><dat> шейин<postp>} \\
			\texttt{ru} & от {\color{green} одн}ого до {\color{blue} пят}и & 
		\end{tabular}
	\end{block}
\end{frame}

\begin{frame}
	%instrumental
	\frametitle{Инструментальный подеж или послелог ``со''?}
	%verb/construction ``like''
	\begin{block}{Формы}
	\begin{tabular}{lp{13em}lp{13em}}
		\texttt{tur} & ile, &  & -(y)l\{A\} \\
		\texttt{uzb} & ila, bilan & \texttt{cjs} & -\{B\}((\{I\})л)\{A\} \\
		\texttt{tat} & белән & \texttt{tyv} & -биле \\
		\texttt{bak} & менән & \texttt{alt} & -л\{A\} \\
		\texttt{kir} & мен(ен)\newline (по умалчанию: менен) & \texttt{kaz} & -\{M\}ен(ен) \newline (по умалчанию: -\{M\}ен) \\	
		\texttt{nog} & \{M\}\{A\}n & \texttt{kjh} & -((\{B\})\{I\})\{N\}\{A\}ң \\
		& & \texttt{krc} & -б(\{I\})л\{A\} \\
		\texttt{chv} & ...? & \\
	\end{tabular}
	\end{block}

	\begin{block}{}
		\begin{itemize}
			\item Все формы связеные по значению и по этомологией
			\item Как можно анализировать каждый язык?
			\item А потом как можно перевести эти формы от одного языка на другой?
		\end{itemize}
	\end{block}
\end{frame}

\begin{frame}
	\frametitle{Разницы по семантике}
	\begin{block}{Слышать / слушать}
		\begin{tabular}{lll}
			%\toprule
			Русский & Казахский & Кыргызский \\
			%\midrule
			слышать & есіт- & ук- \\
			слушать & тыңда- & ук- \\
			%\bottomrule
		\end{tabular}
	\end{block}
\end{frame}

\begin{frame}
	\frametitle{Разницы по подежей}
	\begin{block}{к чему-то / в направления}
		\begin{tabular}{lllll}
			Татарский & мин & урман{\color{blue} га} & таба & киттем \\
			Казахский & мен & орман{\color{blue} ға} & қарай & кеттім \\
			Кыргызский & мен & токой{\color{blue} ду} & көздөй & кеттим \\
		\end{tabular}
	\end{block}
\end{frame}

\end{document}
