\documentclass[10pt,xetex]{beamer} %[compress, blue]
\mode<presentation>

\usepackage{helsingfors}

\usepackage{alltt}

\date{13--17th May 2013}
\title{Session 7: Data consistency and quality}
\begin{document}

\begin{frame}
        \titlepage
\MyLogoBottomCentred
\end{frame}

%\logo{\includegraphics[height=1.6cm]{../logo/logo.pdf}}


\section{Quality}


\begin{frame} %% framesection
 \begin{center}
 {\Large {\bf Quality}} % QUALITY
 \end{center}
\end{frame}
\begin{frame}
  \frametitle{\cyrtext{Quality}}%Quality

%В данном разделе мы поговорим о том, что такое качество и как его оценить
This session will discuss what \textbf{quality} means, and how it is measured \\

~\\

\begin{itemize}
  \item System quality
  \begin{itemize}

     \item Does the system pass automated tests ... e.g. of the dictionaries?
     \item Is the current version an improvement over the previous version?
     \item Does the system cover enough of the language to make it useful for the purpose for which it is intended?
     \item Can how the system works be easily understood ?
     \item How well is the system documented ? 
  \end{itemize}
~\\
~\\ 
  \item Translation quality
  \begin{itemize}

    \item How well does the system work for a particular purpose
    \begin{itemize}
      \item For translating a text such that it is possible to understand
      \item For translating a text so that it is quicker to post-edit than to translate from scratch
    \end{itemize}
  \end{itemize}
\end{itemize}

\end{frame}

\begin{frame}
  \frametitle{System quality}

\end{frame}

\begin{frame}
  \frametitle{System quality: Self-contained system}

Apertium is a self-contained system. For any input, it should have one, predictable, deterministic output. \\

For every lexical form in the source language dictionary, there should be:

\begin{itemize}
  \item A translation in the bilingual dictionary, {\em or} 
  \item A transfer rule which deletes the word
\end{itemize}

If there is a translation in the bilingual dictionary, there should be:

\begin{itemize}
  \item A form of the translated word in the TL morphological dictionary, {\em or}
  \item A transfer rule which changes the source language tags, and any delivered by 
    the bilingual dictionary into tags suitable for the target morphological dictionary
\end{itemize}

Let's look at an example...

\end{frame}

\begin{frame}[fragile]
  \frametitle{System quality: Consistency}

\begin{center}
``{\em Maşa'nın   köpeği yok ,  onun kedisi var.}''
\end{center}

\begin{onlyenv}<1>
Our reference translation is:
\begin{alltt}
{\smallermono Машӑн     йытӑ   ҫук ,  унӑн кушак  пур. }
\end{alltt}
But, our system currently translates it to Chuvash as:
\begin{alltt}
{\smallermono #Маша      *köpeği   ҫук ,  ун   #кушак пур. }
\end{alltt}

So, where do those {\tt *} and {\tt \#} come from ? We can get an idea by looking at 
the ``debug'' output (use {\tt apertium -d . tur-chv-dgen}):
\begin{alltt}
{\smallermono #Маша<np><ant><f><gen> *köpeği ҫук, ун #кушак<n><px3sp><nom> пур. }
\end{alltt}

\end{onlyenv}

\begin{onlyenv}<2>

\begin{alltt}
{\smallermono \alert{\#}Маша<np><ant><f><gen> \alert{*}köpeği ҫук, ун \alert{\#}кушак<n><px3sp><nom> пур. }
\end{alltt}

What do those symbols mean ? 

\begin{itemize}
  \item {\tt *} means either that:
     \item the word is not found in the source language morphological dictionary, or

     \item the word is not found in the bilingual dictionary,

  \item {\tt \#} means either that:
  \begin{itemize}
    \item the word is not in the target language monolingual dictionary, or

    \item the word followed by that particular sequence of tags is not in the target 
      language morphological dictionary, or

    \item a conflict in the phonological rules causes the surface form not to be in the compiled target language 
      dictionary
  \end{itemize}
\end{itemize}

So how can they be fixed ? 


\end{onlyenv}

\begin{onlyenv}<3>

First we add the translation(s) of {\em köpek} to the bilingual dictionary (in our case {\tt apertium-chv-tur.chv-tur.dix})

\begin{alltt}
<e><p><l>йытӑ<s n="n"/></l><r>köpek<s n="n"/></r></p></e>
\end{alltt}

Then we recompile and get,

\end{onlyenv}

\begin{onlyenv}<4>

First we add the translation(s) of {\em köpek} to the bilingual dictionary (in our case {\tt apertium-chv-tur.chv-tur.dix})

\begin{alltt}
<e><p><l>йытӑ<s n="n"/></l><r>köpek<s n="n"/></r></p></e>
\end{alltt}

Then we recompile and get,

\begin{alltt}
{\smallermono \alert{\#}Маша<np><ant><f><gen> йыти ҫук, ун \alert{\#}кушак<n><px3sp><nom> пур. }
\end{alltt}

The {\tt @} symbol has gone, but why do we get {\em йыти} not {\em йытӑ} ?

\end{onlyenv}

\begin{onlyenv}<5>

Next we add {\em Маша} and {\em кушак} to the target language morphological dictionary (in 
this example {\tt apertium-chv-tur.chv.lexc})

\begin{verbatim}

кушак N-INFL ; ! ""

...

Маша NP-ANT-F ; ! ""

\end{verbatim}

Then we recompile and get,

\begin{alltt}
{\smallermono Машӑн йыти ҫук, ун кушакӗ пур. }
\end{alltt}

\end{onlyenv}

\begin{onlyenv}<6>

Let's compare that to the reference translation,

\begin{itemize}
  \item Образец: ~~~~~~~Машӑн йытӑ ҫук, унӑн кушак пур.
  \item Наш вариант: Машӑн \alert{йыти} ҫук, \alert{ун} \alert{кушакӗ} пур.
\end{itemize}

We have removed the debugging symbols by:

\begin{itemize}

  \item Adding a word to the bilingual dictionary
  \item Adding two words to the target-language monolingual dictionary
\end{itemize}

The differences that remain between the machine translation of this sentence 
and the reference are either:

\begin{itemize}

  \item Transfer errors, or
  \item Translation divergence 
\end{itemize}

\end{onlyenv}

\end{frame}

\end{document}
