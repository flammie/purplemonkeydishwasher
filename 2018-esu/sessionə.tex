\documentclass[10pt,xetex]{beamer} %[compress, blue]
\mode<presentation>

\usepackage{helsingfors}
\usepackage{alltt}
\usepackage[normalem]{ulem}

\date{2nd--13th November, 2015}
\title{$\alpha$: Re-usable things, infra-structures}

\begin{document}

\begin{frame}
        \titlepage
\MyLogoBottomCentred
\end{frame}

\logo{\includegraphics[height=1.6cm]{../logo/logo}}

\begin{frame}
  \frametitle{Introduction}

\end{frame}

%%%%%%%%%%%%%%%%%%%%%%%%%%%%%%%%%%%%%%%%%%%%%%%%%%%%%%%%%%%%%
%% Case studies
%%%%%%%%%%%%%%%%%%%%%%%%%%%%%%%%%%%%%%%%%%%%%%%%%%%%%%%%%%%%%

\begin{frame}
  \frametitle{Existing morphologies}
  Usually largely available for big well-resourced languages
    \begin{itemize}
        \item Finnish: omorfi \url{https://github.com/flammie/omorfi/}
        \item Estonian: plamk \url{https://github.com/jjpp/plamk/}
        \item \ldots
    \end{itemize}

\end{frame}

\begin{frame}
  \frametitle{Ad hoc tagsets}
  \texttt{
mikä|UPOS=PRON|PRONTYPE=INT|CASE=INE  \\
mikä|UPOS=PRON|PRONTYPE=REL|CASE=INE  \\
mikä|UPOS=PRON|SUBCAT=QUANTIFIER|CASE=INE  \\
missä|UPOS=ADV \\
olla|UPOS=AUX|VOICE=ACT|MOOD=INDV|TENSE=PRESENT|PERS=SG3  \\
olla|UPOS=VERB|VOICE=ACT|MOOD=INDV|TENSE=PRESENT|PERS=SG3  \\
Jaak|UPOS=PROPN|PROPER=PROPER|NUM=SG|CASE=NOM|CLIT=KO  \\
Jaakko|UPOS=PROPN|PROPER=PROPER|NUM=SG|CASE=NOM }
vs. our hand-written:
\texttt{\textasciicircum missä/missä<adv>\$
\textasciicircum Jaakko/*Jaakko\$
\textasciicircum on/olla<v><3sg><pres>\$}
\end{frame}

\begin{frame}
  \frametitle{Ad hoc tagsets}
  \texttt{
kus+Adv \\
olema+V+indic+pres+ps3+pl+ps+af \\
olema+V+indic+pres+ps3+sg+ps+af \\
James+H+sg+nom
}
vs. 
\texttt{\textasciicircum kus/kus<adv>\$
\textasciicircum on/olema<v><3sg><pres>/olema<v><3pl><pres>\$
\textasciicircum James/*James\$}

\end{frame}

\begin{frame}
  \frametitle{Best common practices to match them: Common research infra}

  Common research infra is
\begin{itemize}
    \item SVN directory structure for lexc, twol, etc. files
    \item coding style and standard tags
    \item used for tons of applications, not just MT
    \item super complicated
    \item hopefully backed up by good tech support team
\end{itemize}

\end{frame}

\begin{frame}
    \frametitle{Giellatekno.uit.no}
    \begin{itemize}
        \item developed in Tromsø for Saami, urals etc. languages with
            finite state morphology in mind
        \item current version is good for spell-checking, MT, analysis,
            computer aided language learning and some text to speech
        \item free / open source, usable by anyone, goood-ish documentation
            in a few languages, smart people answering questions, etc.
        \item we've made a few academic publications that are good source for
            more info (I'm gonna re-use ones from Free Software conference 2013
            Gothenburg now)
    \end{itemize}
\end{frame}



\end{document}
