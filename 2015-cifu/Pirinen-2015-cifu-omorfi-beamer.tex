\documentclass{beamer}


\usepackage{amssymb}
\usepackage{amsmath}
\usepackage{amsfonts}

\usepackage{fontspec}
\usepackage{polyglossia}

\usepackage{graphicx}
\usepackage{color}
\usepackage{url}
\usepackage{textpos}
\usepackage{xspace}
\usepackage{array}

\mode<presentation>
{
  \usetheme{Abumatranpres}
}


\graphicspath{{./fig/}}


\title{Omorfi experiences with crowds and sourcings\\
\scriptsize{CIFU XII Oulu, 2015}}
\author{Tommi A Pirinen \scriptsize \guilsinglleft{}tommi.pirinen@computing.dcu.ie\guilsinglright{}}
\institute{Ollscoil Chathair Bhaile Átha Cliath, ADAPT Centre\\
EU Marie Curie Abu-MaTran project}
\date{\today}

\begin{document}

\selectlanguage{english}

\maketitle

\begin{frame}
    \frametitle{Contents}
    \begin{itemize}
        \item omorfi backgrounds
        \item crowd sources
        \item current datasets
        \item issues
        \item (no MT here this time...)
    \end{itemize}
\end{frame}

\begin{frame}
    \frametitle{Omorfi}
    \begin{itemize}
        \item omorfi (open morphology of Finnish) has grown to be a mature
            computational language description
        \item majority of basic lexical information comes from high quality
            dictionaries (Nykysuomen sanalista)
        \item however, the need for lexical data is an open-ended task; you
            can never have too much information on any given word-form
        \item also, new lexemes come and go all the time, cannot wait for
            kotus to release a new word list to get us selfies
    \end{itemize}
\end{frame}

\begin{frame}{Uses of Crowd-Sourcing in omorfi}
    \begin{itemize}
        \item get new words asap
        \item semantics: is it (can it be) human, sentient, edible, female,
            location, corporation, mass nouns
        \item popularity: common word, rare, obscure
        \item style and usage: dialects, curse words, academic, computer,
            medicine
        \item \ldots approx. every new application built of omorfi needs
            more data than was in the lexical database before
    \end{itemize}
\end{frame}

\begin{frame}{Issues in Crowd-Sourcing Lexicographies}
    \begin{enumerate}
        \item Using data (long) after it has been built by harvesting, scraping,
            etc. requires lots of work
        \item Inputting well-structured data in system not designed for it is
            cumber-some and error prone
        \item That is, wiktionary is really just an attempt of using something
            designed for writing encyclopedic prose in structured dictionaries
        \item Wiktionaries are never stable, trying to use data from outside
            the system requires tracking changes in conventions
        \item Newer systems attempted to bridge the gaps have not been 
            successful either (Omegawiki, ...)
    \end{enumerate}
\end{frame}


\begin{frame}[fragile]{And its Source...}
    \begin{verbatim}
===Verbi===
{{fi-verbi|as|ettaa|muistaa|C}}

# [[laittaa]], [[panna]], [[sijoittaa]] paikalleen
#:''Hän asetti maljakon pöydälle.''
# [[määrätä]], [[määrittää]]
...
====Käännökset====
{{kohta|1|laittaa, panna, sijoittaa paikalleen|
*englanti: [[put]], [[place]], [[set]], move into position, [[locate]]
*hollanti: [[aanbrengen]]
\end{verbatim}
\end{frame}

\begin{frame}{Scraping the Data From Wiktionary}
    \begin{enumerate}
        \item find section for Finnish words
        \item find each definition
        \item find and translate something like
            \texttt{fi-verbi|as|ettaa|muistaa|C}
                into \texttt{asettaa V\_MUISTAA VERB 53 C\ldots}
    \end{enumerate}
    \begin{itemize}
        \item e.g., when I last wrote the script for scraping this data,
            \texttt{fi-verbi|as|ettaa|muistaa|C} was \texttt{fi-verb|53|C}
    \end{itemize}
\end{frame}

\begin{frame}{Example 2: Omegawiki}
    \begin{itemize}
        \item database approach for storing data in well structured form
        \item getting data would be easier and more consistent
        \item still quite cumbersome to edit
        \item lacks some central pieces of information for Finnish and most
            other langs than English, e.g., inflection classification
    \end{itemize}
\end{frame}


\begin{frame}{Quality Issues in Crowd-Sourcing}
    \begin{itemize}
        \item people know lots of their native languages but not always enough
        \item some contributors are language learners
        \item vandalism
        \item Two ways currently used to cope with this: python scripts, regexes
            etc. to check some sanity
        \item Automatic tests with the final software and free texts: do new
            additions work somewhat like old words, etc.
        \item In the end it all falls down to expert reviews again
    \end{itemize}
\end{frame}

\begin{frame}{Conclusions (questions): How to Proceed?}
    \begin{itemize}
        \item How to combine popularity of Wiktionary with forms and structure
            of Omegawiki?
        \item Improve user interfaces?
        \item Better access to wiki data?
        \item Feedback from databases to Wiktionary?
        \item Answers? Questions? 
    \end{itemize}
\end{frame}

\end{document}
% vim: set spell:
