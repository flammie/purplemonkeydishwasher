\documentclass[t,12pt]{beamer}

\usepackage{fontspec}

\usepackage{xunicode}
\usepackage{xltxtra}

\usepackage{polyglossia}

\setmainfont{Liberation Serif}
\newfontfamily\devanagarifont{Lohit Nepali}

\setmainlanguage{english}
\setotherlanguages{sanskrit}


\usepackage{graphicx}
\usepackage{color}
\usepackage{url}
\usepackage{textpos}
\usepackage{xspace}
\usepackage{array}
\usepackage{ulem}

\graphicspath{{./fig/}}

% theme options: hy/ml/hum, rovio/sinetti, hiit
% default: hy,rovio

%\usetheme[hy]{HY}
%\usetheme[hy,sinetti]{HY}
\usetheme[hum,rovio]{HY}
%\usetheme[ml,rovio]{HY}
%\usetheme[ml,rovio,hiit]{HY}


\title{State-of-the-art in Weighted Finite-State Spell-Checking\\
\scriptsize{CICLING 2014, Kathmandu}}

\author{Tommi A Pirinen \scriptsize \guilsinglleft{}tommi.pirinen@helsinki.fi\guilsinglright{}}
\institute{University of Helsinki\\Department of Modern Languages}
\date{\today}

\begin{document}

\selectlanguage{english}

\HyTitle
%\maketitle

\begin{frame}
    \frametitle{Outline}
    \tableofcontents
\end{frame}


\AtBeginSection[]
{
  \begin{frame}<beamer>
    \frametitle{Outline}
    \tableofcontents[currentsection]
  \end{frame}
}

\section{Introduction}
       
\begin{frame}
    \frametitle{Spell-Checking}
    \begin{itemize}
        \item Spell-Checking is everywhere: browsers, office software, mobile
            phones, \ldots
        \item Three tasks / steps: for each word \begin{enumerate}
                \item detect if word is correctly written / how likely the
                    written word is what was meant (language modelling
                    for error detection)
                \item if not, what modifications can be made to find more
                    likely correct word (error modelling)
                \item rank corrections in order of likelihood
                    (language modelling for error correction)
            \end{enumerate}
    \end{itemize}
\end{frame}

\begin{frame}
    \frametitle{Spell-Checking and State-of-the-Art}
    \begin{itemize}
        \item Spell-Checking is everywhere: browsers, office software, mobile
            phones, \ldots
        \item State of the art in software we know and use: hunspell (limited
            edit distance, no statistics)
        \item State of the art in science: correct, bayspell, etc. (statistics,
            linguistic analysis)
        \item State of the art we don't know: mobile phones, closed software
            (word-form lists?)
        \item 
            \only<1>{grammar checkers: LanguageTool, ... (context-based \(n\) grame
            models, programmatic rules)}
            \only<2>{\sout{grammar checkers: LanguageTool, ... (context-based \(n\) grame
            models, programmatic rules)}}
    \end{itemize}
\end{frame}

\begin{frame}
    \frametitle{Weighted Finite-State Spell-Checking Graphically}
    From this misspelled word as automaton:\\
    \includegraphics[height=0.6in]{cta}\\
    Applying this very specific error (correction) model:\\
\includegraphics[height=0.6in]{cta2cat}\\
    We can compose or intersect with a word in this dictionary:\\
\includegraphics[height=0.6in]{catses}\\
\end{frame}


\section{Finite-State Models}

\begin{frame}
    \frametitle{Language Models}
    \begin{itemize}
        \item language models are 1 tape finite-state machines where weight may
            map e.g. probabilities of the word-forms
        \item word-list and frequencies can be compiled into of path automatons,
            trie, etc.
        \item morpheme combinations and cycles done with FST description
            languages like Xerox finite-state morphology, SFST-PL, apertium
        \item weights can be trained using raw text corpora
    \end{itemize}
\end{frame}

\begin{frame}
    \frametitle{A language model as a FST}
    Resulting language model should look like this:
    \includegraphics[height=2.0in]{demo-lm}
\end{frame}

\begin{frame}
   \frametitle{Error models}
   \begin{itemize}
       \item Error models are two-tape automata that map misspelled word-forms
           into a correctly spelled one
       \item Levenshtein--Damerau edit distance errors are essentially single
           arcs in automaton disjuncted between correctly written cycles
       \item Confusion sets are string-pair disjunctions
       \item Full error models are made by combining these with FSA algebra
   \end{itemize}
\end{frame}

\begin{frame}
    \frametitle{Edit distance automaton graphically}
    \includegraphics[width=\textwidth]{errm-ed1}
\end{frame}

\begin{frame}
    \frametitle{Edit distance automaton graphically 2}
    \includegraphics[width=\textwidth]{errm-ed2}
\end{frame}

\begin{frame}
    \frametitle{Edit distance automaton graphically 7}
    \includegraphics[width=\textwidth]{errm-ed7}
\end{frame}

\section{Comparison}

\begin{frame}
    \frametitle{Why is this state-of-the-art}
    {\tiny Besides the fact that it's the only available open source WFSA speller
        for variety of platforms\hyperlink{links}{\beamergotobutton{Links}}}\\
        \begin{table}
            \begin{tabular}{l|rrr}
                \textbf{Language} & foo & bar & baz \\
                \hline
                English & & & \\
                Finnish & & & \\
                North Saami & & & \\
                Greenlandic & & & \\
            \end{tabular}
        \end{table}
\end{frame}

\begin{frame}[label=links]
    \frametitle{Links and references}
    \begin{itemize}
        \item LibreOffice extensions for Mac OS X and Windows:
            \url{http://divvun.no/libreofficeoxt.html}
        \item LO, enchant, etc. sources, for Linux:
            \url{https://sourceforge.net/projects/hfst/files/spell-transducers/}
        \item Previous research \url{http://www.helsinki.fi/\%7etapirine/publications/}
    \end{itemize}
\end{frame}

\begin{frame}
    \frametitle{Obligatory thanks \& questions slide}
    Thank you\\ kiitos\\\devanagarifont{ धेरै धेरै धन्यबाद। }
\end{frame}

\end{document}
% vim: set spell:
