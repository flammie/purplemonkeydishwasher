\documentclass[officiallayout]{unihelcompling}

\usepackage{fontspec}
\usepackage{xunicode}
\usepackage{xltxtra}

\usepackage{ifpdf}
\usepackage{url}
\usepackage{hyperref}

\setmainfont{Times}

\ifpdf
\pdfinfo{
    /Author (Tommi Pirinen)
    /Title (Finite-State Methods to Spell-Checking)
    /CreationDate (D:20121212123456)
    /Subject (Finite-State Spell-Checking)
    /Keywords (Finite-State;Spell-Checking)
}
\fi

\title{Finite-State Methods to Spell-Checking\footnote{Rough draft}}
\author{Tommi A Pirinen}

\authorcontact{\url{tommi.pirinen@helsinki.fi}\par
  \url{http://www.helsinki.fi/\%7etapirine}}
\pubtime{Oddmonth}{2012}
\reportno{0}
\isbnpaperback{000-00-0000-0}
\isbnpdf{000-00-0000-0}
\issn{0000-0000}
\printhouse{Unigrafia?}
\pubpages{000} % --- remember to update this!
\supervisorlist{Kimmo Koskenniemi, University of Helsinki, Finland;
Krister Lindén, University of Helsinki, Finland}
\preexaminera{Kekke Roos, Unseen University, United Kingdom}
\preexaminerb{I.D.K. Kurt, University of Turku, Finland}
\opponent{Advocatus Diaboli, University of Amsterdam, The Netherlands}
\custos{Homer Simpson, University of Springfield, The U.S.A.}
\generalterms{thesis, example, another example, still more examples,
  more and more examples}
\additionalkeywords{example, an example phrase with many words}
\crcshort{A.0, C.0.0}
\crclong{
\item[A.0] Example Category
\item[C.0.0] Another Example
}
\permissionnotice{
  To be presented in \ldots{} text of a long permission notice. Text of
  a long permission notice. Text of a long permission notice. Text of
  a long permission notice. Text of a long permission notice. Text of
  a long permission notice.
}



\date{\today}

\begin{document}

\frontmatter

\maketitle

\begin{abstract}
    The finite-state technologies are one thing.

    Spell-checking is another thing.
\end{abstract}

\tableofcontents

\mainmatter

\chapter*{Preface}

\begin{itemize}
    \item Practical requirement for Finite-State Spell-Checkers
    \item Non-thanks
\end{itemize}

\chapter{Introduction}

\begin{itemize}
    \item what's spell-checking about
    \item Languages and stuff
    \item Spell-checking
    \item Finite-state technology
    \item Layout of the book: compiling models for finite-state spell-checker,
        features of contemporary spell-checking, efficiency and precision of
        these models
\end{itemize}

\section{Spell-Checking}

\begin{itemize}
    \item split of task into error-detection and correction
    \item history, from SPELL to hunspell
\end{itemize}

\subsection{Error-detection and language models}

\begin{itemize}
    \item Word-lists to complex morphologies
    \item The nonword errors vs. real-world errors
    \item The absolute dictionaries vs. probabilities of words
    \item the limits of real-world error detection (scope out cop out)
\end{itemize}


\subsection{Error-correction and error models}

\begin{itemize}
    \item The concept of error modeling
    \item The model of typing errors on keyboard
    \item edit distance
    \item other confusion sets
    \item competence errors, phonemic errors
\end{itemize}

\subsection{Finite-state account of spell-checking}

\begin{itemize}
    \item short history of finite-state morphologies
    \item weighted finite-state methods
    \item training probabilistic dictionaries (see, this is where 1st articles
        come in!)
    \item the finite-state formulation of error models
    \item the finite-state applications of language and error models
\end{itemize}

\chapter{Finite-state Automata in Spell-checking}

\begin{itemize}
    \item the statistical language model inducted from correctly spelled corpora
    \item the compilation of current dictionaries and their finite-state fomrs
    \item error-model from existing dictionaries
    \item error-model from hunspell
    \item error-modls from error corpora
    \item finite-state models for context-aware spell-checking
\end{itemize}

\chapter{Efficiency of Finite-state Spell-checking}

\begin{itemize}
    \item the speed of finite-state spell-chekcing
    \item the precision values of finite-state spell-checking
\end{itemize}

Here we cite* to generate references in the end. \cite{*}

\bibliographystyle{unsrt}
\bibliography{diss}

\end{document}
