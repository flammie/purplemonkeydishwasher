\documentclass[11pt]{article}
%\documentclass{flammie}
\usepackage{nodalida2015}
%\usepackage{times}
\usepackage{mathptmx}
%\usepackage{txfonts}
\usepackage{url}
\usepackage{latexsym}

\usepackage{fontspec}

\special{papersize=210mm,297mm} % to avoid having to use "-t a4" with dvips
%\setlength\titlebox{6.cm} % You can expand the title box if you really have to

\title{Using weighted finite state morphology with VISL CG-3---Some experiments
    with free open source Finnish resources}

\newif\ifpublished
\publishedtrue


\ifpublished
\author{Tommi A Pirinen\\
Ollscoil Chathair Bhaile Átha Cliath\\
CNGL---School of Computing\\
Dublin City University, Dublin 9\\
{\tt tommi.pirinen@computing.dcu.ie}
}
\fi

\date{\today}

\begin{document}

\maketitle
\begin{abstract}
    Traditionally, the coupling of finite state morphology and constraint
    grammar has been strictly  rule-based, making binary distinctions
    between allowed and disallowed readings, however, in the recent years much
    of the research in the finite state morphologies has adapted the
    contemporary paradigm of statistically weighted analysis. This is reflected
    in current versions of free and open source morphology of Finnish, omorfi, in
    the finite state morphology part. In this paper we examine two strategies
    of making use of the weights as a part of VISL CG-3 pipeline.
    We evaluate the results intrinsically on small sample of analyses we
    have disambiguated by hand ourselves, and extrinsically on the effect it
    has on the rule-based machine translation of that text using the freely
    available open source translator, apertium-fin-eng. 
\end{abstract}

\section{Introduction}

In the recent years, use of statistical information in computational
linguistics has gained much interest, with systems like hunpos~\cite{hunpos},
moses~\cite{moses} etc.
being the main points of interest of most research in the field. In finite
state morphology as well as constraint grammars, extensions to handle
probabilities are recent and scarcely
documented~\cite{pirinen2009weighting,bick2009introducing}.  In this paper we
experiment with an existing weighted finite state morphology of
Finnish~\cite{pirinen2011modularisation}\footnote{\url{https://github.com/flammie/omorfi/}}
with VISL CG-3. For CG we have adapted Fred Karlsson's Finnish CG rules to
omorfi's tag set, however, the rules were written for completely different
analyser, which results in relatively low quality and high level of ambiguity
at the current level. We estimate that salvaging these rules for the current
version of analysis would require a substantial re-writing effort. In the
meanwhile, there are a lot of easy targets that correctly trained statistical
analyser can already deal with without extra effort. E.g., one large 
difference we assume between the analyser these CG rules were written for and
omorfi's are that omorfi contains a huge number of proper nouns, dialectal
and sub-standard forms, and rare language, animal etc. names, that are left
ambiguous. It is obvious for a human reader that these words are very
unlikely and given most corpora we expect them to be highly penalised as
well.

The main goal of this experiment is to create a functional
pipeline out of weighted finite-state analysis and current version of the
constraint grammar. There are obvious conflicts between the statistically
driven ranked hypotheses approach and the strictly deleting approach of the
current constraint grammar, which may limit usefulness of our current method
of combining these two information sources. 

The rest of the paper is structured as follows: In section~\ref{sec:background}
we explain our starting point and current pipelines for morphological analysis,
disambiguation and machine translation.  In section~\ref{sec:methods} we
explain various approaches we tried to include and combine weight data from
the weighted finite-state analysers into VISL CG-3 and finally into machine
translation. In section~\ref{sec:experimental-setup} we describe our
experiments and how we measured the workability of our approach. In
section~\ref{sec:evaluation} we show the results of the experiment. In
section~\ref{sec:discussion} we perform error analysis, compare our work with
other existing approaches and lay out the future work. Finally in 
section~\ref{sec:conclusion} we summarise the conclusion of the experiments.

\section{Background}
\label{sec:background}

Our starting point for this experiment is such that we had a modern, weighted
finite-state morphology~\cite{beesley2003finite,pirinen2008weighting} implementation of Finnish morphology in
omorfi~\cite{pirinen2015omorfi}. This morphology has rudimentary support for
probabilistic weighting of surface forms or analyses using corpora-based 
unigram training approach. However, with the lack of high quality free
and open source corpora compatible with omorfi analyses means that it is
distributed with very basic linguist-written weights on the analysis side.
For the main purpose of this experimentation we deemed this sufficient, to
get the weights working through the pipeline at all.

On the other hand we had a free and open source, mature and large CG grammar by
Fred Karlsson, that needed conversion to omorfi compatible tagging format, as
well as some changes from CG-1 syntax to VISL CG-3.\footnote{even though CG-1
    and VISL CG-3 are possibly are mostly compatible, we found that some things
may have started working better when changing to more conventional VISL CG-3
constructions} 

The fact that the CG rules from Karlsson were built using very different
analyser than ours also played a large role in our decision to combine the
weighted approach to with pure constraint grammar approach: the rule-writers
of the original grammar had not seen large portion of the ambiguities
introduced by larger, more varied lexicon of omorfi, including things like
dialects, large inventories of proper nouns and unlikely but attested
readings like plural cases of singular personal pronouns. For example, in
the story we use for reference in our translation experiments, the sentence
initial common words like ``Mutta'' (but) and ``Koira'' (dog)  are also
proper nouns, but also proper nouns like ``Mari'' have been added a
common noun reading (slang for marihuana). Obviously these are not dealt with
in the original ruleset as they have not appeared as ambiguities to the
writers of th rules.

\section{Methods}
\label{sec:methods}

To first convert the original CG-1 ruleset to omorfi format analyses, we went
through the rules by hand from beginning to end. This resulted in a ruleset
where only a subset of rules matched to any constructs in the analysed texts.
To further improve the quality and fix a lot of conversion errors we made use
of the new VISL CG-3 features \texttt{no-inline-sets}. With help of this
feature we got most of the ambiguous word-forms at least to match some of the
rules, which hopefully means conversion has not too many tag formatting 
mismatch errors at the very least. The resulting ruleset with weight-based
rule integrated is available from omorfi git repository.

To feed omorfi analyses into VISL CG-3 we have extended the python interface
of omorfi to output CG stream format analyses, with omor style
\texttt{[FEATURE=VALUE]} tags mapped into more conventional CG style tags,
mostly of form \texttt{VALUE}. There are number of deviations to this of
course, most notable being the \texttt{WEIGHT=} feature, which is turned into
VISL CG-3 \emph{numeric tag}. Other special conversions include things like
usage, dialect and such lexical information, which are all included in angle
bracket tags following VISL CG-3 conventions. Omorfi python interface also
performs some case mangling (uppercasing, lowercasing, title-casing and
removing title case) that seems to be similar as CG-1 rules seem to expect to
appear in some angle-bracketed tags, so we have tried to map these to the
readings in the original ruleset, with limited success. 

The probabilities in omorfi are provided by the underlying HFST~\cite{hfst}
system as a floating point number based on the finite-state implementation of a
tropical semiring.  This weight can be based on negative logarithms of
probabilities of the word-forms, lemmas, analyses etc., as well as
linguist-defined arbitrary values. For the purposes of this experiment we only
used the linguist-defined values that are neatly in range of 0.1---32. This
simplifies the scaling of the weights introduced by VISL CG-3 processing as we
only have to scale against known range instead of e.g.  combinations of
negative logarithms' maxima. As noted earlier in section~\ref{sec:background},
we use the default setting which is based on linguist-approximated tag
likelihoods. Since VISL CG-3 does
not support floating point numbers, e.g. 0.1, we output weight in a
numeric tag multiplied by a 100 before rounding them down and turning 
into a tag of the form \texttt{<W=\emph{weight}>}, where
\emph{weight} is the multiplied weight. This is sufficient for the
coarse weights that default analyser produces, and in line what e.g. 
\texttt{cg-conv} does when it treats stream formats containing decimal data
to be converted into numeric tags.

The basic support for numeric tag processign in VISL CG-3 is done by the
\texttt{SELECT (<W=MIN>)} statement. If applied as a sole rule to result of
omorfi to VISL CG-3 conversion it exactly like traditional weighted finite
state morphology producing 1-best analysis. When combined into existing
ruleset, we add this into a last, separate \texttt{SECTION}, in order to
integrate some weight handling to CG iterations.

One long-term goal of this experimentation was to use VISL CG-3 also as a
part of morphological analysis pipeline that produces n-best lists in same
manner as weighted finite-state analyser does. To make this work, we take
the output of VISL CG-3's cg-proc in trace mode before converting it back
to an n-best list. There are multiple possible strategies to use readigns
for deleted analyses as weights again. With this experiment, we have simply
gone with adding the line number of the rule, this reflects the fact that
later rules in the file are more risky and less ambiguous. Ideally however,
we would like to annotate the rules using rule name labels, such as
``usually'', ``dangerous'' to denote e.g. multipliers for such rules. Furthermore,
it is likely that it is not exactly the line number, but rather the section
number, that is relevant for the rule likelihood, due to way linguists and
rulewriters will organise rules within sections into blocks of related rules
where ordering within and between blocks may not be important.

\section{Experimental Setup}
\label{sec:experimental-setup}

For analysis we use the python API to omorfi version 20150326, to turn the
analyses into the format understood by VISL CG 3. We use a version Fred
Karlsson's Finnish CG found in apertium's
repository.\footnote{\url{http://sourceforge.net/p/apertium/svn/HEAD/tree/nursery/apertium-fin-eng/apertium-fin-eng.fin-eng.rlx}},
with the tag set manually converted to match
omorfi's,\footnote{\url{https://github.com/flammie/omorfi/tree/master/src/vislcg3}}
however, given the amount of ambiguous names of tags and sets and lists in the
grammar, there may be some conversion errors left.  The system is tested with
VISL CG-3 version 0.9.9.10730, compiled from Gentoo
packaging.\footnote{\url{https://github.com/flammie/flammie-overlay/tree/master/sci-misc/vislcg3}}

To test the functionality of our combination of weighted
finite-state analyser and VISL CG-3, we analyse a short text that we have
manually disambiguated and measure the quality of analyses. The source of the
text is found in the apertium's SVN
repository.\footnote{\url{http://sourceforge.net/p/apertium/svn/HEAD/tree/nursery/apertium-fin-eng/texts/tarina.fin.text}}
For the purpose of this experiment, we have manually tokenised the text
before processing it with omorfi.
In addition to analysis we use the results of analyses in apertium's
Finnish-English machine translator, and measure the translation quality. This
way we can ensure that the gold annotation has not been selected to best fit
our results but is actually the semantically most fitting one. The gold
annotations can also be found in the omorfi git repository.

To perform evaluations we used simple python script that performs string
comparisons of the gold file lines between the lines starting with
\texttt{"<} ignoring empty lines and the ADDed CLB tags. The machine
translation analysis was performed against current apertium-fin-eng ruleset
and the reference translation in their svn, with standard machine translation
metrics as measured by NIST's \texttt{mteval-13a.pl}, which is the standard
BLEU metric of machine translation~\cite{bleu}.

\section{Evaluation}
\label{sec:evaluation}

We first evaluated the analysers against the gold standard in
table~\ref{table:intrinsic}. We use simple metrics of \textit{Recall} and
\textit{Precision}, defined as $\mathrm{Recall} =
\frac{\mathrm{Correct}}{\mathrm{Gold}}$, where Correct is number of correct
readings and Gold is number of gold readings, and $\mathrm{Precision} =
\frac{\mathrm{Correct}}{\mathrm{All}}$, where All is number of all readings
given by the disambiguation scheme. The row \textit{Weights} stands for CG with
only select weighted best applied, the row \textit{Rules} stands for only
converted CG ruleset applied, and row \textit{Combination} uses both.

\begin{table}
    \centering
    \begin{tabular}{|l|r|r|}
        \hline
        \bf Rules & \em Precision & \em Recall \\
        \hline
        \em Weights & 60 & 99 \\
        \hline
        \em Rules & 78  & 91 \\
        \hline
        \em Combination & 80 & 90 \\
        \hline
    \end{tabular}
    \caption{Precision and recall of different combinations of
    weighted morphology and rules.\label{table:intrinsic}}
\end{table}

The precision with combination is as expected greater than converted rules,
which in turn is greater than only weight-based rules that currently have not
much large disambiguating effect. The recall conversely is largest as the
weights let most readings through.


The resulting analyses is then converted to format expected by apertium for
machine translation and evaluated for machine translation quality in
table~\ref{table:extrinsic}.

\begin{table}
    \centering
    \begin{tabular}{|l|r|r|}
        \hline
        \bf Rules & \em BLEU & \em PWER \\
        \hline
        \em Weights & 6.86 & 87.11 \\
        \hline
        \em Rules & 5.76  & 88.67 \\
        \hline
        \em Combination & 5.60 & 88.89 \\
        \hline
    \end{tabular}
    \caption{Precision and recall of different combinations of
    weighted morphology and rules.\label{table:extrinsic}}
\end{table}

As can be seen from the scores the translator is still quite far from usable
quality and thus comparison may not be very interesting. However we can see
that the scores are systematically better with system's deemed worse by
precision and better by recall.


\section{Discussion}
\label{sec:discussion}

First of all, we note that the quality differences with adding weights has
diminished from the version prior to conference and current version. This is
largely due to newer version released in the workshop containing features that
greatly improved the tag matching of the converted ruleset. Following this
result we can say that the weights are most useful when the rules are not
as high coverage, i.e. early stages of development or, as in this case,
conversion process.

Nevertheless, the overall effect of combining weights has still improvements
to exactly the shortcomings noted in the introduction as problems of the
mismatching morphologies. In error evaluation, the cases that are affected
by rules are mostly in derivation and productive compounding, but also some
marginal cases that are not covered by rules.

For future work we are aiming to use the n-best list version of the result in
a real-world application pipeline.

\section{Conclusion}
\label{sec:conclusion}

We have implemented a VISL CG-3 output on top of existing weighted finite-state
analysis of Finnish language and tested that it works combined with VISL CG-3.
We have successfully included this combination as a part of apertium machine
translation pipeline. We note that weighted finite-state analysis can be
easily combined with VISL CG 3 and results in an increased accuracy.

\section*{Acknowledgments}

The research leading to these results has received
funding from the European Union Seventh Framework
Programme FP7/2007-2013 under grant agreement
PIAP-GA-2012-324414 (Abu-MaTran).


\bibliographystyle{acl}
\bibliography{nodalida2015}

\end{document}
% vim: set spell:
