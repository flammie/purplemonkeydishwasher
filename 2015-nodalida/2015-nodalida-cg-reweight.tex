\documentclass[11pt]{article}
%\documentclass{flammie}
\usepackage{nodalida2015}
%\usepackage{times}
\usepackage{mathptmx}
%\usepackage{txfonts}
\usepackage{url}
\usepackage{latexsym}
\special{papersize=210mm,297mm} % to avoid having to use "-t a4" with dvips
%\setlength\titlebox{6.cm} % You can expand the title box if you really have to

\title{Using weighted finite state morphology with VISL CG-3? Some experiments
    with free open source Finnish resources}

\newif\ifpublished
\publishedfalse


\ifpublished
\author{Tommi A Pirinen\\
Ollscoil Chathair Bhaile Átha Cliath\\
CNGL---School of Computing\\
Dublin City University, Dublin 9\\
{\tt tommi.pirinen@computing.dcu.ie}
}
\fi

\date{\today}

\begin{document}

\maketitle
\begin{abstract}
    Traditionally, the coupling of finite state morphology and constraint
    grammar has been strictly in the realm of rule-based, binary distinction
    between allowed and disallowed readings, however, in the recent years much
    of the research in the finite state morphologies has adapted the
    contemporary paradigm of statistically weighted analysis. This is reflected
    in current versions of free and open source morphology of Finnish, omorfi, in
    the finite state morphology part. In this paper we examine two strategies
    of making use of the weights as a part of visl cg 3 pipeline: pre- and
    post-processing, and the numeric operator feature of recent versions of cg
    3. We evaluate the results intrinsically on small sample of analyses we
    have disambiguated by hand ourselves, and extrinsically on the effect it
    has to the rule-based machine translation of that text using freely
    available open source translator, apertium-fin-eng. 
\end{abstract}

\section{Introduction}

In the recent years, use of statistical information in computational
linguistics has gained much interest, with systems like hunpos, moses etc.
being the main points of interest of most research in the field. In finite
state morphology as well as constraint grammars, extensions to handle
probabilities are recent and scarcely
documented~\cite{pirinen2009weighting,bick2009introducing}.  In this paper we
experiment on using an existing weighted finite state morphology of
Finnish~\cite{pirinen2011modularisation}\footnote{\url{https://github.com/flammie/omorfi/}}
both by using the VISL CG-3 feature to handle integer numbers and by a
pre-processing and post-processing scheme turning weights of finite-state
analysis into tags and results of CG application back into weights again. For
CG implementation we use a subset of freely available Finnish CG grammar by
Fred Karlsson~\cite{}, that we have manually converted for this experiment and
to match the analyses produced by omorfi.



\bibliographystyle{acl}
\bibliography{nodalida2015}

\end{document}
% vim: set spell:
