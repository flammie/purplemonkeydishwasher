\documentclass[preprint]{flammie}



\title{Role of Linguistics in Improving Statistical Machine Translation as
Scientific Field and as End-Result}

\iffalse
\author{Tommi A Pirinen\\
Dublin City University\\
ADAPT Centre-School of Computing\\
\url{tommi.pirinen@computing.dcu.ie}}
\date{Last Modification: \today}
\fi

\begin{document}

\maketitle

\begin{abstract}
    In statistical machine translation (SMT), the main objective of the most of
    the work is on improving the translation quality along well-known metrics,
    such as BLEU, METEOR or TER. The purpose of the metrics is ideally to
    provide figures that correlates with the quality of the texts when measured
    again large bodies of professionally translated texts, however, a mere race
    for tiny improvements in these metrics has not proven to be useful for the
    actual quality of machine translation. This seems to be especially true for
    some of the non-IE languages. One example of this is reflected in the
    state-of-the-art of the machine translation of Finnish-English pair, which
    has not been improved upon in the past decade in terms of these metrics.
    Metrics however, are based on simple string comparisons and substitutions,
    and as is argued in the papers published on this language pair that show little to no
    improvement~\cite{clifton2011combining,luong2010hybrid,virpioja2007morphology},
    the metric may not be ideal for morphologically complex language such as
    Finnish. We believe that more linguistically motivated measure for the
    metrics would be more apparent, for example, simple compounding mistake or
    substitution of total object case with partial object case is penalised
    heavily while the fluency and meaning in the translation is mostly
    retained. For example combination of morphologically relevant version of
    BLEU (m-BLEU~\cite{luong2010hybrid}) and word-form-based one might show important insight on the
    conspicuous lack of advance in SMT for this pair. 

    Apart from the non-linguistic metrics used in the evaluation of SMT---from
    a meta-analysis of the state-of-the-art in Finnish-English SMT---we have
    identified two parts of the process that would greatly benefit from
    linguistical view being introduced to the process: the initial hypothesis
    when building a new SMT system and in the final error analysis of the
    result; in fact it appears that in majority of the paper the experimental
    setup is not motivated by any linguistically relevant means and the errors
    of the results have not been looked at at all, let alone in linguistic
    detail that would be necessary for future work to improve upon what is been
    shown.
\end{abstract}


\bibliographystyle{unsrt}
\bibliography{why2015}

\end{document}
