\documentclass[free]{flammie}

\usepackage{expex}

\title{Numerals and what
counts\footnotepubrights{\aclanthologypostprintdoi{2021.udw-1.13}}}

\author{Jack Rueter \\
  Department of Digital Humanities \\
%  Affiliation / Address line 2 \\
%     \\
  University of Helsinki \\
  {\tt jack.rueter@helsinki.fi} \\
  Niko Partanen \\
  Department of Finnish, \\Finno-Ugrian and Scandinavian Studies \\
  University of Helsinki \\
  {\tt niko.partanen@helsinki.fi} \\
  Flammie A. Pirinen \\
  Divvun \\
  Uit Norgga árktalaš universitehta \\
  Tromsø, Norway \\
  {\tt tommi.pirinen@uit.no} \\}

\date{}

\begin{document}
\maketitle
\begin{abstract}
  This study discusses the way different numerals and related expressions are
    currently annotated in the Universal Dependencies project, with a specific
    focus on the Uralic language family and only occasional references to the
    other language groups.  We analyse different annotation conventions between
    individual treebanks, and aim to highlight some areas where further
    development work and systematization could prove beneficial.  At the same
    time, the Universal Dependencies project already offers a wide range of
    conventions to mark nuanced variation in numerals and counting expressions,
    and the harmonization of conventions between different languages could be
    the next step to take.  The discussion here makes  specific reference to
    Universal Dependencies version 2.8, and some differences found may already
    have been harmonized in version 2.9.  Regardless of whether this takes place
    or not, we believe that the study still forms an important documentation of
    this period in the project.
\end{abstract}


\section{Introduction}\label{intro}

Numerous treebanks in the Uralic languages have become available within the
\textit{Universal Dependencies} (UD) project~\cite{ud281}. In recent
years, at least within the Uralic language family, we have seen new treebanks
emerging in languages with closely related siblings that already have an
existing treebank. Examples of such languages are Skolt Saami, in relation to
Northern Saami \cite{tyers2017annotation}, Komi-Permyak, in relation to
Komi-Zyrian \cite{partanen2018first}, or Moksha in relation to Erzya
\cite{rueter2018towards}. Although the entirety of Uralic languages is still not
fully represented within the Universal Dependencies project, the situation has
improved in many ways since the last survey on the state of this language family
in UD was conducted \cite{partanen-rueter-2019-survey}. While more extensive
surveys are useful, we think there are situations where individual nuanced
features should be compared between the languages, so that consistency could be
maintained and improved upon. At the same time, this may provide a thoughtful
point of departure for new discussions around such features, as we believe the
questions discussed here are relevant beyond the realm of Uralic languages.
Even in other treatment of UD on different language groups, such as Slavic,
numerals have been recognized as one category that demands special attention
\cite{zeman2015slavic}. Recently Schneider and Zeldes have also discussed
inconsistent nominal constructions in the English treebanks
\cite{schneider2021mischievous}, and even the issues we describe in the
Uralic treebanks here can well be described in a similar vein. These are not
dramatic issues, but small points of divergence that we could pay attention to,
but if we decide to do so, we would also need to devise strategies to
operationalize the edits in numerous languages with a long history of treebank
work.

We can additionally point to recent discussions within the Universal
Dependencies project where the various ways to annotate English numerical
expressions have been
discussed.\footnote{\url{https://github.com/UniversalDependencies/docs/issues/654}}
Conversations such as these are relevant for Universal Dependencies developers
more widely, and for the sake of consistency such decisions should be at least
considered for the other languages in the project.  Our study also discusses
some numeral types in the Uralic languages that are known, but not yet attested
in the treebanks. Thereby, their description provides an important starting
point for future work on these languages, during which these forms will
inevitably be encountered.

\section{Numerals in Universal Dependencies}

In this paper, we discuss numerals in the Uralic languages. Probably the
simplest approach would be to gather all numeral-type words on the basis of
their \textit{Universal part-of-speech numeral} (UPOS NUM) value or features
making reference to numerals in different Uralic languages. Among the features
at least \texttt{NumType} is one that would be presumed to be present with all
numerals, although it also occurs widely with other parts of
speech.\footnote{https://universaldependencies.org/u/feat/NumType.html} The
possible, currently documented numeral types are cardinal numerals, ordinal
numerals, multiplicatives, fractions, distributives, sets or collective numerals
and ranges. These concepts provide a good base for a relatively elaborate and
nuanced system, but at this phase the UD system appears slightly asymmetric.

Potential asymmetry might be dealt with by adding a binary for the split between
numerals and counted nouns versus nouns with sequential deixis-like marking. In
the Erzya, Moksha languages, sequential deixis is readily attested in
combination with multiplicatives and sets, but due to the fact that ordinals
only comprise three combinatorial instances in Erzya, it may strike us as
fruitless to introduce a plus/minus binary for ordinal. The Erzya examples below
illustrate this.

\begin{itemize}
\item nummod [-Ord] \textit{vejke} `one'
\item nummod [-Ord][+Approx] \textit{kavtoška} `couple'
\item nummod [-Ord][+Sets] \textit{kavonst} `two pairs/sets'
\item  nummod [-Ord][+Dist] \textit{kavtoń-kavtoń} `two-by-two'
\item  advmod [-Ord][+Mult] \textit{kavkśt'} `twice, two iterations of the verb'
\item  advmod [-Ord][+Mult] \textit{kavońkirda} `twofold, double the amount'
\item advmod [-Ord][+Mult][+Approx] \textit{kavkst'eška} `a couple of times'
\item advmod [+Ord][+Mult] \textit{omboćed'e} `for the/a second time' % омбоцеде
\item advmod [+Ord][+Mult] \textit{ombońkirda} `a second time'
\item amod [+Ord][+Sets] \textit{ombonst} `a second set' %омбонст
\item amod [+Ord] \textit{omboće} `second'  % омбоце
\item  det [-Ord][+Tot] \textit{kavońeńek} `the both of us' % кавоненек
\item  det [-Ord][+Approx][+Tot] \textit{kevet'eješkańest} `the approximately 15 of them' % кавоненек
\item det [+Ord] \textit{ombot'ks} `the second' % омботькс
\end{itemize}

Above, we can observe that the approximatives and distributives including
universal quantifiers are not associated with sequential deixis in Erzya.
Whereas, sequence and range might readily be combined. In counting iterations of
a predicate, Erzya shows a clear distinction between it and quantification of
mass (`twice' and `twofold' cannot be equated), but this distinction becomes
less obvious when applied to a sequential deixis system. A glimpse at
Komi-Permyak and Komi-Zyrian will remind us that multiplicatives may also be
used in a distributive context \cite[22]{rueter2020questions}. Multiplicatives,
sets, distributives, etc. should not be distinguished from ordinals any more
than they are from cardinals, since the term cardinal might readily be treated
as a \textsc{zero} like nominative singular.  The last three items within the
list above are also exceptional as they would demand syntactic dependency `det',
which according to the guidelines is not allowed. Analogically, chosen
conventions could possibly also be extended to the annotations of items such as
English `both' and Swedish `bägge'.

Conceivably, numerals might be divided into various categories according to
their semantic use. The most predominant numeral types might therefore be
associated with quantification, sequence, and entity naming.  Quantification
articulates distinctions in the mechanisms of counting. Singular entity counting
is typified by the use of cardinals (such as in Finnish \textit{yksi} `one',
\textit{kaksi} `two', \textit{kolme} `three', etc.), and there may be different
marking patterns for the counted noun.

In many languages, there are standards by which the head noun of a
\textit{nummod} dependency takes special marking.  In Komi-Zyrian, Komi-Permyak
and Hungarian, for example, the counted noun shows no deviance from its regular
nominative singular marking strategies when qualified by any cardinal numeral.
In Balto-Finnic, Finnish, Estonian, Livvi and Karelian, the partitive singular
marks the counted nouns when they are qualified by numerals two and above, even
though their syntactic position would otherwise call for a nominative
singular---for other cases a fitting semantic or syntactic case is used, i.e.
phrase agrees in case.

\pex
\a
\begingl
\gla kolme šukupolvie //
\glb three.\textsc{nom.sg} generation.\textsc{par.sg}//
\glft `two generations' (krl: vepkar-1652.40)//
\label{gloss:krl-numcoord1}
\endgl
\a
\begingl
\gla kuutta kertua enemmän //
\glb six.\textsc{par.sg} time.\textsc{par.sg} more //
\glft `six times more' (krl: vepkar-1740.21)//
\label{gloss:krl-numcoord2}
\endgl
\a
\begingl
\gla šuašša muašša //
\glb hundred.\textsc{ine.sg} land.\textsc{ine.sg} //
\glft `in a hundred lands' (krl: vepkar-1740.6)//
\label{gloss:krl-numcoord3}
\endgl
\xe

Contrastively, the Mordvin languages, Erzya and Moksha, exhibit a variation that
has yet to be researched in depth, i.e. counted nouns do not obligatorily take
special marking when qualified by cardinal numerals two and upward, see
Markov~\cite[42]{markov1961prialatyrski} and
Rueter~\cite[107]{rueter2013quantification}, but perhaps also in dialect
studies
\cite{ryabov2016erzdialectresearch,rueter2016erzisogloss,levina2021korpus,agafonova2021azorkschi}.

A similar phenomenon can be observed in Moksha \cite{rueterForthMordvin}.  The
Saami languages attest to two different strategies: Northern Saami takes
genitive singular marking of its counted nouns when qualified by numerals two
and above, whereas Skolt Saami makes a three-way split, a genitive singular
marking the numeral range 2--6, and the partitive marking seven and upward (with
the decline in language proficiency the use of the partitive has become less
certain).


Sets of entities, i.e. sets with more than single members, are counted
synthetically across the languages with various strategies.  In Finnish, for
example, pairs of scissors are counted by using plural forms of the cardinal
numerals and the \textsc{np} head noun alike, e.g. \textit{yhdet sakset} `one
pair of scissors' (here both the numeral and the noun it qualifies are in the
plural, and unlike Russian the distinction is retained for numerals five and
above, too).  In contrast, Erzya has its own numeral forms typically derived in
\textit{-Onst}, hence \textit{kavonst vasońpejeĺt'} `two pairs of scissors' with
the counted noun in the plural.  Although numerals of the sets type are
typically introduced for counting pairs, they are, in fact, often used with
larger sets, such as sets of six cups and saucers.

Iterations of predications are often counted with adverb derivations of cardinal
numerals, but the productivity of these derivations still requires assessment
from language to language. While Finnish only minimally utilizes the word forms
in \textit{-sti}: \textit{kahdesti} `twice', \textit{kolmesti} `thrice' and
\textit{tuhannesti} `a thousand times', the Hungarian, Komi-Zyrian,
Komi-Permyak, Erzya and Moksha languages use regular derivations for indicating
`X times', \textit{-szer/-ször/-szor}, \textit{-ïś}, \textit{-iś},
\textit{-kśt'} and \textit{-kśt'}, respectively. Needless to say, matters become
confusing when these iterative numerals are categorized as multiplicatives in
UD. The result, at least in Erzya, is that `being paid \textit{kavkśt' = twice}'
and `being paid \textit{kavońkirda = double} or \textit{twofold}' are registered
as the same thing, which is by no means always the state of affairs
semantically, but from a syntactic perspective it is plausible.

Distributive numerals are not a simple class. They can be further categorized
into subclasses, as immediately becomes apparent in the two Hungarian
strategies: \textit{két-két} `two each' with a noun head, and
\textit{kettesével} `two at a time' with a verb head. Whereas the former may be
used as a definite numeral in the context \textit{Berta és Rudi két-két csomagot
hozott} `Berta and Rudi brought two suitcases each', implying that a total of
four suitcases were brought, the latter expression is indefinite. The indefinite
distributive numeral \textit{kettesével} `two at a time' in nearly the same
context \textit{Berta és Rudi kettesével hozta a csomagokat} `Berta and Rudi
brought the suitcases two at a time'\footnote{cf.
http://en.utdb.nullpoint.info/type/hungarian/distributive-numerals/dupldnn-sufdnv}
would indicate that each iteration of the predication involves two suitcases,
but there is no indication regarding the number of iterations -- it could be any
number of times. In this context, definiteness is lent by the object, i.e. `the
suitcases'.


Approximative numerals are numerals with values slightly less or more than the
number given. Finnish, for example, attests \textit{parikymmentä} `about twenty'
from the words \textit{pari} `couple' and \textit{kymmentä} `ten (partitive)'.
In addition to constructions with the element \textit{pari}, there are fairly
regular derivations formed from other basic numerals as well: \textit{kolmisen}
+ \textit{kymmentä} `approximately thirty'.

In Erzya, as in Moksha, approximative forms in \textit{-ška} are found for
counting entities \textit{vet'eška lomań} `about five people' and iterations
\textit{kolmoškakst'} `about three times'. With the use of an approximative
numeral, the likelihood rises that no plural marking is indicated on the counted
noun. The predominance of nominative singular marking of the \textsc{np} head
also holds when the approximative is marked with an N–(N + 1) strategy, i.e.
\textit{vet'e-koto lomań} `five-or-six people'. The use of adjacent numerals to
indicate approximate values is also found in Komi-Zyrian, i.e.
\textit{vit-kvajt} and \textit{vit-ö-kvajt} both translate to five or six.

In Finnish, the expression of range with numerals follows the same pattern as is
observed in point of departure to end destination, i.e. the elative case marks
the starting point, and the illative marks the end point. In the range 5--7
kilometers, the Finnish involves \textit{viidestä seitsemään kilometriä}
five+elative, seven+illative and kilometer+partitive, which is the same counted
noun strategy observed in basic numerals.

Fractions in Finnish can be expressed in at least two different ways. One way is
to join the ordinal nominative singular with the noun \textit{osa} `part', hence
\textit{viides} + \textit{osa} = \textit{viidesosa}, where only the end is
declined and as such is distinguished from `the fifth part' of something, where
we would actually be talking of sequences. Syntactically, \textit{neljä
viidesosaa} `four fifths' functions in the same manner as any noun with a
cardinal qualifier, i.e. the \textsc{np} head is marked with the partitive
singular when in an otherwise nominative-singular position
nummod(\textit{viidesosaa}, \textit{neljä}). The second derivational expression
for `fifth' is \textit{viidennes}, it too is treated syntactically as a counted
noun, as appears to be the case in other Uralic languages.

Universal quantifiers, such as the Finnish \textit{molemmat} `both', have more
complex counterparts in Hungarian \textit{mindkettő} (literally `all' + `two'),
which may also take associative marking for first, second and third persons
plural in \textit{mindkettünk}, \textit{mindkettetek}, \textit{mindkettük},
respectively. The Hungarian \textit{mindhárom} `all three', `tous les trois'
then comes as no surprise, and one begins to expect subsequent \textit{mindnégy}
`all four'. Komi-Zyrian and Erzya attest to yet another aspect: the associative
personal reference can also be in the singular, allowing for access. If we are
speaking of a singular `person' and mention that `the (lit.) three of him/her
are moving to town' \cite{rueter2013quantification}, we access a definite
universal quantifier pronoun with reference to this single person. This feature
is not observed in Hill Mari or Udmurt \cite[111--112]{kelmakov2008udmurtin}.
Ordinal numerals can be associated with multiplicative, iterative and sets
features. This has been observed in the presentation of some morphology for
Erzya, above.

Numerals appear in entity naming, for example the Finnish \textit{viitonen}
$\sim$ \textit{vitonen} `fiver' may be used when making reference to money, on
the one hand, but it could also be used in reference to a street car, where we
would be more likely to translate it as `street car five' or `street car number
five'. Thus is fits directly into a list of problems in apposition, such as `the
color purple', `the word terrorist' and many others including numerals discussed
by Schneider and Zeldes~\cite{schneider2021mischievous}. An extension to
this numeral issue is found in Finnish \textit{viitonen} in reference to `house
number five', but the same \texttt{5} is transformed to the cardinal-form
\textit{viisi} if the house is \texttt{5a} or \texttt{5b} -- \textit{viisi a} or
\textit{viisi b}, respectively (no partitive, of course, so we are not counting
letters). Here, the Erzya solution is to use the ordinal \textit{vet'eće} `the
fifth' for \texttt{5} and \textit{vet'eće a} `fifth a' for \texttt{5a}, which
results in ambiguous homonymy.

There are differences observed across languages, where synthetic versus analytic
expressions of the same numerical values might be dealt with differently. Thus,
our first overview discusses the largest spread of numeral types, forms across
languages. Once the collection is complete, the numeral words can be classified
according to the dependencies and features. In Finnish, for example, we predict
four different and regular dependencies: nummod (for cardinals and plural
cardinals with plurale tantum), advmod (for counting iterations of a
predication, e.g. once, twice, thrice), advcl (for distributive quantification),
amod (for ordinals). Other languages, it will be noted, may have extensive det
(this is not really productive in Finnish, but would be the equivalent for
`both' and its analogues with universal quantification of numbers three and up,
probably with person marking as well, e.g. `the two of us').

\subsection{Numeral type}

According to the Universal Dependencies documentation, some numerals can be
classified as adjectives and some as
adverbs.\footnote{\url{https://universaldependencies.org/u/pos/NUM.html}}
Thereby, in the UD guidelines both \textit{ADV} and \textit{ADJ} are often found
as the part of speech categories for numeral expressions. At the same time,
there are also situations where the NumType feature occurs with different parts
of speech.

In several treebanks in the Romance languages, for example, there are pronouns
such as Spanish \textit{mucho} and \textit{poco} which have a feature value
NumType=Card. Such marking on pronouns is not common in the treebanks, although
we do find English \textit{first}, \textit{second}, \textit{third} and
\textit{latter} receiving POS tag \textit{PRON} and feature NumType=Ord. This is
also the style in Finnish, with \textit{toinen} `second; another' being marked
similarly, and Erzya and Komi-Zyrian treebanks offer similar examples. As the
combination PRON and NumType can be found only in treebanks for 10 different
languages, we believe it is highly likely that similar annotations could be
extended to many other languages within the project.

Nouns that are marked with NumType appear in a bit larger array of languages,
all in all within 13 languages, among them, Uralic languages North Saami, Erzya
and Estonian. In North Saami, these instances are collective nouns with
NumType=Coll. In Erzya word \textit{pel'} `half' is marked with NumType=Frac. In
Estonian the only occurrences are with gene names containing numbers, such as
IL-5, where NumType=Card is attested. These are all reasonable uses of NumType,
as these noun types do have countable properties that are relatively well
captured by the NumType feature. But again as the solutions seem language
specific the annotations could be somehow harmonized or extended to more
languages.

In Finnish, Icelandic and Korean treebanks we find examples of punctuation being
marked with NumType=Card. No matter how the annotation is motivated, being this
rare and narrowly distributed is possibly problematic for the comparability of
the languages.  The Estonian treebanks EDT and EWT only use NumType with two
values, Card and Ord.  This does not appear to rule out fractions, but they are
dealt with differently, i.e. 3/4 is given the features NumForm=Digit and
NumType=Card. Of course, here the value \textit{Digit} indicates not written as
words.  A second issue in EWT is that the feature \texttt{NumType=Ord} is used
with both UPOS \textit{NUM} and \textit{ADJ}. It seems that ordinal digital
numerals consisting of an Arabic numeral followed by a full stop are treated as
\textit{ADJ}, whereas automobiles from different years have an abbreviated year
digit pair followed by an apostrophe. This latter type has the UPOS value
\textit{NUM}, should this be the case?  We will not widely compare the
differences between multiple treebanks on the same language, although we do
acknowledge this is an issue that needs further attention.


Having discussed the general use of NumType feature and some rarer patterns that
can be found, we will next describe more in detail different numeral types and
their occurrences, with references both to Uralic and other language families,
as necessary.

\subsubsection{Cardinal numerals}

The cardinal numeral type in UD is typified as an expression for counting
singular items. Thus, this feature might be associated with the UD part of
speech NUM (as in one, two, three, etc.). This feature value is also used with
non-numerals (as in \textit{many}, \textit{few}, Czech \textit{kolik} 'how
many', etc.). Here, however, individual languages make a split between use of
UPOS \textit{DET} and \textit{NUM}. The latter of which, apparently, is defended
in Czech by a strong grammatical tradition, might be used for the interrogative
\textit{kolik} 'how many', which evokes cardinal numerals.  Czech includes yet a
third type of words as cardinals which seem to indicate the total number, e.g.
\textit{čtvero} (as in \textit{Čtvero ročních dob} `The Four Season', all four),
\textit{desatero} (as in `the Ten Commandments', all ten). This presumably
explains the definition of \textit{oba} `both', which in Czech is marked as UPOS
\textit{NUM}, whereas Talbanken deals with \textit{bägge} `both' as a
\textit{DET}. And then there is the one instance of \textit{desatero} in the
treebank \textit{Desatero investora} `Lit. The ten investors', where the word
\textit{desatero} has the UPOS \textit{NOUN}.

This third group of cardinals, which is not observed in Swedish as a consistent
counting system, appears with a nummod dependency in Czech to match the UPOS
\textit{NUM}. In Swedish and other languages without this counting system, words
with the meaning `both' are generally dealt with as \textit{DET}, and they have
a feature \textit{PronType=Tot}.

\subsubsection{Ordinal numerals}

Ordinals can be seen to represent subtypes of adjectives and adverbs. In
addition to the amod dependency associated with the words \textit{first},
\textit{second}, \textit{third}, there are analogical interrogatives, etc.),
there is also an advmod dependency, associated with ordinal multiplicatives,
such as the Czech \textit{poprvé} `for the first time'.  By applying the feature
value NumType=Ord to both UPOS \textit{ADJ} and \textit{ADV}, we could remove
the NumType=OrdMult feature value used in Komi-Zyrian \textit{ńol'öd} `fourth'
UPOS \textit{ADJ} and \textit{ńol'ödyś} `for the fourth time' UPOS \textit{ADV}
and similarly in Erzya, Moksha and Komi-Permyak. The downside is that the
parallel between cardinal and ordinal multiplicatives becomes less obvious. If
we were to do so, we would be faced with the challenge of addressing numerals
with three features: ordinal multiplicative and distributive.

Numerals can be classified according to what they actually enumerate or do they
at all.  In Erzya, the numeral type (a) « вейке, кавто, колмо,
ниле... » \textit{vejke, kavto, kolmo, ńil'e} is used for counting
individual entities. The pertinent dependency is nummod.  (b) «
веенст, кавонст, колмонст, ниленст... » \textit{vejenst, kavonst,
kolmonst, ńil'enst} is used for counting set entities from pairs of scissors to
sets of cups. The pertinent dependency is nummod. NumType=Sets (c)
\textit{vest', kavkst', kolmokst', ńil'ekst'} « весть, кавксть,
колмоксть, нилексть... » is used for counting iterations of a
given predication. Thus this has a advmod dependency. NumType=Mult (d)
Delimiting associative collectives « ськамост, кавонест, колмонест,
ниленест... » \textit{śkamost, kavońest, kolmońest, ńilenest}
provide universal quantification values found in the expressions `alone',
`both', `all three' with the addition of associative reference to number and
person. These numerals are used in secondary predication with reference to the
subject or object. Features include PronType=Tot (e) Distributive, imperfect
\textit{kavtoń-kavtoń, kolmoń-kolmoń, ńil'eń-ńil'eń} «кавтонь-кавтонь,
колмонь-колмонь, нилень-нилень...» NumType=Dist Aspect=Imp (f) Distributive,
perfect \textit{kavtoń-kavto, kolmoń-kolmo, ńil'eń-ńil'e} «кавтонь-кавто,
колмонь-колмо, нилень-ниле,...» NumType=Dist Aspect=Perf (g) \textit{vejeńkirda,
kavońkirda, kolmońkirda, ńil'eńkirda} «веенькирда, кавонькирда, колмонькирда,
ниленькирда,...» has an advmod or amod dependency, and the feature value
NumType=Mult.

\subsection{Numeral dependencies}

Among the dependency relations assigned to the numerals, the most common is
\textit{nummod}. In many Slavic treebanks an additional relation of \textit{det}
is used, as in \textit{det:nummod}. This is not used in other treebanks. In Beja
treebank there is an individual occurrence of \textit{nummod:det}. Another
subtype of \textit{nummod}, \textit{nummod:entity}, appears to be used only in
the Russian treebanks, especially in relation to the symbol `№'. Additionally
\textit{nummod:flat} appears only in one Polish treebank. Phenomena attested and
seen necessary to annotate in the Slavic languages could also be very relevant
for work with the Uralic languages, many of which have been in extensive contact
with Russian.

Our analysis also indicates that the relation \textit{nummod} in the Uralic
languages virtually always connects to part of speech \textit{NUM}. With the
other languages, there is extensive variation, even though this relation is
always the most common. Whether this is simply a matter of annotation
conventions, linguistic description traditions or actual linguistically relevant
differences, remains to be studied.

\section{Discussion}

As we have shown, numerals and related expressions are an area for fruitful and
needed further discussion in the Universal Dependencies project. Which forms all
get numeric features extends widely beyond just numerals themselves, and many
lexical items that have counting properties could be annotated with NumType
features, and already be annotated in different treebanks. Which of the
individual solutions in different treebanks should be described better in the
documentation and adapted further, and which should be harmonized in comparable
uses of the treebanks, remains to be discussed, but we hope our observations
help at least a bit along this path. Of course, how work on various
inconsistencies should or could be coordinated across the hundreds of treebanks
already in the Universal Dependencies project is not entirely clear, and remains
certainly a large challenge. At the same time, new treebanks are still
continuously emerging, and paying attention to various strategies used in
existing treebanks should help the maintainers of these new languages to adapt
their conventions. When diverse language families are included, new questions
inevitably arise. For example, in Apurinã there are very few actual cardinal
numbers and quantification is expressed in verbal constructions
\cite{facundes2021numexpr,rueter2021apurina}.

The issue how to handle Komi-Zyrian numerals was also recently discussed in the
relation to Komi morphological analyser \cite[67]{rueter2021komifst}, which
points to the fact that the best possible annotation scheme is often a very
relevant question for uses beyond the Universal Dependencies project itself. We
also believe that the classification and annotation of numerals is important
from the point of view of basic linguistic research and language description. As
the description of Erzya counting expressions in this study showed, the system
is already very complicated and nuanced in this one language, and is just
starting to be adequately described in the newest grammatical descriptions
\cite{suihkonen2013typology}. We presume the description of many smaller Uralic
languages remains much less complete, not to even mention less studied language
families of the world, which also have started to have significant presence in
the Universal Dependencies project. This kind of easily accessible information
about counting expression at large could be immediately beneficial, for example,
in typological research, and systematic annotations and documentation in
projects such as Universal Dependencies is one modern way to distribute this
description.



\bibliographystyle{unsrt}
\bibliography{anthology,udw2022}

\end{document}
