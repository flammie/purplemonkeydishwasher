\documentclass[b5paper]{article}

\usepackage{polyglossia}
\usepackage{fontspec}
\usepackage{xunicode}
\usepackage{xltxtra}
\usepackage{url}
\usepackage{hyperref}
\usepackage{expex}

\makeatletter
\def\blfootnote{\gdef\@thefnmark{}\@footnotetext}
\makeatother

\setmainfont[Mapping=tex-text]{Linux Libertine O}

\begin{document}

\title{Template for FIWCLUL 2015 Workshop Proceedings\blfootnote{
    This work is licensed under a Creative Commons Attribution–NoDerivatives
    4.0 International Licence.  Licence details:
    \url{http://creativecommons.org/licenses/by-nd/4.0/}
}
}

\author{First Author\\
University or Affiliation\\
Department or School\\
(Optionally Post Address)\\
\url{first.author@example.com} \and
Second Author\\
University or Affiliation\\
Department or School\\
(Optionally Post Address)\\
\url{second.author@example.com} 
}

\date{\today}

\maketitle

\begin{abstract}
    Abstract can be between 150 and 300 words long. Separate keywords are not
    used. Reviewers use abstract to decide which articles to review and readers
    will use abstract to decide which articles to read, so make it informative
    and interesting.
\end{abstract}

\section{Introduction}

This template is for authors submitting papers to First International Workshop
on Computational Linguistics for Uralic Languages. The papers to be submitted
should be in PDF format made using \XeLaTeX{} and this template; that is
documentclass article with b5paper option. The standard xelatex packages
including polyglossia should be loaded for all papers.  Use Linux Libertine as
freely available font with good range of unicode support, and small-caps for
Leipzig glossing. The proceedings are designed for viewing on screen but
for-print versions may be produced.  Authors who are unable to use freely
available \texttt{xelatex} should contact the conference organisers for other
options. Additional information and recent updates to this template are
available from \url{http://gtweb.uit.no/iwclul2015/}. The submissions are
handled using easychair. Required tex packages should be in your texlive
distribution, for ubuntu use: \texttt{sudo apt-get install
texlive-\{xetex,latex-recommended,fonts-recommended\} fonts-linux-libertine}.

To ease the work with licences and permissions, all submissions should retain
a footnote acknowledging that work is to be published under Creative
Commons Attribution-NoDerivatives 4.0 International Licence. This way we do not
need additional copyright forms.

\section{Formatting Instructions}

Follow common English typographical standards. The \emph{emphasis} is used for
sentence level highlights and \textbf{strong emphasis} for paragraph level
highlights. 

Linguistic glosses should follow Leipzig Glossing rules
\url{http://www.eva.mpg.de/lingua/resources/glossing-rules.php}, see (\nextx).

\ex
\begingl
\gla talo-i-ssa-ni //
\glb house-{\sc Pl}-{\sc Ine}-{\sc 1sg} //
\glft `in my houses' //
\endgl
\xe

For inline examples, use emphasis for original and parentheses for translation
\emph{talossasi} (in your house).

Tables like table~\ref{table:example}. should have caption underneath and
borders.

\begin{table}
    \center
    \begin{tabular}{|l|r|}
        \hline
        \bf Header & \bf Header \\
        \hline
        \bf Header & Data \\
        \hline
    \end{tabular}
    \caption{The rows are things and columns stuff and data is in percent units
    \label{table:example}}
\end{table}

Citations are made using bibtex: WALS~\cite{haspelmath2005world} is a good book.
It is possible to cite Russians using plain unicode, such as the original 
version of Levenshtein algorithm~\cite{levenshtein1965}.
Use bibliographystyle unsrt. If you get your bibtex snippets from google
scholar, don't forget to double check the data and add missing parts.

Code examples, program listings and shell sessions can be shown as figures,
e.g. figure~\ref{code:analysis}.
Pseudo-code may be formatted using one of the relevant \LaTeX{} packages, e.g.,
\texttt{algorithm}.

\begin{figure}
    \center
    \begin{verbatim}
    $ hfst-lookup src/morphology.ftb3.hfst
    > talossani
    talossani talo+N+Sg+Ine+PxSg1
    \end{verbatim}
    \caption{Analysing Finnish from command-line with \texttt{hfst-lookup}
    \label{code:analysis}}
\end{figure}

Inline formulas should use standard tex environment: $P(\mathrm{is}) >
P(\mathrm{unlikely})$. Pay attention to the use of roman font for text examples
and multicharacter variables and indexes. Long formulas should be numbered in
equation block:

\begin{equation}
    P(\hat x) = \frac{x+\alpha}{y+\alpha\times z}
\end{equation}

Remember to describe each function, variable and notation you use.

Use a splel-checker before submitting your texts.

\section*{Acknowledgments}

Acknowledgments should be un-numbered last section.

\bibliographystyle{unsrt}
\bibliography{fiwclul2015}

\end{document}

