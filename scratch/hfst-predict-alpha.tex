\documentclass{beamer}

\usepackage{fontspec}
\usepackage{polyglossia}

\usepackage{graphicx}
\usepackage{color}
\usepackage{url}

\mode<presentation>
{
    \usetheme{Divvun}
}


\title{HFST-predict experiments
\scriptsize{UiT---UoA sessions \today.
\url{https://github.com/flammie/hfst-predict}}}
\author{Flammie }
\institute{UiT}
\date{\today}

\begin{document}

%\selectlanguage{english}

\maketitle

\section{HFST-predict}

\begin{frame}
    \frametitle{Introduction}
    \begin{itemize}
        \item Currently:
            \texttt{hfst-ospell}\footnote{\url{https://github.com/hfst/hfst-ospell}}
            $\sim$
            \texttt{divvunspell}\footnote{\url{https://github.com/divvun/divvunspell}}
            $\rightarrow$
            LibreO, MSO, Google something, mobile keyboarDs?
        \item Experiments:
            \texttt{hfst-predict}\footnote{\url{https://github.com/flammie/hfst-predict}}
        \item todo: implement in \texttt{divvunspell} features (analysis,
            prediction)?
    \end{itemize}
\end{frame}

\begin{frame}
    \frametitle{Spelling correction vs.\ prediction}
    \begin{itemize}
        \item Correction: full non-word to dictionary word:-form
            \textit{xat[space]} --- \textit{cat} etc.
        \item Prediction: word-part to dictionary word-form:
            \textit{kod\ldots} --- \textit{kodissa}
        \item NB:\ prediction should include correction: if you write
            \textit{ahku\ldots} --- \textit{áhkus}
        \item we want to predict parts or part (compound part, morphs,\ldots):
            \textit{kod\ldots} --- \textit{kodin\ldots} ---
            \textit{kodinkonekauppakeskuspaikkajajne}?
    \end{itemize}
\end{frame}

\begin{frame}
    \frametitle{HFST-ospell logic}
    \begin{enumerate}
        \item input word \textit{caqt} (string)
        \item error model \texttt{q:0::1} or \texttt{q:r::1} (2-tape
            FSA)\begin{itemize}
                \item Levenshtein Damerau Edit Distance!
                \item and change diacritics
                \item and try linguistic corrections
                \item and other known confusables
            \end{itemize}
        \item dictionary: \textit{caqt} ``?'' or \textit{cat} ``common'' or
            \textit{cart} ``maybe less common'' (FSA)
        \item (actually an analyser: \texttt{caqt+? inf} \texttt{cat+N+Sg}
            \texttt{cart+N+Sg} (2tape FSA))
        \item NB:\ dictionary is used both for checking input and suggestions
    \end{enumerate}
\end{frame}

\begin{frame}
    \frametitle{Future logics}
    \begin{enumerate}
        \item input word
        \item error model
        \item \textbf{prediction model} \begin{itemize}
                \item \texttt{0:?*}!
                \item up until morph / word boundaries
                \item there is no difference b/w error and prediction model in
                    FSA calculus
        \end{itemize}
        \item dictionary (including unfinished morph concatenations? Cannot be
            same as input acceptor then?)
        \item (weigh-er)
        \item (analyser; could be used as dictionary accepting words and
            predictons with some tagging)
    \end{enumerate}
\end{frame}

\begin{frame}
    \frametitle{Limitations}
    \begin{itemize}
        \item Trade-offs for pre-combining models (speed,
            size):\begin{itemize}
                \item This is why error model + dictionary was originally kept
                    apart mostly (c.f. Pirinen et al. 2008 or smth)
                \item prediction + dictionary is also too heavy in general case
                \item dictionary + likelihoods can be heavy
                \item prediction / error model?
            \end{itemize}
        \item we need good dictionaries of morphemic completions (c.f.\
            \texttt{>}
            in our FSA?)
        \item Weights in any model can be used
    \end{itemize}
\end{frame}

\begin{frame}
    \frametitle{Prediction.zhfst?}
    \begin{itemize}
        \item 0---* error / prediction / mutation models (1 + 1)
        \item 1---* acceptor / analyser models (1 + 1)
        \item 0---* weighing models (as needed by optimising)
    \end{itemize}
    e.g. 1 error + 1 acceptor like old spellers, or + 1 prediction model and +1
    acceptor model (analyser or accepting morphs)
\end{frame}

\begin{frame}
    \frametitle{What's missing?}
    hfst-ospell:
    \begin{itemize}
        \item just support for few more automata / type
    \end{itemize}
    divvun-spell:
    \begin{itemize}
        \item that above plus
        \item analysis functionality
    \end{itemize}
    Possibly more?
\end{frame}


\begin{frame}
    \frametitle{Diversion? User interface etc.\ issues}
    \begin{itemize}
        \item there's no pre-exising ui for morph / compound completion in
            mobile keyboards and the UI is pretty overloaded
    \end{itemize}
\end{frame}


\end{document}


