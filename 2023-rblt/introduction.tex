\documentclass[free]{flammie}
%\documentclass[a4paper,draft]{article}
% (You'll need to use the class and packages provided by the conference for
%  submission anyways)

\newif\ifcameraready
\camerareadytrue

\usepackage{polyglossia}
\usepackage{fontspec}
\usepackage{xunicode}
\usepackage{xltxtra}
\usepackage{url}
\usepackage{hyperref}
\usepackage{expex}
\usepackage[obeyDraft]{todonotes}
%\usepackage{todonotes}


\begin{document}


\title{Rule-Based Language Technology:  Introduction \footnotepubrights{
    This work is licensed under a Creative Commons Attribution–NonCommercial-NoDerivatives
    4.0 International Licence.  Licence details:
    \url{http://creativecommons.org/licenses/by-nc-nd/4.0/}.
    Find the Publisher's version from:
    \url{https://dspace.ut.ee/handle/10062/89595}
}}


\ifcameraready
\author{Flammie A Pirinen \\
    \url{flammie.pirinen@uit.no} \\
    Divvun, UiT Norgga árktalaš universitehta \\
}
\fi

\maketitle
\begin{abstract}

    ABSTRACT
\end{abstract}

Rule-Based Language Technology
2023
NEALT Monograph Series, 2
4
Introduction
Flammie A Pirinen
Department of Language and Culture / Divvun
NO-9019 UiT–The Arctic University of Norway
Flammie.pirinen@uit.no

\section{Introduction}

Rule-based language technology (RBLT) refers to a language technology approach
grounded in rules. As rules we understand grammatical rules written by linguists
but in a broader view, also any world knowledge and logical deductions are
rules. The rule-based language technology is indeed often seen as a combination
of general linguistics and software engineering or computer science. The purpose
of the computational component can be seen as a research question in its own
right or as a practical engineering problem of how we express the linguistic
grammar in a computational form to be used by the computer.

One of the core ideas of linguistics as it is understood by language technology
engineers is the ability to abstract away from the specifics in order to
generalize the thinking.  That is, for example, to understand that a
relationship between wordforms ‘cat’ and ‘cats,’ and ‘dog’ and ‘dogs’ is the
same, is a very useful piece of information in rule-based language technology,
and eventually understanding that perhaps the same relationship applies between
‘mouse’ and ‘mice’. Modelling it accurately is the task we often refer to when
talk about language modelling.  Modelling of a whole language naturally
encompasses an entire range of linguistic and even more general notions such as
world knowledge, not just the dictionary of words and morphology. I will detail
these later in this chapter.

An idealistic goal of the rule-based language technology is to encode all
information in the words and their relations in texts in such a way that the
computer can deduce the exact interpretations and make deductions on them. Of
course, the main counterargument to this is that languages are ambiguous and
encode information only partially, leaving much information to be discovered by
using world knowledge or contextual evidence. Therefore, a perfect rule-based
system is not always perfect language technology software (but can often produce
perfect listing of all possibilities that are plausible).

There are numerous approaches, software, toolboxes and theories to rule-based
language technologies, and this book is naturally only going to cover a subset
of them.  Specifically, one big branch of rule-based technology that is heavily
featured in this book is computational morphology, that originated in the
context of European and Uralic languages; the approach has been successfully
used for languages of most typological branches thereafter and includes tools
and technologies that go well beyond morphology and into syntax and semantics of
language as well. Certain linguistic aspects and concepts are common to
virtually all rule-based language technologies, and I will try to cover them in
the rest of this introduction in a generic manner.

\section{History and current trends into 2020s}

In the early days of language technology, a rule-based approach was the norm. It
was not until the 1990’s when an alternative to rule-based approaches started to
gain traction in the language-technology communities to such an extent that the
term rule-based language technology became a common term. The alternative that
came about was that of Statistical Language Technology. In statistically based
approach of language technology, less emphasis is given on the grammar and
linguistics, and the basis of computational modelling of the language lies in
feeding computer enormous quantities of texts (or speech) and using mathematical
algorithms to learn certain characteristics of the language. From this point of
view the two approaches to language technology have also been called
expert-driven, referring to the rule-based language technology, and data-driven,
referring to the statistical language technology. From the data-driven language
technologies, a branch was popularized in the 2000s, based on neural network
technology used in artificial intelligence research, which is at the peak of its
popularity now in 2022 as we are writing this book.  The neural networks are in
fact so overwhelmingly popular that rule-based approach at the time of writing
almost has a reputation of being outdated and useless. It is one of our aims of
this book to show this preconception as false.

Many researchers see these approaches to language technology as mutually
exclusive and competing. However, this is not at all necessary and there are
several approaches that make use of rule-based components and stochastic
methods, forming a kind of technology often referred to as hybrid language
technology.

The fundamental issue in the current trend of forgetting linguistics and
pursuing statistical language technology models is that it is based on research
and experimentation with only such languages that have a very long written
history with practically endless amounts of written texts that machine can learn
from. This is not the case for the majority of world’s languages; according to
the Ethnologue (2022), a language catalogue published by SIL international,
there are 7151 living languages in the world in 2022. It is quite unlikely that
more than top 100 of these have available linguistic resources for the
datadriven learning of the language\footnote{It is noteworthy, that several of
these languages are oral and not written, and many more have only oral tradition
with no pre-existing writing system. There exist language technologies for oral
languages that are beyond the scope of this book; for mainly spoken languages
some are aspiring to have written standard and there are also contents on spoken
language processing in this book while that is not the main component of this
book.}. When discussing lesser resourced and minority languages, one must also
keep in mind that the situations of text corpora can be very different to the
majority languages. There may not be stable standard for written language, or
one is very new or is controversial, and the majority of the text data follows
different standards. It is vastly more difficult for a machine to learn to
understand and produce texts following the standard if it is not represented in
the data that is available. With rule-based approaches, the expert constructing
the language technology can describe and prescribe the language norms as needed.
For this reason, a rule-based approach can be particularly useful for language
revitalization and for support of endangered languages.


\section{Linguistic concepts}

Rule-based language technology draws its information from linguistic grammar and
data, and much of this data has commonalities across theories and frameworks as
well as software toolboxes. One of the most used sources of linguistic
information in language technologies is a dictionary. All language technology
performing rule-based processing of texts (and many working on speech signals as
well) need to know information about words and word-forms. The scientific fields
pertaining to modelling and understanding of words and word-forms span from
lexicography to morphology. The task of modelling the words and their forms
might sound like a simple concept, but there is a lot of detail that goes into
modelling the lexicon of the language and its computational presentation. One of
the most obvious linguistic observations here is that languages do differ in
their morphology: for any English noun in English dictionary there exists at
most four different word-forms of that word: e.g ‘cat,’ ‘cats,’ ‘cat’s,’ and
‘cats’’. If you are modelling a Finnish lexicon computationally, a word exists
in maybe 30 different forms if you look at
Wictionary\footnote{\url{https://en.wiktionary.org/wiki/talo\#Finnish}}, or
several thousand forms if you ask Finnish
linguists\footnote{\url{https://flammie.github.io/omorfi/genkau3.html}}. In any
case, modelling these as abstractions that will be useful in generic language
technology tools requires much care and understanding of linguistic diversity.
Initially one may engineer a system that works with one language only. If a tool
is useful, it will eventually be tried with other languages or with machine
translation in mind, and at that level any language-specific abstraction will
become more apparent.

The other big component of rule-based language modelling is syntax of the
language, what is often colloquially called just grammar, as in the context of
grammar checking and correction. The syntax in this context refers to the rules
governing how the words are arranged in a sentence. In many rule-based language
technology systems, syntax is used more cautiously and sparingly than
morphology, in that many language technology applications can be built with
limited modelling of syntax. The syntax is so to say one level of abstraction
higher than morphology and lexicon.

Going to further levels of abstraction in language modelling, language
technology also uses semantic modelling, where also the semantics of words are
further abstracted. In machine translation, the abstraction often goes even
further up to have a concept of Interlingua, a completely language independent
representation of the language that can be used to translate between languages
or in general to have fully language independent abstract representation of
meaning.

There are rule-based language technology approaches using all levels of
abstraction and some based on mainly morphology or morphology with little help
of syntactic processing.  As a comparison one can say that a lot of statistical
language technology tries to bypass all this abstraction by relying on just
statistics of data, for example distributions of wordforms.



\section{Language technology software}

Language technology gets used in a multitude of end-user software, spanning from
research tools to software aimed at big audiences. The research software that is
built from language technology is used in linguistic research, but also in other
near fields such as literature research. The uses of language technology in
end-user software for big audiences are many.  Language technology is needed in
a high number of everyday applications, such as: spellchecking and correction,
text input, search engines, grammar checkers, conversational agents (chatbots),
recommendation systems (users who liked this will also like that), automatic
subtitles, text-to-speech, machine translation, and summarisation. Practically
any application that needs to process language has some language technology
component in it.

It is also in the context of software that is based on language technology that
one can easily appreciate the differences between the rule-based language
technology and the datadriven approach. With rule-based approach, one is in
control of every phase of the language processing in a very detailed and
specific manner; if spell-checking application underlines a wrong word, we can
find out if it is missing in our dictionary component. If a data-driven
spell-checker underlines the wrong word, we only have an intuition that maybe it
is underrepresented in the texts we used to train the model. Similarly, when a
rule-based machine translation gives out wrong translation, the expert who made
it can often just look at the result to see where his rules, i.e., his mental
modelling of the language grammars went wrong, whereas the same problem with
data-driven methods will only point to lack of data with the only solution being
that we should feed the machine more data and it may or may not get better.

Because I work with minority and under-resourced languages, another point of
view I often like to bring out, is the risk and accountability of the language
technology software you make. As an example, if you bring a spell-checker that
underlines many correct words in red squiggly lines, to a community of language
users, who are not confident in the use of their language and the current norms,
it can be destructive for the whole culture. In rulebased approach we have a
control of the software in the extent that we can limit the risk and take
account of the errors of the software, whereas the statistical software can be
of bad quality and there is no accountability since it was the data that was
bad. On the other end for example for research software I have had experience
with many researchers that a language technology software that makes “mistakes”
of giving plausible but unlikely ideas of sentences is interesting because it is
part of the research.  Ultimately, the choice here, like I have alluded to
before, is not necessarily binary and one can incorporate various levels of
statistical information to rule-based language technology to make a hybrid
model, depending on the needs of the application. There is a popular saying in
our sub-field of rule-based language technology “don’t guess if you know,”
(e.g., Tapanainen and Voutilainen, 1994) which also encompasses well the most
common approach I have to hybridisation in this sense and I believe is a general
approach within the community.

\section{Summary}

Rule-based language technology is the form of natural language software
engineering based on using linguistic information as the main guideline of
language understanding and generation. Rule-based language technology’s main
benefits are that it gives full control of the language, this is opposed to the
other view on language technology that is based on statistical processing of the
language, where the outcome is largely decided by the kinds of data you feed to
it. Rule-based language technology is also not dependent on existing enormous
quantities of well-written texts (or recorded speech) of the language.
Furthermore, the approaches to language technology are not mutually exclusive,
and can be combined for hybrid language technology.

\bibliographystyle{unsrt}
\bibliography{rblt2023}

\end{document}
% vim: set spell:
