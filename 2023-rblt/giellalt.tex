\documentclass[free]{flammie}
%\documentclass[a4paper,draft]{article}
% (You'll need to use the class and packages provided by the conference for
%  submission anyways)

\newif\ifcameraready
\camerareadytrue

\usepackage{polyglossia}
\usepackage{fontspec}
\usepackage{xunicode}
\usepackage{xltxtra}
\usepackage{url}
\usepackage{hyperref}
\usepackage{expex}
\usepackage[obeyDraft]{todonotes}
%\usepackage{todonotes}


\begin{document}


\title{GiellaLT infrastructure: A multilingual infrastructures for rule-based
NLP\footnotepubrights{
    This work is licensed under a Creative Commons Attribution–NonCommercial-NoDerivatives
    4.0 International Licence.  Licence details:
    \url{http://creativecommons.org/licenses/by-nc-nd/4.0/}.
    Find the Publisher's version from:
    \url{https://dspace.ut.ee/handle/10062/89595}
}}


\ifcameraready
\author{
Sjur Nørstebø Moshagen, Flammie Pirinen, Lene Antonsen, Børre Gaup, Inga
Mikkelsen, Trond Trosterud, Linda Wiechetek, Katri Hiovain-Asikainen\\
Department of Language and Culture\\
NO-9019 UiT The Arctic University of Norway, Norway\\
\url{sjur.n.moshagen@uit.no}
}
\fi

\maketitle

\begin{abstract}
This article gives an overview of the GiellaLT infrastructure, the main parts of
    it, and how it has been and can be used to support a large number of
    indigenous and minority languages, from keyboards to speech technology and
    advanced proofing tools. A special focus is given to languages with few or
    non-existing digital resources, and it is shown that many tools useful to
    the daily digital life of language communities can be created with
    reasonable effort, even when you start from nothing. A time estimate is
    given to reach alpha, beta and final status for various tools, as a guide to
    interested language communities.

    Keywords: Infrastructure, spelling checkers, keyboards, rule-based, machine
    translation, finite state transducers, constraint grammar
\end{abstract}


\section{Introduction}

You know your language, you are technically quite confident, you want to support
your community by making language tools. But where do you start? How do you go
about it?  How do you make your tools work in Word or Google Docs? The GiellaLT
infrastructure is meant to be a possible answer to these and similar questions:
it is a tool chest to take you from initial steps all the way to the final
products delivered on computers or mobile phones in your language community. In
this respect, the GiellaLT infrastructure is world leading in its broad support
for many languages and tools, and in making them possible to develop
irrespective of the size of your language community.

As an example, Inari Sami with 450 speakers is one of the over 130 languages in
the GiellaLT\@. Within the GiellaLT infrastructure Inari Sami linguists have been
able to support the Inari Sami language community with high quality tools like
keyboards, a spellchecker, MT, a smart dictionary and are now developing an
Inari Sami grammar checker. All these tools are available for Inari Sami
speakers in the app stores of various operating systems, and through the
GiellaLT distribution system. In a revitalization perspective these tools are
crucial, but without an infrastructure supporting and reusing resources this
would have been impossible for the language community to solve on its own.

The GiellaLT infrastructure supports development of various tools in several operating
systems, as shown in Table 1 — the tools are further described later on. Infrastructure setup,
existing resources and tools are available on Github and documented on
\url{https://giellalt.github.io}.

\begin{table}
    \caption{
Tools built by the GiellaLT infrastructure, and their support for various systems.
= Windows, = macOS, = Linux, = iOS/iPadOS,
= Android, = ChromeOS, = browser,
= MS Word, = Google Docs, = LibreOffice, = REST/GraphQL API\label{tbl:}}
    \begin{tabular}{lll}
    Tools & Supported systems & Further info \\
        \hline
Keyboards \\
Spelling checkers \\
Hyphenators \\
Text tokenisation and analysis \\
        Grammar checking & & Includes speller, LO only Linux \\+
Machine translation\\
Text-to-speech & & planned, not ready\\
Dictionaries \\
Language learning \\
Automatic installation and updating & & \\
    \end{tabular}
\end{table}

When adding a new language to the GiellaLT infrastructure, the system builds a toy
model and sets it up with all language independent build files needed in order to make the
tools shown in Table 1. The main advantage of GiellaLT is thus that the quite substantial
work that has gone into writing the files needed for compiling, testing and integrating the
language models into the various tools. The language models differ from tool to tool in
systematic ways, in some cases a descriptive model (accepting de facto usage) is needed,
in other cases a normative one (accepting only forms according to the norm) is called for.
Part of the language model, e.g. the treatment of Arabic numerals and non-linguistic symbols is also re-used from language to language. A large part of any practical language technology project is the integration of the tools in user programs in different operative systems.
For the GiellaLT infrastructure this has been done for a wide range of cases, as shown in
Table 1. This makes it possible to make language technology tools for new languages without spending several man-years on building a general infrastructure.

GiellaLT is developed by two research groups at UiT. At present, it includes support
and tools for 130 different languages, most of them extremely low resource like Inari Sámi
and Plains Cree. Such languages have few or no text resources, where the few that exist
typically are too noisy to be able to represent a norm. These and similar languages must
thus rely on mostly rule-based methods like the ones described in this article. Another point
to consider is that the smaller the language community, the larger the need of tools to
support using the language in writing. Having such an infrastructure is thus of crucial importance for the future of many of the world’s languages.

Not only does the GiellaLT infrastructure offer a pipeline for building tools, it also supports the process in the other end: when the language work is approaching production quality, there exists a delivery and update ecosystem, making it easy to distribute the tools to
the user community. The infrastructure also contains tools to develop the linguistic data
and verify its integrity and quality automatically.

What is needed to utilise GiellaLT is a linguistic description of the language and a
skilled linguist, a native speaker to fill in the gaps in the description, and a programmer to
help with configuring the infrastructure for a new language and support the linguist in the
daily work. A language project will also benefit from activists who need the tools: they will
tell what they need, test the tools, and thus ensure the linguistic quality of the output. In
practice every new language added to the infrastructure for which practical tools are developed will also require adaptions and additions in GiellaLT, thereby contribute to the
strength of the GiellaLT infrastructure.

Throughout this chapter of the book, we try to give an estimate of the expected amount
of work needed to reach a certain maturity level for tools and resources. The information
is given in side-bar info boxes, as seen below. The symbols and abbreviations used are as
follows:

⏱
mm: ¼ ߙ
ߚ :1 mm
1.0: 3-4 m

\begin{itemize}
    \item ߙ – alpha version, first useful but clearly unfinished
        version\footnote{A more precise definition of what these labels mean within the GiellaLT infrastructure is available
at https://giellalt.github.io/MaturityClassification.html}
    \item ߚ – beta version, getting ready, but some polish still needed
    \item 1.0 – final version, released to the language community
    \item mm – man month, one month’s work
\end{itemize}
The estimates given are based on our experience, but a number of external factors will
influence the actual time a given project takes. We still believe that having an indication of
expected work estimates can be useful when planning language technology projects. It
should be emphasized that all estimates assumes that the GiellaLT infrastructure and support system is being used.

\section{An overview of the infrastructure}

The GiellaLT infrastructure contains all the necessary pieces to go from linguistic source
code to install-ready products. Most of the steps are automated as part of a continuous
integration \& continuous delivery (CI/CD) system, and language-independent parts
are included from or taken from various independent
repositories\footnote{\url{https://github.com/divvun}}.


The infrastructure is in practice a way of organising all the linguistic data
and scripts in a manner that is easily maintainable by humans who work on
various aspects of the text and then can be systematically built into
ready-to-use software products by automated tools. The way we approach this in
practice changes from time to time as the software engineering ecosystem
develops; however, the organisation of the system aims to keep the linguistic
data more constant, as the linguistics do not change at the pace software
engineering tools. The crux of the infrastructure in the large scale is though
having the right files of linguistic data in the right place. That is,
standardisation of the folder and filenames and standardisation of the analysis
tags are main features of the infrastructure.  The data is stored in the version
control system git\footnote{}, hosted by GitHub\footnote{}. In GitHub the data
is organised in repositories. Each repository is a unit of mostly self-standing
tools and source code.

There are a few different kinds of linguistic repositories in our
infrastructure, mainly ones for keyboard development (keyboard repositories) and
ones for development of morphological dictionaries, grammars, and all other
linguistic data (language repositories); these repositories are language
specific, i.e., there’s one repository for each language\footnote{}. The
different repository types and the content they contain, along with its
structure, is described in the following sections.

\subsection{Keyboard repositories}

Keyboards enable us to enter text into digital devices. Without keyboards, no
text. This is a very real obstacle for the majority of the languages of the
world, the ones with no keyboards. It thus needs to be easy to create and
maintain keyboard definitions and make them available to users.  In the GiellaLT
infrastructure, keyboard definitions have their own repositories (using the
repository name pattern keyboard-*) that contain the linguistic data defining
the layout of the keyboards of the languages, and all metadata to build the
final installation packages. The repositories are organised as a bundle, which
is consumed by the tool kbdgen\footnote{}.

The bundle structure is as follows:
\begin{verbatim}
sma.kbdgen
├── layouts # actual layout definitions
│ ├── sma-NO.yaml # desktop layout for Norway
│ ├── sma-SE.yaml # ditto for Sweden – they are different
│ └── sma.yaml # mobile keyboard, identical for SE/NO
├── project.yaml # top-level metadata
├── resources # platform specific resources
│ └── mac
│ ├── icon.sma-NO.png
│ └── icon.sma-SE.png -> icon.sma-NO.png
└── targets # metadata for various platforms
 ├── android.yaml
 ├── ios.yaml
 ├── mac.yaml
 └── win.yaml
\end{verbatim}

The layout definitions are described in the next chapter.
Based on the layouts and metadata, kbdgen builds installers, packages, or suitable target files for the following systems: macOS, Windows, Linux (X11, IBus m17n), Android,
iOS/iPadOS, and ChromeOS. For macOS and Windows, installers are readily available via
Divvun’s package manager Páhkat, further described towards the end of this chapter. For
iOS and Android, layouts are included in one or both of two keyboard apps: Divvun Keyboards, and Divvun Dev Keyboards. Divvun Dev Keyboards functions as a testing and
development ground, whereas production ready keyboards go into the Divvun Keyboards
app. All of this is done automatically or semi-automatically using CI and CD servers8
.

The kbdgen tool also supports generating SVG files for debugging and documentation,
as well as exporting the layouts as xml files suitable for upload to CLDR. And finally, it
can also generate finite state error models for keyboard-based typing errors, giving suitable
penalties to neighbouring letters based on the layout.

The Windows installer includes a tool to register unknown languages, so that even languages never seen on a Windows computer will be properly registered, and thus making
Windows ready to support proofing tools and other language processing tools for those
languages.

\subsection{Language repositories}

The language repositories contain lexical data, grammar rules, morphophonology, phonetics, etc., anything linguistic and specific to the language or even specific to the tools is built
from the language data.

The language repositories, using a repository name pattern of lang-*, contain the
whole dictionaries of the languages, laid out in a format that can be compiled into the NLP
tools we provide. To achieve this, the lexical data has to be rich enough to achieve inflecting
dictionaries, that is, the words have to be added some information of their inflectional patterns for example. In practice, there is an unlimited amount of information that can be recorded per dictionary word that can be interesting. So in practice, this central part of the
language repository becomes like a lexical database of linguistic data. On top of that we
need different kinds of rules governing morphographemics, phonology, syntax, semantics
and so forth.

In practice, writing a finite state (see 2.2.1) standardised language model in src/fst/ will provide
the user with the basis for all the NLP tools we can
build. To draw a parallel on how this works, if one is
familiar with java programming for example, this is
akin putting your maven-based project into
src/java/ or rust-based configurations in
Cargo.toml etc. would be a software engineering
interpretation of what an infrastructure is.

We also have some standards as to how to tag specific linguistic phenomena, as well as other lexical information. The linguistic software we write is in part
based on that similar phenomena are marked in same
manner in all languages. This ensures that components that are language-independent work the best. If
specific languages deviate from some standards, it practically can mean that for those languages specific exceptions need to be written for every application. This is especially clear
when working with such grammar-based machine translation, even a small mismatch in
marking the same structures makes the translation fail whereas systematic use of standard
annotations makes everything work automatically.
8 CI = continuous integration, CD = continuous delivery (Shahin et al, 2017)
The language repositories follow a specific template, structure, and practice to make
building everything easier:
\begin{verbatim}
lang-sme
├── docs
├── src
│ ├── cg3
│ ├── filters
│ ├── fst
│ …
└── tools
 ├── spell-checkers
 ├── grammarcheckers
 ├── mt
 …
\end{verbatim}

\subsubsection{Morphological analysis}

The underlying format for the linguistic models in the GiellaLT infrastructure is based on
finite state morphology (FST), combining the lexc and twolc programming languages
(Koskenniemi 1983, Beesley and Karttunen 2003). This is no accident: These programming
languages as well as the Constraint Grammar formalism presented in the next subsection
were all developed for Finnish, a language with a complex grammar and many skilled computational linguists. The contribution of the persons behind GiellaLT has been to port these
compilers into open formats and set them up in an integrated infrastructure. Most GiellaLT
languages are of no interest to commercial language technology companies, and the infrastructure thus contains pipelines for all aspects of language technology.

The morphology is written as sets of entries,
where each entry contains two parts (divided by
space and terminated by semicolon). The first part
contains a pair of two symbol strings (to the left and
the right of the colon, called the upper and lower
level). The part after the space is a pointer to a lexicon containing a set of entries, so that the content of
all entries in this lexicon is concatenated to the content of all entries
pointing to it. The symbol \verb|#| has
special status, denoting the end of the string. In the
text box to the right, the entry kana:kana is directed
to the lexicon n1. This lexicon contains 3 entries,
each pointing to the lexicon clitics. These 3 lexica
will then give rise to 3*3*4=36 distinct forms. The
upper level of the entries contains lemmas and morphosyntactic tags, whereas the lower level contains
stems and affixes. It may also contain archiphonemes, such as \verb|^A| representing front ä and back a, and triggers for morphophonological
processes. In this example \verb|^WG| triggers the weak grade of the consonant gradation process,
a process which in this example assimilates nt into nn in certain grammatical contexts (here:
genitive and inessive singular).

The morphographemics is taken care of in a separate finite state transducer, written in a
separate language, twolc, in a separate file (src/fst/phonology.twolc):

\begin{verbatim}
Alphabet
a b c ... å ä ö %^U:y %^A:ä %^O:ö ;
Sets
Vow = a e i o u y ä ö å %^U %^A %^O ;
Rules
“vowel harmony”
%^A <=> a [a|o|u] \[ä|ö|y]* %> (\[ä|ö|y]* %>) \[ä|ö|y]* ;
"Consonant Gradation nt:nn"
t:n <=> n _ Vow: %^WG: ;
\end{verbatim}

Twolc defines the alphabet and sets of the model. Also, this transducer has an upper and
lower level, so that the upper level of this transducer is identical to the lower level of the
morphological (lexc) transducer. The rule format \verb|A:B <=> L _ R ;| denotes that there is
a relation between upper level A and lower level B when occurring between the contexts L
and R. The result is that non-concatenative processes such as vowel harmony and consonant
gradation are done in a morphographemic transducer separate from the morphological one.

⏱
ߙ :1-2 mm
ߚ :6 mm
1.0: 12 mm

An excerpt from the lexc file
for Kven nouns (the file
src/fst/stems/nouns.lexc
in the infrastructure):

\begin{verbatim}
LEXICON Nouns
kana:kana n1 "hen" ;
kynä:kynä n1 " pen" ;
hinta:hinta n1 "price" ;
LEXICON n1
+N+Sg+Nom: clitics ;
+N+Sg+Gen:^WG%>n clitics ;
+N+Sg+Ine:^WG%>ss^A clitics ;
...
LEXICON clitics
# ;
+Qst:%>k^O # ;
+Foc/han:%>h^An # ;
+Foc/ken:%>kin #
\end{verbatim}

These two transducers are then compiled into one transducer, containing string pairs like
for example hinta+N+Sg+Ine:hinnassa, where the intermediate representation
\verb|hinta^WG>ss^A| is not visible in the resulting transducer. The result is a model containing
the pairs of all and only the grammatical wordforms in the language and their corresponding lemma and grammatical analyses.

The actual amount of work needed to get to a reasonable quality will vary depending on
the complexity of the language, available electronic resources, existing documentation in
the form of grammars and dictionaries, and experience, but based on previous projects a
reasonable first version with decent coverage can be made in about six months. For good
coverage, one should estimate at least a year of work.

\subsubsection{Morphosyntactic disambiguation}

Most wordforms are grammatically ambiguous, such as the English verb and noun walks.
The correct analysis is in most cases clear from the context. The form walks may e.g. occur
after determiners (and be a noun) or after personal pronouns (and be a verb). More complex
grammars typically contain more homonymy, such as the South Saami leah ‘they/you are’
- has four different analyses with the same lemma, three of them finite verb analyses and
one of them a non-finite analysis, the con-negative verb reading:

\begin{verbatim}
"<leah>"
"lea" V IV Ind Prs Pl3 <W:0.0>
"lea" V IV ConNeg <W:0.0>
"lea" V IV Imprt Sg2 <W:0.0>
"lea" V IV Ind Prs Sg2 <W:0.0>
\end{verbatim}

Within the GiellaLT infrastructure, disambiguation and further analysis of text is made
with constraint grammar (Karlsson 1990) compiled with the free open-source implementation VISLCG-3 (Bick 2015). The morphological output of the transducer feeds into a chain
of CG-modules that do disambiguation, parsing, and dependency analysis. The analysis
may be sent to applied CG modules such as e.g., grammar checking.
Syntactic rules of the parser disambiguate the morphological ambiguity and can add
syntactic function tags. In the following sentence the context shows that leah should be
con-negative based on the negation verb ij to the left.

\begin{verbatim}
Im leah.
NEG.1sg.prs be.con-negative
"I am not./No."
"<Im>"
"ij" V IV Neg Ind Prs Sg1 <W:0.0> @+FAUXV
:
"<leah>"
"lea" V IV ConNeg <W:0.0> @-FMAINV
 "lea" V IV ConNeg <W:0.0> @-FMAINV
; "lea" V IV Imprt Sg2 <W:0.0>
; "lea" V IV Ind Prs Pl3 <W:0.0>
; "lea" V IV Ind Prs Sg2 <W:0.0>
IFF ConNeg (*-1 Neg BARRIER CC OR COMMA OR ConNeg);
\end{verbatim}

The rule that is responsible for removing all the other readings of leah and only picking the
con-negative (ConNeg) reading is an IFF rule that refers to the lemma lea “be” in its ConNeg form and a negation verb to the left, without any conjunction (CC) or comma or
another ConNeg form in between. The rule here is simplified, there are more constraints
for special cases. The IFF operator either selects the ConNeg reading if the constraints are
true, or removes it if the constraints are not true.

⏱
ߙ :3 mm
ߚ :8 mm
1.0: 18 m

As the analysis is moved further and further away from the language-specific morphology, the rules become increasingly language independent. Antonsen et al (2010) have
shown that whereas disambiguation should be done by language-specific grammars,
closely related languages may share the same function grammars (assigning roles like subject, object). The dependency grammar was shown to be largely language independent.

\subsection{Other repository types}

The GiellaLT infrastructure contains several other repository types, although not as structured as the keyboard and language repositories. Corpus repositories (repository name pattern corpus-*) contain corpus texts, mostly in original format, with metadata and conversion instructions in an accompanying xsl file. This is done to make it easy to rerun
conversions as needed. The corpus tools and processing are further described later on.

There are a few repositories with shared linguistic data (repository name pattern
shared-*). Typically, they contain proper names shared among many language repositories, definition of punctuation symbols, numbers, etc. The shared data is included in the
regular language repositories by reference.

Both keyboard and language repositories are structurally set up and maintained using a
templating system, for which we have two template repositories (repository name pattern
template-*). Updates to the build system, basic directory structure, and other shared
properties are propagated from the templates to all repositories using the tool gut,
9
and
allows all supported repositories to grow in tandem when new features and technologies
are introduced. This ensures relatively low-cost scaling in terms of features and abilities
for each language, so that a new feature or tool with general usability easily can be introduced to all languages.

There are separate repositories for speech technology projects (repository name pattern
speech-*). As speech technology is a quite recent addition to the GiellaLT, these repositories are not standardised in their structure yet, and have no templates to support setup
and updates. As we gain more experience in this area, we expect to develop our support for
speech technologies.

\section{Linguistic tools and software}

The main point of the GiellaLT infrastructure is to provide tools for language communities.
In this chapter we present the tools that are most prominent and have been most used and
useful for our users: keyboards, grammar and spell checking and correction, dictionaries,
linguistic analysis, and machine translation between or from the minority languages. Also,
language learning and speech synthesis is covered, and finally we show how to distribute
the tools to the language communities and to ensure that the tools stay up to date.
We also show what constitutes the starting point for building a software tool for your
language as well as the prerequisites needed for getting that far. Finally, we show some
ideas that are under construction or showing promising results, ideas for which we cannot
yet provide an exact recipe on how to build working systems.

⏱
ߙ1/ :₂₀ mm
mm: ¼ ߚ
1.0: 1 mm
⏱
ߙ :1-3 mm
ߚ :6 mm
1.0: 12 mm

\subsection{Keyboards}

To be able to type and write a language, you need a keyboard. Using the tool kbdgen, one
can easily specify a keyboard layout in a YAML file, mimicking the actual layout of the
keyboard. The listing below shows the definition of the Android mobile keyboard layout
for Lule Sámi. The kbdgen tool takes this definition and a bit of metadata, combines it
with code for an Android keyboard app, compiles everything, signs the built artefact and
uploads it to the Google Play Store, ready for testing.

\begin{verbatim}
modes:
 android:
 default: |
 á w e r t y u i o p å
 a s d f g h j k l ø æ
 z x c v b n m ŋ
⇒ kbdgen ⇒
\end{verbatim}

Additional metadata is for example language name in the native language, names for some
of the keys, icons for easy recognition of the active keyboard, and references to speller
files, if available.

The YAML files can contain definitions for dead key sequences for desktop keyboards,
as well as long-press popup definitions for touch-screen keyboards. The newest version of
the kbdgen tool supports various physical layouts for desktop keyboards, to allow for nonISO keyboard definitions. Further details for the layout specification can be found in the
kbdgen documentation\footnote{}.

The overall goal is to make it as easy as possible for linguists to write a keyboard definition and get it into the hands of the language community. A first draft can be created in
less than a day, and a good keyboard layout for most operating systems will take about a
week to develop, with a couple of weeks more for testing, feedback, and adjustments.

The keyboard infra in GiellaLT works well for any alphabetic and syllabary-based writing system, essentially everything except iconographic and similar systems.

Because of technical limitations by Apple and Google, it is not possible to create keyboard definitions for external, physical keyboards for Android tablets and iPads. Our onscreen keyboards work as they should even when a physical keyboard is attached to such
tablets.

\subsection{Spell-checking and correction}

In GiellaLT, spell-checking and correction are built on a language model (in the form of a
finite-state transducer) for the language in question. A spell-checker is a mechanism that
recognises wordforms that do not belong to a dictionary of all acceptable wordforms in the
language and tries to suggest the most likely wordforms for the unknown ones. The
GiellaLT spellcheckers differ from this approach in not containing a list of wordforms, but
rather a list of stems with a pointer to concatenative morphology (see also the section 2.2.2
on language model repositories for specifics on what this looks like and is built) as well as
a separate morphophonological component. The resulting language model should then recognise all and only the correct wordforms of the language in question. Since the language
model is dynamic it is also able to recognise wordforms resulting from productive but unlexicalized morphological processes, such as dynamic compounding or derivation. In languages like German or Finnish, one can freely combine for example words like kissa ‘cat’,
bussi ‘bus’ and kauppias ‘salesman’ into kissabussikauppias that will be acceptable for
language users and thus should be supported by the spell-checker. The main challenge
when building a spell-checker and corrector is to have a good coverage in the lemma list,
and in our experience, several months of work in dictionary building or around 10,000 wellselected words will suffice for a good entry-level spell-checker. Since the FST also will be
used for analysing texts (chapter 2.2), it will be necessary to compile a normative version
of the FST, which excludes non-normative forms. They are excluded by means of tags.

The mechanism for correcting wrongly spelled words into correct ones is called error
correction. The most basic error correction model is based upon the so-called Levenshtein
distance (Levenshtein 1965): For an unknown wordform, suggest known wordforms resulting from one of the following operations: delete character, add character, exchange
character with another one or swap the order of two characters. The benefit of this baseline
model is language independent, so we can use it for all languages as a starting point.

Spell-checking and correction can be improved on a per language basis drawing from
the knowledge of the language and its users. One of the most basic ways of improving the
quality of spelling corrections, is improving the error modelling. f we know what kind of
errors users make and why, we can ensure that the relevant corrections are suggested more
commonly. Segment length, especially consonant length, may be hard to perceive and
hence likely to be misspelled. Doubling and omitting identical letters are thus built into
most error models. The same are parallel diphthongs such as North Saami uo/oa, ie/ea,
where L2 writers may be likely to pick the wrong one. By making the diphthong pairs part
of the error model the errors can easily be corrected. The position in the word may also be
relevant: For a language where final stop devoicing is not shown in the written language,
exchanging b, d, g with p, t, k may be given a penalty number lower than the default value.

When the dictionary reaches sufficiently high coverage the problems caused by suggesting relatively rare words become more apparent, one way of dealing with this problem
is codifying the rare words on the lexical level. If there are corpora of correctly written
texts available, it is also possible to use statistical approaches to make sure that very rare
words are not suggested as corrections unless we are sure they are the most likely ones.

It is also possible to list common misspellings of whole words. The South Saami error
model contains the pair uvre:eevre (for eevre "just, precisely"), where word forms like
muvre and duvre otherwise would have had a shorter Levenshtein distance.

Seen from a minority language perspective, the main point is that the GiellaLT infrastructure offers a ready-made way of building not only language models but also speller
error models and a possibility to integrate them into a wide range of word processors. And
having a mechanism for automatically separating normative from descriptive forms means
that the same source code and language model can be used and reused in many different
contexts.

⏱
mm: ¼ ߙ
ߚ :1 mm
1.0: 3-4 mm

⏱
mm: ¼ ߙ
ߚ :1 mm
1.0: 3 mm

\subsection{Automatic hyphenation}

The rules for proper hyphenation vary from language to language, but usually it is based
on the syllable structure of the words. Depending on the language, morphology may also
play a role, especially word boundaries in languages with compounding.
The GiellaLT infrastructure supports hyphenation using several mechanisms. The core
component is an FST rule component defining the syllable structure. Exceptions can be
specified in the lexicon, and both lexicalised and dynamic compounds will be used to override the rule-based hyphenation. The result is high-quality hyphenation, but it requires good
coverage by the lexicon. The lexical component is based on the morphological analyser
described in earlier sections.
Given that the morphological analyser is already done, adding hyphenation rules does
not take a lot of work. The most time-consuming work is testing and ensuring that the result
is as it should be. Getting or building hyphenated test data can be time consuming.

\subsection{Dictionaries}

The GiellaLT infrastructure includes a setup for combining dictionaries and language models. Most GiellaLT languages possess a rich morphology, with tens or even hundreds of
inflected forms for each lemma, often including complex stem-alternation processes, prefixing or dynamic compounding. Looking up unknown words may be a tedious endeavour,
since few of the instances of a lemma in running text may be the lemma form itself.
The GiellaLT dictionaries combine the dictionary with an FST-based lookup model that
finds the lemma form, sends it to the dictionary and presents the translation to the user.
This may be done via a click-in-text-function, as in Figure 1.
The FST may then also be used to generate paradigms of different sizes to the user, as well
as facilitate example extraction from corpora. The tags in the analysis can also be used as
triggers for additional information about the word, e.g. derivation, which can be linked to
information in an online-grammar (see Johnson et al 2013 for a presentation).
Dictionary source files are written in a simple XML format. If one has access to a bilingual machine-readable dictionary and the lemmas in the dictionary have the same form as
the lemmas in the FST, the dictionary may easily be turned into a morphology-enriched
dictionary in the GiellaLT infrastructure. Less structured dictionaries will require far more
Figure 1: The dictionary set-up. The word, which is clicked for in this image, is both inflected and a compound. The analyser finds the base form for both parts, gives them to the
dictionary, and the translation is then sent to the use
work. Homonymous lemmas with different inflection should be distinguished by getting
different tags, both in lexc and in the XML schema. In this way we can ensure that the
words are presented with the correct inflectional paradigm to the dictionary user.

⏱
ߙ :3 mm
ߚ :8 mm
1.0: 12 mm

\subsection{Grammar-checking and correction}

Everybody *make errors – both grammatical errors and typos. Expectations are higher as
to what grammar checkers should do compared to a spellchecker, even if we do not think
about it consciously. Whereas to find that *teh is a typo for the we only need to check the
word itself, to find that *their is a typo for there we need to actually look at the whole
sentence. So even for typos we need a more powerful tool that understands grammar.
These rather trivial errors make up a big part in English grammar checking (e.g., the
Grammarly program11). Most minority languages, especially the circumpolar ones in the
GiellaLT infrastructure, are morphologically much more complex. They are rich in inflections and derivations, with complex wordforms that bear a lot of potential for errors. When
wordforms are pronounced close to each other or endings are omitted in speech they are
also frequently misspelled in writing. Typical errors are those that concern agreement between subject and verb or determiner and noun (cf. the North Sámi example), as well as
case marking errors in different parts of the sentence.

\begin{verbatim}
Mii sámit maid *áigot gullot. > áigut
We Sámi.Nom.Pl want.3Pl listen > want.1Pl
We Sámi also (they) want to listen > we want to
\end{verbatim}

When writing a rule-based grammar checker we first identify the morphological and
syntactic structure of a sentence similar to parsing. Each word is associated with a lemma
and a number of morphological tags. If the word is homonymous either in its form or already based on different lexemes (like address – 1. noun 2. verb infinitive 3. finite verb) it
is associated with more than one possible analysis. Disambiguation is not the same as in
parsing as the goal of it is another one. Instead of aiming for one remaining analysis, we
only want to remove as much ambiguity as is necessary to find a potential error. But since
we expect the sentence to contain errors, therefore being a bit more unreliable, we are a bit
more relaxed on disambiguation. At the same time, we do need reliable information as to
what the context of our error is. In case of address, we would need to identify it as a noun
before looking for an agreement error with a subsequent finite verb.

The analysis must be robust enough to recognise the grammatical pattern in question
even when the grammar is wrong. Constraint grammar differs from other rule-based grammar formalisms in being bottom-up, taking the words and their morphological analysis as
a starting point, removing irrelevant analyses, and adding grammatical functions. For robust rules we make use of all the potential of rule-based methods. We have access to semantic information (e.g. human, time, vehicle), valency information (case and semantic
role of the arguments of a verb), pronunciation related traits which makes the form bound
to certain misspellings. We can also add dependencies and map semantic roles to their respective verbs in order to find valency-based case or pre-/postposition errors.

GiellaLT’s first grammar checker was made for North Sámi and work started in 2011.
(Wiechetek 2012; Wiechetek 2017). Since 2019 grammar checkers are supported by GiellaLT for any language. Its setup includes various modules to ensure availability of the required information to find a grammatical error and generate suggestions and user feedback.

As Microsoft Word unfortunately does not open for integrating third-party solutions for
low-resource languages, we are forced to using its web-based plugin interface instead of
the usual blue line below the text. The grammar checker interface, detected errors and suggested corrections are presented in a sidebar to the right of the text, as seen in the screen
shot of a version of the North Sámi grammar checker in Figure 2.

When making a grammar checker from scratch without any previous study of which
grammatical errors are frequent for either L1 or L2 users, building a grammar checker becomes a study of errors at the same time as the tool is made. A well-functioning grammar
checker requires hand-written rules that are validated by a set of example sentences from
the corpus (newspaper texts, fiction, etc.) so exceptions to a certain grammatical construction are well covered. This set of examples should be included in a daily testing routine to
see if rules break when they are modified and to follow development of precision and recall.
One should start with a set of at least 50 examples including both positive and negative
examples of a certain error so that both precision and recall can be tested.

Error detection and correction rules that add error tags to a specific token and exchange
incorrect lemmata or morphological tag combinations with correct ones. The most complex
of these rules include very specific context conditions and numerous negative conditions
for exceptions of the rules. The rule in the following example detects an agreement error in
a word form that is expected to be third person plural and fails to be so.

The expectation is based on a preceding pronominal subject in third person plural or a
nominal subject in nominative plural. *-1 specifies its position to the left. The BARRIER
operator restricts the distance to the subjects by specifying word (forms) that may or may
not appear between the subject and its verb. In this case only adverbs or particles may
appear between them. The target of error detection is a verb in present or past tense.
The exceptions are specified in separate condition specifications afterwards. Some regard the target and exclude common homonyms like illative case forms or idiosyncratic
adverbs, common spelling mistakes of nouns (that then get a verbal analysis), homonymous
non-finite verbforms. The rule also excludes 1st person plural present tense forms except
for those ending in –at (here specified by a regex), and infinitives that are preceded by

Figure 2 North Sámi Divvun grammar checker in MS Word in the right sidebar
instead of the blue underline
lemmata with infinitive valency tags. The exceptions specify also coordinated nounphrases, dual forms with coordinated human subjects or coordinations that involve first or
second person pronouns, to name some of them.

⏱
ߙ :1 mm
ߚ :3 mm
1.0: 12 mm

\begin{verbatim}
ADD (&syn-number_congruence-subj-verb) TARGET (V Ind Prs) - ConNeg OR (V Ind Prt)
IF
 (*-1 (Pron Sem/Hum Pers Pl3 Nom) OR (N Pl Nom) BARRIER NOT-ADV-PCLE)
 (NEGATE 0 (Sg Ill))
 (NEGATE 0 N + Err/Orth-any)
 (NEGATE 0 Prs Pl1) - ("<.*at>r"))
 (NEGATE 0 Inf LINK -1 <TH-Inf> OR <*Inf>)
 (NEGATE 0 Du3 LINK *1 Sem/Human BARRIER NOT-ADV-PCLE LINK 1 CC LINK 1 Sem/Human)
 (NEGATE 0 Pl1 LINK -1 Sem/Human + (Nom Pl) LINK -1 CC LINK -1 ("mun" Nom));
ADD rules cooccur with COPY rules which replace the incorrect tags (in this case any person
number that is not Pl3) with the correct one, i.e., Pl3 with the error tag added by the accompanying ADD rule.
COPY (Pl3 &SUGGEST) EXCEPT Sg1 OR Sg2 OR Sg3 OR Du1 OR Du2 OR Du3 OR Pl1 OR Pl2
TARGET (V Ind Prs &syn-number_congruence-subj-verb) ;
\end{verbatim}

\subsection{Machine translation}

Machine translation (MT) in GiellaLT is handled by making our morphological analysers,
and syntactic parsers available to the Apertium MT infrastructure (see Khanna et al 2021).
This is basically performed by a series of conversions in order to convert the output of the
GiellaLT language models to adhere to Apertium conventions as well as by adding some
new components. The basic building block is the morphological and syntactic analyser for
the source language (see section 2.2.2) and a bilingual dictionary. This analyser must have
a good coverage, which also includes non-normative forms. If the source language and the
target language are not closely related, the syntactic tags in the analysis of the source language are very useful.

The output in the target language is generated by a morphological analyser which at
least covers the lemmas in the dictionary. In addition, one needs a grammatical description
and phrases where the grammars of source and target languages do not meet. This grammar
handles all mismatches between source and target language, whatever they may be: dropping of pronouns, introduction of articles, gendering of pronouns, idioms and MWE’s, etc.

We do the bilingual lexicography like this:

\begin{verbatim}
<e><p><l>bivdi<s n="n"/></l><r>jeger<s n="n"/><s n="m"/></r></p></e>
<e><p><l>bivdi<s n="n"/></l><r>fisker<s n="n"/><s n="m"/></r></p></e>
<e><p><l>bivdin<s n="n"></l><r>jakt<s n="n"/><s n="f"/></r></p></e>
\end{verbatim}

Inside the e (entry) and p (pair) node there are two parts, l(left) and r (right), with the
two languages as node content. Each word is followed by a set of s nodes, containing
grammatical information (here: POS (n = noun) and gender (m, f, nt = masculine, feminine and neuter). This is a basic XML format used with Apertium. The logic is simple, and
it benefits from the tags being systematic. If the dictionary contains two or more word pairs
with identic lemma on the left side, one must consider which word pair to choose, or one
can make lexical selection rules, based on context in the sentence. These rules make a lexical selection module, which can be made as XML, based on position and lemma and tags
in the context, or can be made with CG rules, see 3.2.2.
The system supports multi-words in both directions and idioms. The primary goal is to
get a good coverage of singleton words. Syntactic transfer is done as follows

⏱
ߙ :12 mm
ߚ :15 mm
1.0: 18 mm

\begin{verbatim}
S -> VP NP { 1 _
 *(maybe_adp)[case=2.case]
 *(maybe_art)[number=2.number,
 case=2.case,gender=2.gender
 ,def=ind] 2 } ;
V -> %vblex {1[person =
(if (1.tense = imp) "" else 1.person),
 number = (if (1.number = du)
 pl else 1.number)] } ;
\end{verbatim}

In this simplified example from a real-world grammar for syntactic transfer from NorthSaami to a Germanic language we show that the machine translation syntax is quite like
the notation conventions from e.g., the Standard Theory (Chomsky 1965). Rules may thus
operate on either word or phrase level. Here we handle distribution of case, number and
gender for the generated adpositions and articles, which are based on the case and the position in the sentence, and translating the singular-dual-plural system of North Sámi into
the Germanic singular-plural system (Pirinen and Wiechetek 2022). After chunking words
together in the first two modules, the following modules change the word order, to the
extent it is necessary.

In our experience the systems start to be usable for understanding the idea of the text in
translation when containing 5,000 well-selected translation pairs, if it is possible also to
transfer compounding and/or derivations from source language to target language. This
makes a few months of work. If the language pair requires large re-ordering of the grammar, it will demand more work.

Although there is a request for MT programs from the majority language to the minority
language, we have chosen to make systems the other way, from the minority language to
the majority language. These systems give the freedom to write in the minority language.
A possible bad quality of the output in a majority language has no bad impact on the language itself. We have also built MT systems for translating between minority languages,
which are related to each other, both to avoid using the majority language as a lingua franca,
and to make it possible to use the bigger minority language as a pivot language for text
production (see Antonsen et al 2017).

\subsection{Text-to-speech}

Developing TTS for an indigenous language with few resources available can be challenging. Resources such as grammars, language learning books or phonetic descriptions are
important in designing the project, building, and checking the corpora and evaluating the
TTS output phonetically. With using only a phonetic/phonological description and utilizing
the knowledge of native speakers of a language, a simple and ”old-fashioned” but still usable TTS application, such as the Espeak formant synthesis (Kastrati et al., 2014; Pronk et
al., 2013) can be built from scratch. The downside of this is that while it might be a working
speech synthesizer, the sound quality is very machine-like and the users’ expectations for
the quality of a TTS system are very high due to the realistic and very human-like examples
from well-resourced languages such as English.

We are now working on a modern, open-source TTS system that could be openly available for anyone who wants to develop speech technology for any low-resource language.
The system will make all language-independent parts integrated into the larger GiellaLT
infrastructure, ensuring that maintenance and updates are done regularly. When finished, it
will also ensure that all voices will be available on all supported platforms, and that new
platforms will be available to all existing voices. It will utilise the rule-based text
processing technologies already present in the GiellaLT infrastructure, using machine
learning technologies only for the soundwave synthesis step.

The development of a TTS system requires multidisciplinary input from fields like natural language processing (NLP), phonetics and phonology, machine learning (ML) and
digital signal processing (DSP). Tasks connected to NLP are important in developing the
text front-end for the TTS – these are, for example, automatically converting numbers and
abbreviations to full words in a correct way. Phonetics and phonology are essential in corpus design, writing text-to-IPA rules and evaluating the TTS output. Importantly, by using
phonetic annotations of the texts, it is possible to address phenomena that are not visible in
the orthographic texts. The importance of ML is growing in the field of speech technology
as neural networks are used to model the acoustic features of human speech, allowing for
realistic and natural-sounding TTS. Procedures related to DSP are important in (pre)processing the audio data: these include filtering, resampling, and normalizing the corpus for
suitable audio quality.

The first step in developing a TTS system is designing and building a text corpus, considering the requirements of the speech technology. In the next section, we describe our
working process in developing a TTS voice for the Lule Sámi language.

\subsubsection{Building and processing a speech corpus}

Building a speech corpus starts by collecting a suitable multi-domain text corpus which
corresponds to at least 10 hours of recorded read speech, that has been shown to be enough
to achieve an end-user suitable TTS system for North Sámi (Makashova, 2021). Before the
speech recordings could be done, a suitable text corpus had to be collected. In the Lule
Sámi project we reused part of a Lule Sámi gold corpus developed in 2013 within the GiellaLT community and collected additional texts of various text styles we knew to be well
written. The resulting Lule Sámi text corpus for TTS consists of text styles such as news,
educational, parliament etc. with altogether over 74,000 words. We also checked that our
corpus covers all important phonological contrasts, sound combinations, and consonant
gradation patterns according to a grammatical description (Spiik, 1989), and in the case of
missing or very scarce gradation patterns, we added additional sentences to cover for these
and to balance the occurrences of various patterns. All texts will be open source, as well as
the recordings and the rest of the resources needed for the synthesis.

When using machine-learning methods to build up a speech model for TTS, the quality
of the recordings must be excellent, i.e., room reverberation or background noise must be
avoided in the recordings, because the noise would be modelled as well and audible in the
TTS output. Thus, the recordings must be done in a sound-treated room with professional
microphones and recording set-up. The minimal requirement for the audio recording is socalled CD quality (44.1 kHz sample, 16-bit). In spring 2022, we have recorded our first
TTS corpus of 10 hours with a male speech talent. Ultimately, our plan is to build both
male and female voices (each representing different areal/dialectal varieties of Lule Sámi)
and thus altogether 20 hours of speech is going to be recorded, processed, and made publicly available for future projects and for the language community to use.

After recording the speech corpus, a lot of effort must be put into cleaning the audio
from unnecessary noise (coughing, clearing one’s throat, background sounds if any) and
editing the original corpus texts so that they correspond to the audio as accurately as possible. Even mistakes, repetitions and hesitations are transcribed if they are not completely
cut out from the audio. In our case, we did not cut these because it would have led to great
loss of data. Subsequently, after editing the texts the audio and corresponding texts are
automatically time-aligned (i.e., force-aligned) using an online tool called WebMAUS. By using this tool together with manual
inspection and correction afterwards, finding the word and sentence boundaries are streamlined for effective processing of the data: in this way, the audio can be split into sentencelong files very quickly, as most of the modern machine-learning based TTS frameworks
require the training data to be in this format.

\subsubsection{Experiments with different frameworks}

As described in Makashova (2021), a North Sámi TTS voice was trained with a female
voice, with a data set consisting of 3500 training sentences, corresponding to approximately
3 hours of speech. The TTS model consisted of four components: Tacotron, ForwardTacotron, Tacotron2 and WaveGlow, the two latter ones from the official Nvidia repository13
. The training of this successful and good quality Tacotron model and the WaveGlow
model took one month, and for the ForwardTacotron three days, on a single graphical processing unit (GPU). Access to high-performance computing and storage services can significantly reduce the training time needed for the models.

To reduce the environmental costs of model training, one could consider adapting existing speech models by training the models further with additional data and pre-trained
models from a “neighbouring” language. This so-called transfer learning (Tu et al., 2019;
Debnath et al., 2020) allows for utilising smaller data sets for training, making it possible,
for example, to use the North Sámi TTS model as the starting point for the Lule Sámi TTS.

We actually tried this on a TTS model using transfer learning between North and Lule
Sámi. With a miniature data set (approx. one hour of speech data recorded with a cell
phone), we were able to train a Lule Sámi voice, but the quality of the output showed that
this corpus did not cover all necessary phonemes of the language and thus there were phonological inaccuracies. Moreover, the North and Lule Sámi orthographies are somewhat
different: for example, the alveolar fricative sound written in English as sh, is written as
<š> in North Sámi, and as <sj> in Lule Sámi. This creates unnecessary differences in the
training data. By converting both North and Lule Sámi texts to an approximate IPA (International Phonetic Alphabet) transcription, these differences could be resolved, and the
transfer learning would presumably be more successful.

\subsection{Corpus analysis and processing}

The GiellaLT infrastructure contains text corpora as well as a set of tools for treating them.
Given its rule-based approach, GiellaLT mainly use corpora for testing of tools, as well as
for linguistic research.

\subsubsection{Corpus infrastructure and tools}

The basic principle is to store all incoming material in a triple set of files: The original file
(untouched), an xsl file containing metadata (including information on languages found in
the document as well as special conversion challenges) and a derived XML file containing
the metadata and the content of the original file divided into paragraphs annotated for language. The process deriving the XML file can be repeated as the metadata is improved.
There are also tools for extracting different parts out of the corpus (only this or that
language, only list items, etc.). The metadata files also keep track of parallel language versions of a given file. Corpus collection, annotation and conversion is done by CorpusTools,
a software package we have made in order to add and administrate corpus content, create
and keep track of metadata files and convert files for internal and external use. The tools
also keep trach of translated files.


⏱
ߙ :1 mm
ߚ :3 mm
1.0: 6 mm

The corpora are then morphologically annotated and given syntactic functions and for
some languages also dependency relations by the FST and CG tools presented above, with
a very high degree of coverage. The corpora are presented for linguists and dictionary users
via the corpus interface Korp14
. At present, the GiellaLT instances of Korp present annotated corpora for 18 different languages, ranging from 300,000 to 65 million tokens. Parallel text is presented as such in Korp and also available for download as translation memories.


\subsubsection{Annotated corpora for special purposes}

To aid the development of spelling and grammar checkers, we have developed a language
independent markup of errors\footnote{}. We then manually mark errors in parts of our corpus files,
and use CorpusTools to extract error/correction pairs to test our tools. Here is an examples
from error markup of Lule Sámi:

\begin{verbatim}
{Valla aj la ájnas}¥{wo|Valla l aj ájnas} dádjadit dajt hásstalusájt {mij}£{congr,pron|ma} varresvuodan ja máhtsastimen
{lulu}£{congr,cond|lulun} gå ulmusj la sábme, tjielggi Trond Bliksvær.
Rahpamin lij aj {áttják}${vowlat,á-a|áttjak} álggám ådå sámeássjij stáhttatjálle Ragnhild Vassvik Kalstad.
\end{verbatim}

In 2013 we made a Lule Sámi gold corpus with this markup. The purpose was to test the
existing Lule Sámi spellchecker developed in 2007 against this gold corpus. In order to be
able to use the gold corpus to test the spellchecker, we collected texts that had not been
proofread before and with a range of different authors. We aimed for a gold corpus with at
least 1,000 typos (non-words) in a corpus of at least 20,000 words of continuous text. Over
a period of 9 months one Lule Sámi linguist used 390 hours to mark and correct a gold
corpus of 29,527 words with 1,505 non-word errors and additional 1,322 morpho-syntactic,
syntactic and lexical errors. Even though the purpose in 2013 was to test the Lule Sámi
spellchecker, it was important to mark all types of errors so that we could reuse the gold
corpus. We are now developing a Lule Sámi grammarchecker and have been testing the
grammarchecker on that same gold corpus (Mikkelsen et al. 2022). During the linguistic
work with building a Lule Sámi gold corpus, it was the markup format that was most challenging. It is important that the markup is consistent, the same type of error should always
be marked in the same way. In addition, the marking of errors should follow the same
pattern for all languages.

\subsection{Computer-assisted language-learning tools}

⏱
ߙ :₂₀ mm
ߚ :₂₀ mm
1.0: ⅟₂₀ mm

Computer-assisted language learning (CALL) is typically used towards vocabulary learning. Programs are enriched with various techniques for remembering new words, ranging
from encouraging association rules to following the learning process on a word-for-word
basis.

Learners of morphology-rich languages face more fundamental challenges. In addition
to learning the words, the pupils must also learn both their inflection patterns and in what
context each inflection form should be used. Within GiellaLT, the Oahpa learning platform
provides just that. It offers automatically generated exercises for practicing morphology,
with or without context. We have also been experimenting with a system-governed dialogue mode, where the system asks questions based upon input from the pupil and analyses
and comments the answers provided by the pupil. The morphological exercises are created
by means of finite state transducers, and the dialogue games are created by means of constraint grammar (see Antonsen et al 2009a, 2009b for a presentation).

A further extension to this was the Konteaksta program, based on the WERTi system
(Meurers et al 2010). The idea is to offer language teachers a possibility to give the pupils
authentic online texts and have the system creating exercises based upon an analysis of the
text in question. The exercise may be to find e.g. the finite verbs in a text, and then to do
multiple choice tests or type in the wordform, for these finite verbs. The philosophy is
partly that the pupils may be more motivated when the texts are interesting to them and
partly that the teachers may want to use the program if it really is able to offer the teacher
a time-saving way of making grammatical exercises. When adapting the program to North
Sámi, we had to find solutions for challenges in authentic texts, typical for a minority
language: variations in orthography, and large proportion of non-normative forms
(Antonsen and Argese 2018).

It is necessary to involve language teachers in the design of CALL programs. The
amount of work depends entirely on the design of the program.

\subsection{Software management, distribution, and updates}

Experience has shown that users have various problems installing and making use of the
tools offered by the GiellaLT infrastructure. In addition, when a tool is first installed, it is
rarely updated, leading to people being stuck with old versions and non-working software.

To avoid this, we have developed a package manager named Páhkat16 for the tools offered by the GiellaLT infrastructure. Users download this manager, install the tools for the
language(s) they want, and from there on, the package manager makes sure all dependencies are installed, and that all software is kept up to date. There are stand-alone app front
ends for Windows and macOS, whereas client libraries are built into mobile apps. This
means that the keyboard apps automatically keep the spellers up to date for users.

We use GitHub and continuous integration and deployment to ensure that the linguistic
resources at the centre of our tools such as spelling checkers are always up to date. The
system uses centralised build instructions in combination with Taskcluster17, and stores the
generated artifacts in a separate repository accessible by the Páhkat clients. All artifacts are
signed with developer certificates, and all communication is encrypted to ensure the integrity of the artifacts and the package manager system.

The work involved for a single language is minimal, a few initial configurations, and
tagged commits to get a new speller or keyboard out the door. Other tools still need to be
included in the automatic distribution, but as the system is developed more tools will be
added, and for each tool it will apply for all languages at once.

\section{The future}

The GiellaLT infrastructure is constantly being developed and enhanced. In this section we
will look into possible future paths for the infrastructure. It is split in three parts: first, we
will look at further development of rule-based tools, then have a peek at possible hybrid
systems, and finally what other technology development one could imagine.

\subsection{Rule based development}

No work has been done related to indexing and searching. While there has not been
enough texts to warrant a major effort in this direction, the North Sámi text production
within some institutions is growing all the time, thus the need will soon be there. The existing, rule-based tools are already fit for stemming and lemmatisation, and should be easy
to integrate with existing systems, but the work needs to be done. There also need to be a
fall-back system for unknown words.

Another area where not much work has been done is localisation. While we have
worked quite a lot on machine translation (see above 3.6), and we have done some work
on translation memory and general translation support, no effort has been made to set up a
system for streamlining software localisation. As the digital world becomes ubiquitous,
there will be an increased need for this type of service, especially for the education system.

A third area for future development is to expand the writing support tool set. Adding
synonym dictionaries, making more advanced use of our sentence parsing tools to offer
simple rewrites of sentences (cf e.g. Piotrowski and Mahlow 2009), or to help writers
choose more native constructions as opposed to using sentence structures borrowed from
the majority language. There are many possibilities, and we have just started.

One popular use of language technology is for named entity recognition (NER) (see
Ruokolainen 2020, Luoma et al 2020). Although nothing has been done explicitly to deliver
such a tool, it would be very easy to make one. All morphological analyses include explicit
tagging of names, and it is thus trivial to turn that into a basic NER tool. A bit of work to
cover unknown names should be added, but also that should not be too hard, and can probably be done in a largely language independent manner.

\subsection{Hybrid systems}

As described earlier, hybrid systems combine machine learning technologies with rulebased systems. In this section we look at automatic speech recognition, dialog systems and
text summarisation.

\subsubsection{Automatic speech recognition (ASR)}

In addition to TTS, we are working towards developing a tool for automatic speech recognition (ASR) for Sámi. This section describes materials and experiments only for North
Sámi, but in the future, we hope to expand our work to Lule Sámi ASR as well. In Makashova (2021), TTS and ASR models were trained simultaneously in a dual transformation
loop, using the same read speech data set, corresponding to only six hours of speech from
two speakers, three hours each. The ASR model in this work was based on the Wav2Vec2
model which is a part of the HuggingFace18 library. The model was trained for 30 000 steps,
reaching a WER (Word-Error-Rate) of 41% and 0.5 loss. The most common error types in
the ASR predictions seem to be in word boundaries (*earáláhkai – eará láhkái) and in
lengths of some sounds (*rinškit – rinškkit). However, these kinds of errors would be easy
to correct using Divvun’s spell checking software (described above in Section 3.2).
One of the most important differences between training the TTS and ASR models would
be that for TTS, the training material needs to be very clean in terms of sound quality and
there needs to be as many recordings from a single speaker as possible. For ASR, on the
other hand, the recorded materials can be of poorer sound quality and preferably from multiple speakers and from different areal varieties of a language as long as there are good and
accurate transcriptions of the speech.

State-of-the-art ASR frameworks normally require up to 10,000 hours of multi-speaker
data for training reliable and universal models that can generalise to any unseen speaker
(see, e. g., Hannun et al., 2014). As collecting these amounts of data from small minority
languages is not a realistic goal, alternatives such as utilising existing archive materials can
be considered for developing speech technology for Sámi. These are provided by, e.g., The
language bank of Finland and The language bank of Norway. These archive materials contain spontaneous, transcribed spoken materials from various dialects and dozens of North
Sámi speakers. Although extremely valuable, archive data may not be enough. Thus online
data sourcing campaigns, such as the ongoing Lahjoita puhetta19 (“Donate your speech”)
project for developing ASR for Finnish, should be considered.

\subsubsection{Dialogue systems}

The resulting TTS and ASR models under development can be used in building more advanced speech technology frameworks, such as dialogue systems (see, e.g., Jokinen et al.
(2017; Wilcock et al. (2017; Trong et al. (2019)) and various kinds of mobile applications
for, e.g., language learning. Dialog systems can also be used as front ends for public sector
services, and in parts of the health services.
As the society becomes more and more digitised, it is more and more crucial that also
minority languages can be used in all digital contexts. Eventually it should be possible to
get a North Sámi spoken dialog plugin for your electric car, or a South Sámi instance of
digital assistants in mobile phones.

\subsubsection{Text prediction}

People on mobile platforms nowadays expect their keyboards to offer suggestions on how
to complete the word they are writing, as well as also suggestions for the next word. This
is available for a number of majority languages. We have experimented a bit with this, but
nothing has entered production quality. One challenge is the scarcity of relevant text corpora, and further work needs to investigate what possibilities there are to alleviate this
shortcoming using various hybrid approaches.

\subsection{Other technology development}

The software technology develops much faster than linguistics and lexicography, this
means that in comparison to the linguistic resources that are long-term projects for decades
and centuries, the usage of those resources in computational linguistics change in shape
and form every few years. A system made for usage of our infrastructure and resource ten
years ago is probably outdated for most of the contemporary programmers.
There exists a number of various NLP platforms for various tasks, such as UralicNLP20
(Hämäläinen 2019) and NLTK21 (Bird et al 2009). It would be very useful to make the
GiellaLT platform and tools available as free-standing components that can be easily integrated with other tools such as the two mentioned, or preferably with premade integration,
so that the GiellaLT components can be fetched and used just like any other component,
independent of technology. Contrary to these platforms, GiellaLT puts an emphasis on the
language models, what is missing is routines for integrating the language models into modular systems.
For users it is important that the technology is available where the users are. That means
a never-ending story of making sure language technology tools get access to the platforms
users are using, adding new platforms and systems as needed. It also means we need to
work with technology producers to make them give access to their platforms as needed for
our users.

\section{Conclusion}

The GiellaLT infrastructure and most resources are available under open licences in
GitHub, free for anyone to use. Setting up all the surrounding infrastructure components,
CI/CD, build servers, distribution systems and so on is a non-trivial task, and we encourage
interested communities to use the existing GiellaLT infrastructure. Using it, one can build
a complete NLP system for a new language in a matter of months and have keyboards,
spelling checkers and other applications derived from it. This can be achieved virtually
with no pre-existing digital resources. Combined with a technology that fits very well for
morphologically or phonologically complex languages, it ensures that tools for the modern,
digital society can be developed for any language.

GiellaLT has permanent funding from a Norwegian ministry due to the needs of the
Sámi people of Norway, and in combination with open-source code everywhere possible,
this guarantees that the infrastructure and its resources will continue to live on and be maintained independent of other changes in the world.

New technologies will be added as needed, as shown above with speech technology in
a hybrid setup. Whatever the direction of the technological development, it is our goal to
ensure that the GiellaLT infrastructure stays updated and is developed in concert with the
needs of the society, combining existing and new technologies for new purposes.


\bibliographystyle{unsrt}
\bibliography{rblt2023}

\end{document}
% vim: set spell:
