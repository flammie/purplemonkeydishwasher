\documentclass[11pt,a4paper]{article}
\usepackage[margin=70pt,head=50pt,headsep=30pt,foot=50pt]{geometry}
\usepackage{fontspec}
\usepackage{xltxtra}
\usepackage{xspace}
\usepackage[english]{babel}
\usepackage{calc}
\usepackage[table]{xcolor}
\usepackage{verbatim}
\usepackage[normalem]{ulem}
\usepackage{multicol}
\usepackage[small,bf]{caption}
\usepackage{multirow}
\usepackage{natbib}
\usepackage{tikz-dependency}
\usepackage{setspace}
\usepackage{fancyhdr}
\usepackage{expex} % another possibility
\usepackage{url}


\title{Proposal: Creation of SIGUR (ACL SIG on Uralic languages)}
\date{}

\begin{document}
\maketitle

We submit this proposal to form a new ACL SIG on
computational linguistics of Uralic languages. The
aim of this SIG is to promote interest in
computational approaches to research of uralic
linguistics.

\section*{Past Events}

We held a workshop on computational linguistics of
Uralic languages in Tromsø,
Norway\footnote{\url{http://gtweb.uit.no/iwclul2015/}}
earlier this year. The prospect of forming an ACL
SIG was discussed among the participants and all
agreed that it would be a good time to found a SIG
in ACL.

We have issued a call for papers in Northern
European Journal of Language Technology (NEJLT) to
publish a special issue on Uralic language
technology containing revised papers from our
workshop and also an open call for new submissions.

Prior to the workshop conference series like nodalida have been central publication channel for the work on uralic computational linguistics within nordic countries. (One of the points of forming the SIG is to extend the co-operation to all sites interested in uralistics.)

\section*{Future plans}

For the next year we are organising the second edition of the 
International Workshop on Computational Linguistics for the Uralic
Languages  in co-operation with
the Hungarian Academy of science and the University of Szeged. The workshop
will be held in Szeged in January 2016. The workshop would be an ideal place to hold the election of the formed SIG board.

(Some blah about preparatory meetings we had in Budapest and Szeged.)

In CIFU XXX fenno-ugristics confence organised this year in Oulu on September we will arrange an ad-hoc meeting and hopefully gather more interested parties from the linguistic circles.

\section*{Rationale}
The existence of a special interest group specific to
Uralistics in computational linguistics is well-founded: on the
linguistic and computational side the Uralic languages share the
morphological and linguistic complexity that is characteristic to the whole
language group. The methodology for tackling the linguistic and
social / geo-political problems within the Uralic group is common and
the purpose of the SIG is to help researchers to share their work and
solutions. Research on Uralistics is scattered along sites in Northern
Europe, Russia, Hungary, Germany and Central Europe, the purpose of the
SIG is to promote co-operation among the sites and platforms for 
common work.

At the moment we have an infrastructure for rule-based free/libre open
source computational linguistics descriptions and tools in the
university of Tromsø. The founding of the SIG would hopefully
increase the interaction with various sites on the sharing of the
resources. 

\section*{Expressions of interest}

The following list contains 31 people who have expressed an interest in 
participating in the SIG:

\begin{itemize}
  \item Francis M. Tyers (UiT Norgga árktalaš universitehta)
  \item Tommi A. Pirinen (Dublin City University)
  \item Trond Trosterud (UiT Norgga árktalaš universitehta)
  \item Marja-Liisa Olthuis (UiT Norgga árktalaš universitehta)
  \item Rogier Blokland (Uppsala universitet)
  \item Niko Partanen (Albert-Ludwigs-Universität Freiburg)
  \item Wouter Van Hemel (Helsingin yliopisto)
  \item Jussi Ylikoski (UiT Norgga árktalaš universitehta)
  \item Lene Antonsen (UiT Norgga árktalaš universitehta)
  \item Anis Moubarik (Helsingin yliopisto)
  \item Aarne Ranta (Chalmers tekniska högskola)
  \item Kaili Müürisep (Tartu ülikooli)
  \item Tiina Puolakainen (Tartu ülikooli)
  \item Péter Koczka (Hungarian Academy of Sciences)
  \item Eszter Simon (Hungarian Academy of Sciences)
  \item Marina Fedina (Komi republican academy of state service and administration)
  \item Ivett Benyeda (Hungarian Academy of Sciences)
  \item Zsófia Schön (Ludwig-Maximilians-Universität München)
  \item Inari Listenmaa (Chalmers tekniska högskola)
  \item Jaak Pruulmann-Vengerfeldt (Tartu ülikooli)
  \item Heiki-Jaan Kaalep (Tartu ülikooli)
  \item Miikka Silfverberg (Helsingin yliopisto)
  \item Kimmo Koskenniemi (Helsingin yliopisto)
  \item Ciprian Gerstenberger (UiT Norgga árktalaš universitehta)
  \item Jeremy Bradley (University of Vienna)
  \item Stig-Arne Grönroos (Aalton yliopisto)
  \item Niklas Laxström (Helsingin yliopisto)
  \item Linda Wiechetek (UiT Norgga árktalaš universitehta)
  \item Sjur Moshagen (Sámediggi)
  \item Jack Rueter (Helsingin yliopisto)
  \item Veronika Vincze (Szeged university)
\end{itemize}


\end{document}
