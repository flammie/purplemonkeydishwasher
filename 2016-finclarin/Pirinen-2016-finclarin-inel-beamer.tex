\documentclass{beamer}

\usepackage{fontspec}
\usepackage{polyglossia}

\usepackage{graphicx}
\usepackage{color}
\usepackage{url}


\mode<presentation>
{
    \usetheme{HZSK}
}

\makeatletter
\newcommand\listofframes{\@starttoc{lbf}}
\makeatother

\addtobeamertemplate{frametitle}{}{%
  \addcontentsline{lbf}{section}{\protect\makebox[2em][l]{%
    \protect\usebeamercolor[fg]{structure}\insertframenumber\hfill}%
  \insertframetitle\par}%
}


\title{INEL resources and HZSK
\scriptsize{in Fin-CLARIN, Helsinki, 2016.\\
\url{http://inel.corpora.uni-hamburg.de}}}
\author{Tommi A Pirinen \scriptsize \guilsinglleft
tommi.antero.pirinen@uni-hamburg.de \guilsinglright }
\institute{Universität Hamburg}
\date{\today}

\begin{document}

\selectlanguage{english}

\maketitle

\begin{frame}
    \frametitle{Introduction}
    \begin{itemize}
        \item Grammars, corpora and language technologies for Indigenous
            Northern Eurasian Languages
        \item A long-term project (18 years) funded by Academy of Sciences and
            Humanities in Hamburg
        \item Research: Institut für Finnougristik / Uralistik at Universität
            Hamburg (IFUU)
        \item Infrastructure: Hamburger Zentrum für Sprachkorpora (HZSK)
    \end{itemize}
\end{frame}

\begin{frame}
    \frametitle{Languages}
    \begin{itemize}
        \item Uralic:
        \begin{itemize}
                \item Selkup (two varieties) [sel], Kamas $\dagger$ [xas],
                    Nenets (two varieties) [yrk]
                \item Komi (two varieties) [kom]
        \end{itemize}
        \item Altaic:
        \begin{itemize}
            \item Dolgan [dlg], Siberian Tatar [sty]
            \item Evenki (Northern) [evn]
        \end{itemize}
        \item Yenissean:
        \begin{itemize}
            \item Ket [ket]
        \end{itemize}
    \end{itemize}
\end{frame}

\begin{frame}
    \frametitle{Language map}
    \includegraphics[height=\textheight]{inelmap}
\end{frame}

\begin{frame}
    \frametitle{Goals}
    For each language (3 years):
    \begin{itemize}
        \item a corpus of annotated texts
        \item combining oral and written data
        \item to provide a solid empirical
            foundation for grammatical descriptions
    \end{itemize}
    Also:
    \begin{itemize}
        \item Inventory of existing resources
        \item Lexical data
    \end{itemize}
\end{frame}

\begin{frame}
    \frametitle{Data first phase}
    \begin{itemize}
        \item Kamas:
        \begin{itemize}
            \item \emph{Kamassisches Wörterbuch} (Donner, 1944),
                glossing complete
            \item Audio from last speaker, Klavdia Plotnikova: from AEDKL
                archive, KOTUS archives, partially transcribed
        \end{itemize}
        \item Selkup:
        \begin{itemize}
           \item A.~I.~Kuzmina's archive at IFUU: field notes, recordings, ...
           \item Recordings by G.~I.~Pelikh (1960s)
           \item Recordings by S.~V.~GLushkov (1990s)
        \end{itemize}
    \end{itemize}
\end{frame}

\begin{frame}
    \frametitle{Timeline}
    \includegraphics[width=\textwidth]{ineltimeline-macro}
\end{frame}

\begin{frame}
    \frametitle{HZSK infrastructure}
    \includegraphics[width=\textwidth]{hzskrepo}

\end{frame}


\begin{frame}
    \frametitle{People}
    \begin{itemize}
        \item INEL: \\
            Béata Wagner-Nagy (Project leader)\\
            Michael Rießler (Principal Investigator?)\\
            Alexander Arkhipov (Research coordinator)\\
            Timm Lehmberg (Technical coordinator)\\
            Maria Brykina, Svetlana Orlova, Tiina Klooster, Chris Lasse Däbritz,
            Josefina Budzisch, Hannah Wegener (Linguists)\\
            Daniel Jettka, Niko Partanen (Developers)
        \item HZSK / CLARIN-D: \\
            Hanna Hedeland, Tommi A Pirinen
    \end{itemize}
\end{frame}

\end{document}
