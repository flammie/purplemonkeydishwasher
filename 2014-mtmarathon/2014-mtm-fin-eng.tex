\documentclass{pbml}
%\documentclass[nofonts]{pbml} % for XeLaTeX without Pagella and DejaVu fonts
%\documentclass[color]{pbml} % for color images and hypertext links

% This is a sample file for the PBML article.
% You can compile it with (ordered by preference)
%  1. XeLaTeX with installed fonts TeX Gyre Pagella and DejaVu
%  2. XeLaTeX without the fonts -> you must use \documentclass[nofonts]{pbml}
%  3. pdfLaTeX
%
% In all three methods, you can use Unicode (utf8) encoding for special letters
% (so instead of na\"{i}ve you can write directly naïve).
% Note that with methods 2 and 3, your output will be slightly DIFFERENT than
% our final print version (different line breaks and page breaks).
% See the end of this file for more information.

% Packages
% ========
% In this place you can load required packages by the \usepackage commands.
% The following packages are loaded automatically by the class:
% euler    (for math fonts)
% graphicx (for inclusion of images)
% multicol (for multicolumn typesetting)
% natbib   (for bibliography citations)
% amssymb  (for various symbols)
% If XeLaTeX is used, fontspec and xltxtra are loaded as well.

% Definitions
% ===========
% You can use your own macros defined by \newcommand, \providecommand,
% \DeclareRobustCommand (or even by \def) as well as environments declared by
% \newenvironment. You can also use \newcounter in order to declare your own counters.

\begin{document}

% Document title and authors
% ==========================
% Due to journal organization, the article title and authors must be specified
% AFTER \begin{document}. Note that \subtitle is not allowed anymore due to
% problems with indexing in science databases. If needed use colon in the title.

\title{Unseen Teddy-Bears and Morphological Segments in SMT and RBMT\titlelinebreak{}
an Experiment in Morphological Joining and Splitting with English-Finnish MT systems}


% Now put the affiliated institutes first, then author names and the labels
% of their institutes in the "institute" field. The order of institutes and
% authors printed below the title will be the same as the order of your commands.
% Each author can be associated with more than one institute.
% One author must be chosen as the "corresponding author" (using attribute
% corresponding) and his or her email and full address must be provided.
% PLEASE, use Unicode (utf8) encoding (e.g ï instead of \"{i}).

\institute{label1}{UiT Norgga árktalaš universitehta}
\institute{label2}{CNGL-School of Computing, Dublin City University}
\institute{label3}{Helsingin Yliopisto}

\author{
  firstname=Francis,
  surname=Tyers,
  institute=label1,
}
\author{
  firstname=Tommi,
  initials=A,
  surname=Pirinen,
  institute=label2,
  corresponding=yes,
  email={tommi.pirinen@computing.dcu.ie},
  address={Dublin City University\\Glasnevin, Dublin 9, Dublin, Ireland}
}
\author{
  firstname=A,
  initials=B.,
  surname=C,
  institute=label3,
}

% If all authors belong to the same institute, you can use simpler syntax:
% \institute{}{Charles University in Prague, Faculty of Mathematics and Physics, Institute of Formal and Applied Linguistics}
% \author{firstname=Humpty, surname=Dumpty}
% \author{firstname=Mock, surname=Turtle,
%   corresponding=yes,
%   email={turtle@seacoast.wl},
%   address={Institute of Formal and Applied Linguistics\\
%            Faculty of Mathematics and Physics,\\
%            Charles University in Prague\\
%            Malostranské náměstí 25\\
%            118 00 Praha 1, Czech Republic}}
% \author{firstname=Cheshire, surname=Cat}


% The title and authors' names are used in the running head. If they are
% long, you should define short versions. These definitions are optional. You
% define them only if they are needed. The example follows:
\shorttitle{Finnish-English MT Experiments}
\shortauthor{F. Tyers, T. A Pirinen, ABC}

% Now print the title by:
\PBMLmaketitle


% Abstract
% ========
% The abstract is placed within the "abstract" environment. It is a mandatory
% part of the article. PLEASE, do not use your own macros in abstract, if possible.

\begin{abstract}


\end{abstract}

\section{Introduction}
% The body of the article
% =======================
% The PBML class is modelled after the standard article class. This means
% that you can use almost everything that is allowed in articles as described
% in the textbooks of LaTeX. We support sectioning commands \section,
% \subsection and \subsubsection.

% In addition to the \cite command, you can use natbib style of citations.
% PLEASE, use \citet instead of \cite if the author names are part of the sentence.

Compound joining is an important part of the machine translation when dealing
with translation from morphologically less productive language in the
compounding department to a language where compounding is productive and
common. \citet{cap2014produce} has described a method for combining CRF-based
(conditional random field) statistical compound joiner to German-Russian
statistical machine translation.

The inverse process, compound segmentation, as well as any other
morphologically relevant segmentation of semantic units, is required when
translating to direction of morphologically rich language to one that uses
syntactic and lexical means for expressing the concepts.
\citet{fishel2010linguistically} has shown that for Estonian to English the
morphological segmentation shows promising increase in scores.

In this paper we study the state-of-the-art methods of morphological splitting
and joining when combined with baseline statistical and rule-based machine
translation systems using Finnish and English as the experiment pair. For the
baseline in SMT we have selected moses, and the language models specified on
their baseline
documentation\footnote{\url{http://www.statmt.org/moses/?n=moses.baseline}},
the corpus used is europarl~\citep{koehn2005europarl} as in the example, for
tuning XXX???  For RBMT we use apertium~\citep{forcada2011apertium} and the
Finnish---English translation pair̃\footnote{\url{SVN URL}} we have built
earlier. The experiments we make and data needed for reproducing the results
are available freely in our github repository\footnote{\url{}}.


% For figures and tables always use "figure" resp. "table" floating environments
% and always supply a caption. See the paragraph about color images below.
%\begin{figure}
% \begin{center}
%  \includegraphics[width=\textwidth]{example_figure}
%  \caption{Example figure}\label{fig:example}
% \end{center}
%\end{figure}

\section*{Acknowledgements}
This research is supported by \ldots

% Bibliography
% ============
% You may either enter the bibliography manually, possibly making use of
% \label and \ref, or use BibTeX. The bibliography style is set
% automatically. You process the bibliography by BibTeX in the
% standard way and include it by:
\bibliography{mtm2014}

% If needed, add appendices here
%\section*{Appendix A: \ldots}

\correspondingaddress
\end{document}

% ======= Additional information ==========

% XeLaTeX
% =======
% The Prague Bulletin of Mathematical Linguistics is typeset by XeLaTeX.
% It is not required that you use XeLaTeX and the same fonts as will be used in the
% journal but it is better to do so if you could (so you have the same line and page breaks).
% If XeLaTeX is not available on your computer, you can even used standard LaTeX.

% Fonts
% =====
% In the printed version, fonts TeX Gyre Pagella (shipped with TeX Live 2008 or later)
% and DejaVu (http://dejavu.sourceforge.net/wiki/index.php/Main_Page)
% will be used. If you have XeLaTeX but not these fonts, you may need to install
% the fonts somewhere where your system (fontconfig) can find them. On Linux,
% try something like this:
% ln -s YOUR-TEXLIVE/2008/texmf-dist/fonts/opentype/public/ \
%  ${HOME}/.fonts/texlive-2008-otf-public
% If you are not able to install these fonts, you can instruct XeLaTeX
% to use its default fonts by \document[nofonts]{pbml}.

% If you want to typset examples in east Asian scripts, you have to use
% OpenType Unicode fonts that are freely redistributable and you have to
% include them with your article. If you must use nonfree or non-Unicode
% fonts, you must supply the examples as EPS or PDF with fonts embedded in
% the files (as the required subset).
% If encountering problems with the "bidi" package, contact pbml@ufal.mff.cuni.cz.

% Vector Graphics
% ===============
% The most portable graphics package is tikz which is a front end to pgf.
% This package is fully supported both in standard LaTeX and XeLaTeX.
% PSTricks is not currently available for XeLaTeX.

% Color Images
% ============
% The printed version of PBML does not allow color figures, so please convert
% all your figures to grayscale (or black&white), make sure they are legible
% and put them in a subdirectory called "grayscale".
% When specifying the filename in \includegraphics DO NOT include the
% subdirectory.
% Optionally, in addition to the default grayscale images, you can add also
% color versions of the figures to be published in the online version.
% Put such images in a subdirectory called "color" with the same filenames
% as the grayscale ones.
% To temporarily switch to color output and hypertext links, use
% \documentclass[color]{pbml}

% Including images and other files
% ===================================
% When loading the file you must use a relative path, never the full path.
% Load the images by \includegraphics
% and other files by \input, never use \include.

% vim: set spell:
