\documentclass[a4paper,12pt]{article}
\usepackage[latin1]{inputenc} % or another input encoding, as needed
\usepackage{fullpage}
\usepackage{covington} % or another package for examples, as needed
\usepackage{natbib}
%\usepackage[T1]{fontenc} % or another encoding, as needed
\pagestyle{empty}
%\bibliographystyle{natbib.fullname} % can be downloaded from T�bingen
\bibliographystyle{cslipubs-natbib} % can be downloaded from CSLI Publications

\title{Title of the Article} % All words except prepositions, conjunctions and determiners capitalized

\author{Firstname Lastname\\ % name
 [0.5cm] University of Somewhere\\ % top level affiliation
 Department of Something\\ % basic academic or research unit
 \texttt{email@domain}}   % email

\date{}

\begin{document}
\maketitle 
\thispagestyle{empty}

\begin{abstract}
\noindent 
The abstract should be between five and eight lines in length.
It should provide a very concise summary of the research goal, data, method, and results.
The page size is A4 with margins of about 2 cm.
Pages should not be numbered, nor should there be headers or footers.
The text should be set in a standard roman 12 pt serif font with justified lines, allowing hyphenation to achieve even spacing.
\end{abstract}

\section{Section Heading} % All words except prepositions, conjunctions and determiners capitalized

Bibliographic references should use the author and year format \citep[section 12.3]{Mittelbach04}.  Bibliographic items in the References should contain full names, in the style of Computional Linguistics and CSLI Publications.  Style file \emph{cslipubs-natbib.bst} can be downloaded from CSLI Publications.

Examples should be in italics and numbered with Arabic numbers in parentheses, as illustrated in example (\ref{ex:bjoerka}).
Examples in languages other than English should contain glosses, preferably adhering to the Leipzig glossing rules when dealing with morphologically rich languages.

\begin{example} \label{ex:bjoerka}
\gll	Her g�r bj�rk-a over 1200 meter, det er h�yest i land-et.
	here goes birch-the over 1200 meters, that is highest in country-the
\glt	``Here birches grow over 1200 meters, that is the highest in the country."\glend
\end{example}

\noindent Figures should have a numbered caption below, while tables should have a numbered caption above.  Tables have thin borders on all sides and may have borders between cells as visually appropriate.

\bibliography{kds}
\end{document} 