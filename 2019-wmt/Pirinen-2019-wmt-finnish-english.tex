%\title{emnlp 2017 instructions}
% File emnlp2017.tex
%

\documentclass[free]{flammie}

\usepackage{url}
\usepackage{booktabs}

%\aclfinalcopy % Uncomment this line for the final submission
%\def\aclpaperid{***} %  Enter the acl Paper ID here

%\setlength\titlebox{5cm}

\newcommand\BibTeX{B{\sc ib}\TeX}



\title{Apertium-fin-eng---Rule-based shallow machine translation for WMT 2019
shared task}


\author{Tommi A Pirinen\\
    Universit\"at Hamburg\\
    Hamburger Zentrum f\"ur Sprachkorpora\\
    \url{tommi.antero.pirinen@uni-hamburg.de}
}



\begin{document}
\maketitle
\begin{abstract}
    In this paper I describe a rule-based, bi-directional machine translation
    system for the Finnish---English language pair. The original system is
    based on the existing data of FinnWordNet, omorfi and apertium-eng.
    I have built the disambiguation, lexical selection and translation rules
    by hand. The dictionaries and rules have been developed based
    on the shared task data. I describe in this article the use of the
    shared task data as a kind of a test-driven development workflow in RBMT
    development and show that it suits perfectly to a modern software
    engineering continuous integration workflow of RBMT and yields big increases
    to BLEU scores with minimal effort. The system described in the article
    is mainly developed during shared tasks.\@
\end{abstract}


\section{Introduction}\label{sec:introduction}

\footnotepubrights{Official version in ACL Anthology:
\url{https://www.aclweb.org/anthology/sigs/sigmt/} (to appear), CC-BY version
4.0 international: \url{https://creativecommons.org/licenses/by/4.0/}}

This paper describes our submission for Finnish---English language pair to the
machine translation shared task of the \textit{Fourth conference on machine
translation} (WMT19) at ACL 2019. Traditionally \textit{rule-based machine
translation} (RBMT) is not in the focus for WMT shared tasks, however,
there are two reasons I experimented with this system this year. One is that we
have had an extensively large amount of lesser used resources for this pair:
omorfi\footnote{\url{https://github.com/flammie/omorfi}}~\cite{omorfi} has well
over 400,000
lexemes\footnote{\url{https://flammie.github.io/omorfi/statistics.html}},
apertium-eng\footnote{\url{http://wiki.apertium.org/wiki/English}} has over
40,000 lexemes and
apertium-fin-eng\footnote{\url{https://github.com/apertium/apertium-fin-eng}}
over 160,000 lexeme-to-lexeme translations. One of our key interests in the
shared task like this is that it provides an ideal data for test-driven
development of lexical resources.

One concept I experimented with the shared task is various degrees of
automation---expert supervision for the lexical data enrichment. In this
experiment I used automatic methods to refine the lexical selection of the
machine translation, and semi-automatised workflows for the generation
of the lexical data, as well as some expert-driven development of the more
grammatical rules like noun phrase chunking and determiner generation.
It might be noteworthy that this machine translator I describe in the
article is not actively developed outside the shared tasks,
so the article is moreso motivated as
an exploration of the workflow and methods on semi-automatically generated
shallow RBMT than a description of a fully developed RBMT.\@


The rest of the article is organised as follows: In Section~\ref{sec:data}
I describe the components of our RBMT pipeline, in Section~\ref{sec:training}
I describe the development workflow and in Section~\ref{sec:error-analysis} I show
the shared task results and I perform error
analysis and discuss the results and finally in Section~\ref{sec:conclusions} we
summarise the findings.

\section{System description and setup}
\label{sec:data}

The morphological analyser for Finnish is based on omorfi~\cite{omorfi}, a large
morphological lexical database for Finnish. Data from omorfi has been converted
into Apertium format and is freely available in the apertium-style format in the
github repository
apertium-fin\footnote{\url{https://github.com/apertium/apertium-fin}}. For
English I have used Apertium's standard English analyser
apertium-eng\footnote{\url{https://github.com/apertium/apertium-eng}}. Both
analysers were downloaded from github in the beginning of the shared task and we
have updated and further developed them based on the development data during the
shared task. I developed the  Apertium's Finnish-English\footnote{\url{https://github.com/apertium/apertium-fin-eng}} dictionary initially
based on the FinnWordNet's translated data, which was over 260,000 Wordnet-style
lexical items; of these I discarded most which had multiple spaces in them or
didn't match any source or target words in Finnish and English dictionaries,
ending with around 150,000 lexical translations.  The size of dictionaries at
the time of writing is summarized in Table~\ref{table:dictionaries}, however
more up-to-date numbers can be found in Apertium's
Wiki~\footnote{\url{http://wiki.apertium.org/wiki/List_of_dictionaries}}

\begin{table}
    \begin{center}
    \begin{tabular}{lrr}
        \toprule
        Dictionary & Lexemes & Manual rules \\
        \midrule
        Finnish & 426,425 & 143 \\
        English & 40,185 & 187 \\
        Finnish-English & 164,501 & 273 \\
        \bottomrule
    \end{tabular}
    \caption{Sizes of dictionaries. The numbers are numbers of unique word
    entries or translation entries as defined in the dictionary, e.g., homonymy
    judgements have been made by the dictionary
    writers. The rule counts are combined counts of all sorts of linguistic
    rules: disambiguation, lexical selection, transfer and so
    forth.\label{table:dictionaries}}
    \end{center}
\end{table}


The system is based on the Apertium\footnote{\url{https://github.com/apertium}}
machine translation platform~\cite{forcada2011apertium}, a shallow transfer
rule-based machine translation toolkit. For morphological analysis and
generation, HFST\footnote{\url{https://hfst.github.io}}~\cite{linden2011hfst} is
used and for morphological disambiguation VISL
CG-3~\footnote{\url{http://visl.sdu.dk/cg3.html}} is used.  The whole platform
as well as all the linguistic data are licensed under the GNU General Public
Licence (GPL).

Apertium is a modular NLP system based on UNIX command-line ideology. The source
text is processed step-by-step to form a shallow analysis (morphological
analysis), then translated (lexical transfer) and re-arranged (structural
transfer) to target language analyses and finally generated (morphological
generation). Each of the steps can be processed with arbitrary command-line tool
that transforms the input in expected formats. All of the steps also involve
ambiguity or one-to-many mappings, that requires a decision, and while these
decisions can be made using expert written rules, the writing of the rules is
also a demanding task, and it is interesting to see how much can be achieved by
simply bootstrapping the rulesets using automatic rule acquisition.

To illuminate how apertium does RBMT in Finnish---English, and the kinds of
ambiguities I resolve, I show in Table~\ref{table:translation} examples of the
ambiguities with an example sentence. The ambiguity of source morphology is the
true ambiguity rate of the language (according to the morphological analyser),
i.e.\ how many potential interpretation each word has. It is no surprise that
Finnish has relatively high ambiguity rate, however, English is nearly
unambiguous is more due to limitation of apertium's English dictionary than
feature of English per se, given that English has a bit of productive
zero-derivations, e.g.\ verbing nouns and vice versa. The lexical selection
ambiguity is the translation dictionary's rate of choices per source word, and
FinnWordNet on average has 5 synonyms per word to suggest. The target morphology
ambiguity is the rate of allomorphy or free variation, in Finnish as target
language there's some systematic problems, such as plural genitives and
partitives, whereas English literally has two incidents in the whole dev set:
\textit{sown / sowed} and \textit{fish / fishes}.  Assuming a perfect RBMT
system would keep all options open, until final decision, the number of
hypotheses at the end would be at least
$\mathrm{MA_{SL}}\times\mathrm{LS_{SL\rightarrow TL}}\times\mathrm{MA_{TL}}$,
where $MA$ is morphological ambiguity rate, $LS$ is lexical selection ambiguity
rate, $_{SL}$ is source language and $_{TL}$ is target language. For
Finnish---English I show the example figures of the ambiguities based on the
development and test sets in Table~\ref{table:ambiguities}.

\begin{table*}
    \begin{center}
    \begin{tabular}{lrrrr}
        \toprule
        \hfill Feature: & \bf Source & \bf Lexical & \bf
        Target & \bf \it Total\\
        Corpus & \bf morphology & selection & morphology & \\
        \midrule
        Finnish dev set & 1.68 & 5.04 & 1.0002 & 8.46 \\
        Finnish test set & 1.69 & 4.80 & 1.0003 & 8.13 \\
        \midrule
        English dev set & 1.04 & 1.15 & 1.0013 & 1.19 \\
        English test set & 1.03 & 1.12 & 1.0006 & 1.15\\
        \bottomrule
    \end{tabular}
    \caption{Ambiguity influencing RBMT Finnish-to-English and
    English-to-Finnish\label{table:ambiguities}}
    \end{center}
\end{table*}

The rule-based machine translation process as it is performed by apertium is
shown in Table~\ref{table:translation}. The first step of the RBMT here is
morphological analysis, in apertium this covers both tokenisation and
morphological analysis as seen here; in apertium-eng the expression `in front
of' is considered to be a single token and is packaged as a preposition (we have
also omitted an ambiguity between attributive and nominal reading of the house,
since the distinction does not currently make difference in English to Finnish
translation, in order to fit the table in the paper). The morphological analysis
in apertium is performed by finite-state morphological analysis as defined
in~\cite{beesley2003finite} and implemented in open source format
by~\cite{linden2011hfst}.  After analysis, the next step is to disambiguate,
i.e.\ pick 1-best lists of morphological analyses; in apertium this is done by
constraint grammar, as described by~\cite{karlsson1990constraint} and
implemented in open source by VISL CG
3.\footnote{\url{http://visl.sdu.dk/cg3.html}}. In lexical translation phase,
each lemma is looked up from the translation dictionary, and in lexical
selection the translation that is most suitable by the context and statistics is
selected. In the structural transfer phase a number of things is performed: the
English morphological analyses are rewritten into Finnish analyses, e.g.\  the
adjective and noun will receive a genitive case tag due to the adposition, and
the adposition is moved before the noun phrase since it is a preposition in
Finnish and postposition in English, and the article is just removed, as the use
of articles is non-standard in Finnish.  Finally the Finnish analysis is
generated into a surface string using a finite-state morphological analyser,
since they are inherently bidirectional this needs no extra software or
algorithms.


\begin{table*}
\begin{center}
    \begin{tabular}{lp{0.66\textwidth}}
\toprule
\bf Input: & In front of the big house \\
\midrule
\bf Morphological analysis: & In front of.\textsc{Prep} the.\textsc{Det.Def.Sp}
big.\textsc{Adj} house.\textsc{N.Sg}
\\
\midrule
\bf Morphological disambiguation: &
In front of.\textsc{Prep} the.\textsc{Det.Def.Sp}
big.\textsc{Adj} house.\textsc{N.Sg}
\\
\midrule
\bf Lexical translation: &
In front of.\textsc{Prep}$\rightarrow$Edess\"a.\textsc{Post}
the.\textsc{Det.Def.Sp}$\rightarrow$se.\textsc{Det.Def.Sp}
big.\textsc{Adj}$\rightarrow$iso$\sim$raju$\sim$paha$\sim$kova$\sim$\ldots
jalomielinen.\textsc{adj}
house.\textsc{N.Sg}$\rightarrow$huone$\sim$talo$\sim$suku$\sim$\ldots
edustajainhuone.\textsc{N.Sg}
\\
\midrule
\bf Lexical selection: & Edess\"a.\textsc{Post}  se.\textsc{Det.Def.Sp} iso.\textsc{Adj} talo.\textsc{N.Sg}
\\
\midrule
\bf Structural transfer: & iso.\textsc{Adj.Pos.Sg.Gen} talo.\textsc{N.Sg.Gen}
Edess\"a.\textsc{Post}
\\
\midrule
\bf Finnish translation: & ison talon Edess\"a \\
\bottomrule
\end{tabular}
\caption{Translation process for the English phrase `In front of the big house'
\label{table:translation}
    }
\end{center}
\end{table*}

\section{RBMT development workflow}
\label{sec:training}

I present here different levels of automation in the RBMT workflow: in
Subsection~\ref{sec:lexical-selection} I have automated the generation of
rules, in Subsection~\ref{sec:lexicon-development} I have a semi-automated
workflow and finally in Subsection~\ref{sec:grammar-development} I have
an expert-driven development workflow.

\subsection{Lexical selection training}
\label{sec:lexical-selection}

One of the key components of this experiment was to try automatic rule-creation
mechanisms for the converted Wordnet dictionary refinement.  A large number of
translation quality issues in the initial converted Wordnet dictionary was a
high number of low-frequency `synonyms' in translations. To overcome this some
automatic methods were used. For automatic bootstrapping of the lexical
selection rules I used Europarl corpus~\cite{koehn2005europarl} data and the
methods demonstrated by~\cite{tyers2012flexible}.  Since the result of this
training seemed also insufficient, I experimented with another system to
generate more rules for lexical
selection.\footnote{\url{https://svn.code.sf.net/p/apertium/svn/trunk/apertium-swe-nor/dev/lex-learn-unigram.sh}}
On top of that, I have updated the lexical selection with some manual rules,
that were either not covered by Europarl hits or skewed wrongly for the news
domain, for example, the word `letter' seemed to mainly have translations of
\textit{kirje} (a message written on paper), while in the development set all
the sentences I sampled, a more suitable translation would of been
\textit{kirjain} (a character of alphabet).  The resulting lexical selection
rule sets are summarised in the table~\ref{table:lexical-selection}. The first
method of creating rules is based on n-gram patterns, due to restricted time and
processing resources I have only included bigrams into this model, and the
second model only considers unigrams.  The results are added up in the table
lines \textit{+ bigrams} and \textit{+ unigrams} respectively.

\begin{table}
\begin{center}
    \begin{tabular}{lr}
        \toprule
        Orig. Fin-Eng & 18,066 \\
        + bigrams &     24,662\\
        + unigrams &    30,049 \\
        \midrule
        Orig. Eng-Fin & 22 \\
        + bigrams & 24,631 \\
        + unigrams & 25,748 \\
        \bottomrule
    \end{tabular}
    \caption{Lexical selection rules statistically generated\label{table:lexical-selection}}
\end{center}
\end{table}


\subsection{Lexicon development workflow}
\label{sec:lexicon-development}

One of the key components of this experiment is to show that a
\textit{shared-task driven development} (STDD) is a usable workflow
for the development of the lexical data in rule-based machine translation
system.  As such, a `training' phase in the RBMT development has been replaced
by a very simple semi-automated native speaker -driven project workflow
consisting of following:

\begin{enumerate}
    \item Collect all lexemes unknown to source language dictionary, and add them
        with necessary morpholexical information
    \item Collect all lexemes unknown to bilingual translation dictionary,
        and add their translations
    \item Collect all lexemes unknown to the target language dictionary,
        and add them to the dictionary with necessary morpholexical information
\end{enumerate}

The semi-automation that I have developed lies in collecting the different
unknown lexemes or \textit{out-of-vocabulary} items (OOVs), and guessing a
lexical entry or multiple plausible entries for them and have the dictionary
writer select and correct them.


\subsection{Grammar development}
\label{sec:grammar-development}

An expert-driven part of the RBMT workflow in our current methodology is the
grammar development. This consists manually reading the sentences produced by
the MT system to spot systematic errors caused by grammatical differences
between languages. For the purposes of this shared task and the workshop, the
linguistics or grammar are not a central concept, so I will not detail it here
in detail. In practice this concerns of such grammatical rules as mapping
between no articles in Finnish to articles in English, mapping between case or
possessive suffixes and their corresponding lexical representations in English
and so forth. The details can be seen in the code that is available in github.


\section{Evaluation, error analysis and discussion}
\label{sec:error-analysis}

The automatic measurements as used by the shared task are given in the
table~\ref{table:bleu}. I show here the BLEU~\cite{papineni2002bleu} and the CharacTER scores.
BLEU, as it is a kind of industry standard, and CharacTER~\cite{wang2016character} as it is maybe more suited for
morphologically complex languages. As the automatic scores show, the rule-based
system has still room for improvement.

\begin{table}
\begin{center}
    \begin{tabular}{lrrr}
        \toprule
        \bf Corpus & BLEU-cased & CharacTER \\
        \midrule
        MSRA.NAO & 27.4 & 0.515 \\
        HelsinkiNLP RBMT & 8.9 & 0.650 \\
        \bf apertium-eng-fin & 4.3 & 0.756  \\
        \midrule
        USYD & 33.0 & 0.494 \\
        \bf apertium-fin-eng & 7.6 & 0.736  \\
        \bottomrule
    \end{tabular}
    \caption{automatic scores from \url{http://matrix.statmt.org}, we show our scores (boldfaced), the highest
    ranking RBMT and the highest ranking NMT for reference.\label{table:bleu}}
\end{center}
\end{table}



I find that a linguistic error analysis is one of the most interesting part of
this experiment. The reason for this is is that the experiment's scientific
contribution lies more in the extension of linguistic resources and workflows
than machine learning algorithm design. It is noteworthy, that in
the sustainable workflow I demonstrate in this article, error analysis is a
part of the workflow, namely, adding of the lexical data and rules follows the
layout given in Section~\ref{sec:training} and is the same for development and
error analysis phase. I have, to that effect, categorised the errors in
translations along the workflow:

\begin{enumerate}
    \item OOV in source language dictionary (including typos and non-words)
    \item OOV in bilingual dictionary
    \item OOV in target language dictionary
    \item disambiguation or lexical selection fail
    \item structural failure or higher level
\end{enumerate}

The OOV's can be calculated automatically from the corpus data, but the higher
level failures need human annotation. A summary of the errors can be seen in the
table~\ref{table:errors}, this is based on the errors that were fixed as a part
of error analysis process. As a result of this workflow, I have improved the
BLEU points of apertium-fin-eng over the years, as can be seen in the
table~\ref{table:progress}.

\begin{table}
\begin{center}
    \begin{tabular}{lr}
        \toprule
        Error & count \\
        \midrule
        OOVs in Finnish & 763 \\
        OOVs in English & 943 \\
        OOVs in Fin↔Eng & 2696 \\
        \bottomrule
    \end{tabular}
    \caption{Classification of mainly lexical errors in apertium-fin-eng
    submissions for 2019\label{table:errors}}
\end{center}
\end{table}


\begin{table}
\begin{center}
    \begin{tabular}{lr}
        \toprule
        \bf Corpus & BLEU-cased  \\
        \midrule
        apertium-eng-fin 2015 & 2.9 \\
        \hfill 2017 & 3.5  \\
        \hfill 2019 & 4.3  \\
        \midrule
        apertium-fin-eng 2015 & 6.9\\
        \hfill 2017 & 6.3 \\
        \hfill 2019 & 7.6 \\
        \bottomrule
    \end{tabular}
    \caption{Progress of apertium-fin-eng over the years using only the WMT
    shared task driven development method.\label{table:progress}}
\end{center}
\end{table}

The OOV numbers might look moderately large but a major part falls under proper
nouns, which are generally low frequency and do not cause a large problem in
translation pipeline, the untranslated proper noun is recognisable and the
mapping of adpositions and case inflections will fail where applicable. The task
of adding proper nouns to the dictionaries is also simplest, they are easy to
gather from the text, and for English and bilingual dictionaries no further
classification is necessary; for the Finnish dictionary entry generation,
paradigm guessing is necessary, although the paradigms used in foreign names are
much more limited than with other parts-of-speech to be added. In the
\textit{newstest 2019} data there was a number of words that I decided not to
add to our dictionaries, unlike our usual workflow where I aim at virtual 100
\% coverage with gold corpora. The unadded words were for example words like
``Toimiluvanmuodossatoteutettavajulki-senjayksityisensektorinkumppanuus'', which
seems to have a large number of missing spaces and extra hyphen, these as well
as extraneous spaces were quite common in the data in our error analysis as well
as `words' like `OIet', `OIi', `OIin', `OIisi', `OIIut', i.e.\ forms of `olla'
(to be) where lowercase L has been replaced with uppercase I. While I do
account for common spelling mistakes in our dictionaries, these kind of errors
are probably more suited for robustness testing and implemented with spelling
correction methods for specific problematic generated text, such as OCR\@.  We
will look into implementing spelling correction into our pipeline in the
future. Comparing the performance of RBMT to NMT, it can be clearly seen that
contemporary NMT is better suited for error tolerance, in part because it can
be more character-based than token-based, in part because any large training
data set will actually have some OCR errors and run-in tokens.

After OOV-errors one of the biggest easily solvable problems is ambiguity, so
word sense disambiguation and lexical selection. For lexical selection I found
about 200 lexical translations that were still badly wrong and could be solved
without coming up complex context conditions. For disambiguation problems, a
surprisingly common problem was sentence-initial proper noun that is a common
noun as well, as a high frequency example, for the word `trump' meaning a
winning suit in card games (= Finnish `valtti') would get selected over the
POTUS, plausibly when most of the training and development before WMT 2019 did
not contain so many proper noun Trumps. Also rather common problem still is the
ambiguity in English verb forms, and between English zero derivations.

In the structural transfer a large number of errors are caused by long-distance
re-ordering. For example for Finnish to English proper noun phrases regardless
of length of the phrase, the Finnish shows case in last word or postposition
after the last word, English has preposition before the word, but when phrase
gets chunked partially the adpositions or case suffixes end up in the middle
with a rather jarring effect to the translated sentence. The same applies for
other effects where generating correct language depends on correct chunk
detection, e.g.\ the article generation is very limited in the current code
because the articles need to be generated from nothing, when translating from
Finnish to English, only at the very beginning of specific noun phrases.

Finally a number of problems were caused for such grammatical differences
between languages that do not have a good solution in lexical rule-based machine
translation, such as difference between English noun phrases and corresponding
Finnish compound nouns or for example the common English class of -able suffixed
adjectives that does not have accurate lexical Finnish translation at all.

In terms of where RBMT is perhaps more usable than NMT, one important factor is
how predictable and systematic the errors are when they appear. For example just
looking at the first page of the top-ranking system in
Finnish-to-English\footnote{\url{http://matrix.statmt.org/matrix/output/1903?score_id=39757}} one can
see the Finnish ``Aika nopeasti saatiin hommat sovittua, Kouki sanoi'' translated into
``Pretty quickly we got the gays agreed, Kouki said.'' whereas the correct translation
is ``We reached a pretty quick agreement, Kouki said.'', the big  problem with the neural
translation is that it is deceptively fluent language but conveys something completely
different, comparing to the rule-based version: ``Kinda swiftly let jobs agreed, Kouki said.'' which
is not fluent at all, but doesn't hallucinate gays there so it may be more usable for
post-editing. For further research in the problems of NMT for real-world use, see for
example~\cite{moorkens2018translators}.

In comparison to neural and statistical systems, the rule-based approach does not
generally fare well as measured with automatic metrics like BLEU, for a human
evaluation refer to~\cite{bojar-EtAl:2019:WMT2}. However, the experiment I describe here is also
not the most actively developed machine translators, rather I use the experiment
to gauge the effects the described workflow has to quality of semi-automatically
generated RBMT, to see how more developed systems fare on the same task you
should also refer to~\cite{hurskainen2017rule,kolachina2015gf}.

\section{Concluding remarks}
\label{sec:conclusions}

In this article I've shown a workflow of \textit{shared task driven
development} for rule-based machine translations, namely the lexicons and rules.
I show that a small effort to update lexical data based on yearly released gold
corpora increases BLEU points and enlarges dictionaries as well as improves
rulesets sizes and qualities by a significant amount. In future I aim to build
more automatisation for the workflow to make it trivially usable with continuous
integration.

The systems are all available as free/libre open-source software under the GNU
GPL licence, and can be downloaded from the internet.

\section*{Acknowledgements}
\label{sec:acknowlegdements}

This work has been written while employed in the \textit{Hamburger Zentrum f\"ur
Sprachkorpora} by CLARIN-D. Thanks to all contributors of the related projects:
omorfi, FinnWordNet, Apertium, and everyone helping with
apertium-fin, apertium-eng and apertium-fin-eng.

\bibliographystyle{unsrt}

\bibliography{acl2019}


\end{document}

