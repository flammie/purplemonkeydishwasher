\documentclass[a4paper,12pt]{article}

\usepackage{fontspec}
\usepackage{xunicode}
\usepackage{xltxtra}

\usepackage{url}
\usepackage{hyperref}

\usepackage[margin=3cm]{geometry}

\usepackage{natbib}

\linespread{1.3}

\setmainfont{Times New Roman}

\title{Omorfi–Newest Computational Morphological Analysis of Finnish}

\iffalse
\author{Tommi A Pirinen}
\fi

\begin{document}

\maketitle

\section{Introduction}

Computational morphological analysis is a central component for most of the
computational applications of linguistic analysis. The morphological analysis
for Finnish language was first described some 30 years ago~\cite{}. In this
article, I show a new system for computational analysis of Finnish language that
has practically been in use and freely available for some years now, but has not
been documented through a scientific publication so far. The aim of this paper
is to examine the differences of our current approach to traditional systems
that are used for the task and to highlight some of the new developments that
are relevant to our system. The new developments that we discuss are mainly
in line with the recent trends in the field of computational linguistics. In the
recent years, there has been sharply rising interest in the statistical methods
for computational linguistics. However, morphologically complex language like
Finnish does not lend itself as easily for statistical treatment, so we try
to bring into focus the modifications to the basic statistical approaches that
we have applied to have them working for Finnish morphological analysis.

In computational linguistics, the beginning of 2000's has been
largely time for statistical language models and engineering. It is commonly
argued that statistical models are not as easily suitable for morphologically
complex languages like Finnish as they are for e.g. English. Our system is
based on the same assumption, and the core of the system is similar rule-based
system as described in earlier research of Finnish analysis. In this article
we show how we have integrated statistical features to traditional rule-based
morphological analyser of Finnish language that has been developed
earlier in University of Helsinki~\cite{pirinen2008}. We furthermore demonstrate
a full-fledged version

Another recent development in the computational language models is the concept
of \emph{maintainability} of these computational systems, e.g.
in~\cite{maxwell}. Specifically we will show how we use the power of
\emph{crowd-sourcing} to keep up with the new words, neologisms and other rare
words missing from dictionary. In particular we study the use of the popular
online dictionary Wiktionary as a source of additional lexical data. The
crowd-sourcing as well as few newer morphological phenomena in the language
have implications to lexicographical structure of the data as well, so we try
to describe in this article some of the newest findings based on the word data
we have added to lexical resources of our systems. While morphology of language
is quite resistant to change, variations such as quantitative consonant
gradation of the bleh stops have required new additions to existing
classifications. Furthermore we have re-analysed the generalisations of old
lexicographical classifications and noted that resulting system is more
favorable for computational systems as well as human classifiers to make
educated guesses when classifying unknown words–as well as verifying
classifications of existing lexical data.

The basic framework of the computational system we present here is largely
unchanged from the one introduced in~\cite{koskenniemi1983twolevel}. 
The finite-state automata remain the state-of-the-art for morphological analysis
of morphologically complex languages to date. The main
technological difference is that we are now using weighted finite-state 
automata~\cite{openfst}, which practically means that we have capability to
express statistics or preference relations in our morphological dictionaries.
In this article we show how we have used existing methods and findings with
our Finnish morphological analyser.

To summarise, this article tries to give an overview of the Finnish analyser,
collecting together the past years of experiments that have gotten into the
working system and contrasting them with other systems that have been or are
in practice use for Finnish computational morphologies.

The paper is organised as follows: In section~\ref{sec:background}, we go
through the basic concepts of Finnish morphology, modern finite-state
technology and the management of lexicographical data that is relatively
central to the new system. 

\section{Backgrounds}

Finnish morphology is well understood and documented, e.g. in \cite{,,} and
most recently in \cite{visk}. This is important to note, since we believe that
the point of computational morphological analyser is primarily to capture
contemporary linguistic knowledge about word-forms as accurately as possible.
Therefore we have made an attempt to follow the descriptions of the newest
research on words in \cite{visk}[chapter 1].

The features of Finnish morphophonology that are directly relevant to this
topic are stem variations, vowel harmony, and otherwise mostly concatenative
morphotactics. These are the only aspects of Finnish morphophonology that
determine the lexicographical classification of Finnish words for means of this
computational implementation. 

There have been numerous implementations of Finnish morphological analysis
along the years. The most popular ones can be dated back to 1980's,
including~\cite{koskenniemi1983twolevel}, which could be seen as the closest
technological relative of our implementation, and among the more influential
works in the field of computational morphology. Other such work along the years
include~\cite{holman1988finnmorf}.  More recent works have showed e.g. a fully
statistical approach for approximating a morphological segmentation of
Finnish~\cite{creutz2005unsupervised}.

In~\cite{ranta2008predictable}, Ranta presents a morphological analysis of
Finnish based on \emph{smart paradigm} system. The treatment of Finnish
lexicography in the smart paradigm system is useful reference for our work,
since it maintains similar differences to contemporary dictionaries and other
analysers as our system. 

\section{Data}

One of the main building blocks for rule based analysis of morphology is
lexicographical data. In order to correctly inflect a word, a root and some
classification to determine inflection is needed. Without classification, it is
only possible to guess the inflectional patterns of the word. For most Finnish
words, guessing a correct pattern is quite possible, but for purposes where
high precision is required, such as spell-checking, a manually verified lexicon
is necessary. Also, the task of classification is only needed once per word,
and does not necessarily require expert skills beyond native-like language
understanding.

The requirement for large amounts of lexicographical data is very apparent in
applications like spell-checking that require knowledge of how good a word-form
really is. For example for compound words---a potentially infinite class of
words in Finnish---a compound that is known to exist and has been attested in
a number of texts is very much a better word than a compound that can be made
up by morpholegal combination of nominals' forms.

The sources for lexicographical data for Finnish are many. The most traditional
is the dictionary maintained by Research institute of languages in Finland. For
this, the lexicographical data has been available under free software licence
since 2007, and it is the original base of our analyser. This data consists of
some 90,000 word-forms. The classification uses 80 numeric classes, which could
freely combine with 14 alphabetic classes for the stem consonant gradation.
This means theoretically 1,120 classes, though not all combinations are possible
or used. On top of that there are few words that have multiple classes, or some
that do not fit into classification and the exceptions are defined in the prose
of the dictionary---all in all, not an ideal classification for computational
treatment. Furthermore, the background of this classification scheme were in
removing some classes from the previous one by making phonological
generalisations, such as grouping all simplifying diphthong stems under one
class, on the other hand, some classes only differ from each other by perceived
popularity of one allomorph---and neither of these classification logics has
been carried out thoroughly and systematically.

The data on official dictionary is relatively conservative and does not update
often, so it has been necessary to seek other sources of lexical data for number
of classes of words that are important in practical applications, like
neologisms, jargon, proper nouns and so forth. The first source of lexical data
we attained from the internet is a free open source user-built database named
Joukahainen\footnote{\url{http://joukahainen}}. Their database uses another
classification scheme, \ldots

Another source of lexical data we are using is crowd-sourcing, in this case the
popular Wiktionary project. Wiktionary is dictionary that is built on Wikipedia
style open for everyone editing. This results that the data quality fluctuates
quite a bit between the lexicographical entries, and it needs to be verified
more carefully. The classification of Wiktionary words varies between the
official dictionary classification to none to plain bogus classes, which happens
because Wiktionary does not restrict anyhow how the data can be inputted; users
can basically write anything in place of the classification as they see fit.

There is a fourth source of data that is commonly used, that is expert
classification at University of Helsinki, done in project basis to collect
new sets of data or to verify the data from above sources. For the needs of
these projects, we have devised a new classification, as well as computational
methods to help the workers to classify and verify the data.

\section{Methods}

The core computational methodology for the implementation of the analyser is
finite-state technology, introduced
in~\cite{koskenniemi1983twolevel,beesley2003finitestate}. On top of that we
have applied recent extensions from the research of finite-state morphology,
such as weighted finite-state methods~\cite{hfst2012}, especially those
that have been  built with morphologically complex languages in mind, such
as compounding-aware weighted finite-state models~\cite{pirinen2009weighted}.

\section{Evaluations}

In this section we present the evaluation of our system as a full-fledged
morphological, and we also evaluate the lexicographical system we use for
the classification of new word-forms. To show how our new lexicographical
classification improves from the baseline of the contemporary dictionary
classification, we show a run of word-forms found in number of resources that
old dictionary deemed non-existing or rare.

\subsection{Dictionary Coverage}

First we measure the naive coverage for the systems where it is possible to
measure naive coverage. This tells how many of the word-forms in the material
are out-of-vocabulary items. 

\begin{table}
    \centering
    \begin{tabular}{|l|r|r|r|}
        \hline
        \bf System: & Ours & FINTWOL & textmorfo$\dagger$ \\
        \hline
        \bf Corpus & & & \\
        \hline
        Gutenberg & 99~\% & 74~\% & 100~\% \\
        Wikipedia & 99~\% &       & 100~\% \\
        Europarl  &       &       & 100~\% \\
        \hline
    \end{tabular}
    \caption{Naive coverages when analysing common corpora
    \label{table:coverage}}
\end{table}


\subsection{Classification Precision}


\subsection{Analysis of Word Forms Outside Dictionary Classification}


\section{Discussion}


\section{Conclusion}



% apalike with underscores???
\bibliographystyle{apalike}
\bibliography{omorfi2013}

\end{document}
% vim: set spell:
