\documentclass{beamer}

\usepackage{fontspec}
\usepackage{polyglossia}

\usepackage{graphicx}
\usepackage{color}
\usepackage{url}

\mode<presentation>
{
    \usetheme{HZSK}
}


\title{Omorfi as a part of an MT pipeline
\scriptsize{in FinMT, Helsinki, 2016-09-12\\
\url{https://github.com/flammie/omorfi/}}}
\author{Tommi A Pirinen \scriptsize \guilsinglleft tommi.antero.pirinen@uni-hamburg.de \guilsinglright }
\institute{HZSK.de, de.CLARIN.eu, etc.}
\date{\today}

\begin{document}

\selectlanguage{english}

\maketitle

\begin{frame}
    \frametitle{Introduction}
    \begin{itemize}
        \item \url{http://flammie.github.io/omorfi}
        \item Omorfi is a lexical database that can be compiled into different
            sorts of rule-based parsers and similar stuff
        \item can be useful for machine translation with: morphological
            analysis/generation, \{de,\}segmentation, tokenisation,
            lemmatisation
    \end{itemize}
\end{frame}


\begin{frame}
    \frametitle{Installation and use}
    \begin{itemize}
        \item For serious use: modern Linux distro, install recent autotools,
            gcc or clang, python3 etc. basics, then
            HFST\footnote{\url{http://hfst.github.io}} with python bindings
        \item \texttt{git clone git@github.com:flammie/omorfi \&\& cd omorfi
            \&\& ./autogen.sh \&\& make \&\& make install}
        \item Scripts starting with \texttt{omorfi-} should now just work
            (tabtab).
        \item e.g. \texttt{\$MOSES/scripts/tokenizer/tokenizer.perl <
            europarl-v8.fi.text | omorfi-factorise.py}
    \end{itemize}
\end{frame}

\begin{frame}
    \frametitle{Example sentence}
    newstestB2016-enfi.ref.fi.sgml docid="aamulehti.fi.22966" seg="1":
    ``Poliisilla on ollut etsintätehtävä 9-vuotiaan pojan karkumatkan takia''
    police has had a search mission due to an escape journey of a 9-year-old boy
\end{frame}

\begin{frame}
    \frametitle{Omorfi can be used to add stuff to word-forms}
    \begin{tabular}{l|llll}
        \bf Surface: &   Poliisilla & on & ollut & etsintätehtävä \\
        \hline
        \bf ``Lemmas'': &  poliisi & olla & olla & etsintätehtävä \\
        \bf UPOS: &   NOUN & AUX & VERB & NOUN \\
        \bf CPOS: &   SG.ADE & ACT.INDV & ACT.PCP.SG & SG.NOM \\
        & & PRES.SG3 \\
        \bf morphs: &  poliisi lla & o n & ol lut & etsintä tehtävä \\
        (and more) \\
        \hline
        \bf Target: & Police & has & had & a rescue mission \\
    \end{tabular}
\end{frame}

\begin{frame}
    \frametitle{more word-forms}
    \begin{tabular}{l|llll}
        \bf Surface: & 9-vuotiaan & pojan & karkumatkan & \alert{takia} \\
        \hline
        \bf ``Lemmas'': & 9-vuotias & poika & karkumatka & takia \\
        \bf UPOS: & ADJ & NOUN & NOUN & ADP \\
        \bf CPOS: & SG.GEN & SG.GEN & SG.GEN & POSTP \\
        \bf morphs: & 9 -vuotiaan & poja n & karku matka n & takia \\
        (and more) \\
        \hline
        \bf Target: & \alert{because of} & 9-year-old & boy's & escape \\
    \end{tabular}
\end{frame}

\begin{frame}
    \frametitle{In moses pipeline}
    \begin{itemize}
        \item Put stuff in factors:
            \texttt{Poliisilla|poliisi|NOUN|SG.ADE|poliisi lla}
        \item Other pre-preprocessing like splitting compounds and morphs:
            \texttt{etsintä @@tehtävä} (a rescue mission)
    \end{itemize}
\end{frame}

\begin{frame}
    \frametitle{In apertium pipeline}
    1. ambiguous parses
    \begin{tabular}{llll}
        Poliisilla & on & ollut & etsintätehtävä \\
        poliisi.n.sg.ade & olla.vaux.pri.p3 & olla.vaux.actv.past.conneg &
        etsintätehtävä.n.sg.nom \\
        & olla.vblex.pri.p3.sg & olla.vaux.actv.pp &
        etsintä.n.sg.nom + tehtävä.n.sg.nom \\
        & & olla.vaux.pass.nut.pl.nom & \\
        & & ... & \\
    \end{tabular}
\end{frame}

\begin{frame}
    \frametitle{In apertium pipeline}
    2. disambiguate
    \begin{tabular}{llll}
        Poliisilla & on & ollut & etsintätehtävä \\
        poliisi.n.sg.ade & olla.vaux.pri.p3 & olla.vblex.actv.pp &
        etsintätehtävä.n.sg.nom \\
    \end{tabular}
\end{frame}

\begin{frame}
    \frametitle{In apertium pipeline}
    3. see the rest in the next presentation!
\end{frame}
\end{document}

