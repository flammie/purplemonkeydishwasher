\documentclass[free]{flammie}

\hyphenation{ana-lyser}
\hyphenation{ana-lysers}
\hyphenation{resul-ting}
\hyphenation{Morpho-logy}
\hyphenation{morpho-lo-gical}
\hyphenation{morpho-phono-lo-gical}
\hyphenation{gramma-tical}
\hyphenation{dia-lect}
\hyphenation{tes-ting}

\usepackage{graphicx}
\usepackage{enumitem}

\title{Language technology for the minority Finnic languages}

\author{Flammie A Pirinen \\
  Divvun --- UiT Norgga\\
   árktala\v{s} universitehta \\
  Tromsø, Norway \\
  \texttt{first.last@uit.no} \\\and
  Trond Trosterud \\
  Giellatekno --- UiT Norgga\\
   árktala\v{s} universitehta \\
  Tromsø, Norway \\
  \texttt{first.last@uit.no} \\\and
  Jack Rueter \\
  Helsingin Yliopisto \\
  Helsinki, Finland \\
  Affiliation / Address line 3 \\
  \texttt{first.last@helsinki.fi} \\}

\begin{document}

\maketitle
\begin{abstract}
This article gives an overview of the state of the art in language technology
    tools for Balto-Finnic minority languages, i.e., Balto-Finnic languages
    other than Estonian and Finnish. For simplicity, we will use the term
    \textit{Finnic} in this article when referring to all members of this
    language branch \textit{except} the Estonian and Finnish literary languages.
    All in all, there are nine standardised languages represented in existing
    language technology infrastructures with keyboards, grammatical language
    models, proofing tools, annotated corpora and (for one of the langauges)
    extensive ICALL programs. This article presents these tools and resources,
    discusses the relation between language models and proofing tool quality, as
    well as the (potential) impact of these tools on the respective language
    communities. The article rounds off with a discussion on prospects for
    future development.
\end{abstract}

\section{Introduction}

In contemporary Uralic language technology, the majority languages of the
countries such as Finnish, Estonian and Hungarian are well researched and
documented, whereas minority languages lack some of the resources.  For example,
in terms of mapping the status of language technology of European languages,
there exist two series of white-papers from the central European research
infrastructures, one by
Springer~\cite{koskenniemi2012finnish,liin2012estonian,simon2012hungarian} and
another by
ELE~\cite{muischnek2022report,linden2022report,jelencsik-matyus2022report}. For
minority languages in the Nordic countries, there are also two such reports
(\cite{moshagen2022report} and~\cite{steingrimsson2024language}).  Two Finnic
languages were covered by the two last reports (Kven and Meänkieli), but, to our
knowledge, no such overviews exist for the Finnic minority languages as a whole.
One of our aims is to fill that gap.

Much of the Finnic language technology has been done within the GiellaLT
infrastructure\footnote{\url{https://giellalt.github.io}, see also
\cite{pirinen-etal-2023-giellalt,moshagen2023giellalt}}, where the present authors
all have been active, but both the Apertium\footnote{\url{https://apertium.org},
\cite{khanna2021recent}} and Neurotõlge\footnote{\url{https://neurotolge.ee},
\cite{yankovskaya2023machine}} machine translation systems have been applied to
Finnic languages as well. In this paper, we give an overview of current and
ongoing work in the field of Finnic language technology.

\section{Background}

This section gives a brief presentation of the languages and thereafter the
technological foundation for the language technology used with them.

\subsection{Languages}

\begin{figure}
    \centering
    \includegraphics[width=1.0\linewidth]{2-Finnic-branch.png}
    \caption{The Finnic Languages \cite{rantanen2022best}}
    \label{map}
\end{figure}

The Finnic language area is shown on the map in Figure~\ref{map}.  The map is
ordered according to linguistic criteria and does not quite correspond to the
written Finnic languages. Subsumed under (1) in the map are also Meänkieli and
Kven (marked as ``Finnish'') on the Swedish and Norwegian side of the border in
Northern Fennoscandinavia, respectively. Within the South Estonian area (8)
there is only one written standard, whereas the Karelian area (2) covers North
Karelian Proper (krl) and Livvi (see~\ref{howmany} below for a discussion).
Outside the present presentation fall the majority languages (Estonian and
Finnish). This leaves us with a linguistic map quite close to the 11 Finnish
language codes, shown in Table~\ref{lgs}.


\begin{table}[ht]
    \begin{tabular}{|l|c|c|c|c|}
\toprule
Language & \bf ISO & \it Glottolog & \it Finnish  \\
\midrule
\bf Meänkieli& fit & torn1244 & meänkieli \\
\bf Kven     & fkv & kven1236 & kveeni  \\
\bf Karelian & krl & kare1335 & karjala  \\
\bf Livvi & olo & livv1243 & livvi  \\
\bf Ludic & lud & ludi1246 & lyydi \\
\bf Veps & vep & veps1250 & vepsä  \\
\bf Ingrian   & izh & ingr1248 & inkeroinen  \\
\bf Votic   & vot &  voti1245 & vatja  \\
\bf Võro   & vro &  sout2679 & vöro  \\
\bf Livonian   & liv & livv1244 & liivi  \\
\bottomrule
\end{tabular}
    \caption{Names and codes for the Finnic minority languages}
    \label{lgs}
\end{table}

All the Finnic minority languages are written in the Latin script, using
orthographic principles much in line with the ones used for Finnish.
Typologically, the language branch is quite homogenous, the languages are mainly
agglutinative with rich case systems for the nominals and tense-mode systems for
the verbs. The size of the case systems ranges from 8
(Livonian,~\cite{Viitso-et-al-2012-livokiel},~\cite{laakso2022livonian}) to 18
(Veps,~\cite{grunthal2022veps}), and most of the languages use possessive suffixes
for all nouns. Most of the languages have consonant gradation and vowel harmony,
whereas Livonian and Veps have neither.

All the Finnic languages are presented in two recent handbooks on Uralic
languages,~\cite{bakro-nagy2022uralic}\footnote{See especially the chapters
on Ingrian~\cite{markus2022ingrian}, Karelian~\cite{sarhimaa2022karelian},
Livonian~\cite{laakso2022livonian}, Seto~\cite{pajusalo2022seto},
Veps~\cite{grunthal2022veps}, Votic~\cite{markus2022votic}.}
and~\cite{abondolo2023uralic}\footnote{Relevant chapters
are~\cite{grunthal2023finnic} on
Finnic and~\cite{pladoetal2022voro} on Võro.}. Kven is presented
in~\cite{soderholm2017kvensk} and Meänkieli in~\cite{pohjanen2022meankieli}.


\subsection{Technologies}

The main technologies used for language modelling in the GiellaLT infrastructure
are \textit{Finite State Morphology}~\cite[FSM]{beesley2003finite},
\textit{Constraint Grammar}~\cite[CG]{karlsson1990constraint}, and
\textit{Two-Level Morphology}~\cite[TWOL]{koskenniemi1983twolevel}.  This means
that morphology and syntax is implemented based on (hand-written) dictionaries
of lemma-stem pairs and on rules governing morphology, morphophonology and
syntax. These dictionaries and rules are then compiled into finite-state
automata for efficient processing.  Contextually determined disambiguation and
higher level syntax rules are written in constraint grammar and processed
programmatically.  The grammatical models are compiled with Helsinki
Finite-State Technology (HFST)~\cite{linden2009hfst} and the constraint grammars
with VISL CG 3~\cite{bick2015cg}, both free and open source products.  HFST is
based on weighted finite-state automata and can contain statistical information
about words and word-forms.  Throughout this article, we use the term
\textit{language model} broadly for any system that can analyse or validate
word-forms and may or may not have statistical information.  The
\textit{grammatical model} is used to point to the rule-based model consisting
of the traditional FSM, CG and TWOL.\@

The source code for the grammatical models is stored on Github as open
source~\footnote{\url{htts://github.com/giellalt/}, see
\url{https://github.com/divvungiellatekno} for a full overview}.  The
applications that can be developed with the language models include
spell-checking and correction, grammatical error correction, computer-assisted
language learning and speech technology applications.

The GiellaLT infrastructure also holds corpora. They are used both for
development and testing of the language models and are presented as annotated
corpora, accessible via dictionaries or for corpus
linguistics\footnote{\url{https://gtweb.uit.no/korp}}. The tools are also used
in collaborative infrastructures, such as the Language Bank of Finland Korp
server \cite{rueter2024testing}.  For minority Uralic languages, the availability of
texts in general is limited, and certain genres might be totally absent. The
variance in ``quality'' in relation to standards is more extensive than what is
available for majority languages that have long established writing systems.

The universal dependencies project~\cite{ud214} contains several Finnic
language datasets: Karelian and Livvi have been built based on GiellaLT
analysers and manual annotation~\cite{pirinen2019building}.

The grammatical models generate paradigms and the corpora present usage
expamples for digital dictionaries for most of the Finnic languages\footnote{The
dictionaries are available at \url{https://sanat.oahpa.no} (Kven, Livvi,
Meänkieli, Veps) and \url{https://sonad.oahpa.no} (Ingrian, Liv, Võro and
Votic), respectively.}. The dictionaries are very useful for language
communities and language learners\footnote{See e.g.~\cite{raisanen2024kvensk} for
an analysis of the role of the Kven dictionary in revitalisation.}.


The underlying technology for rule-based machine translation of the minority
Uralic languages is traditionally based on the Apertium
tools~\cite{khanna2021recent}.  What this means in practice is that we can make
use of the above-mentioned Finite State Morphology for language modelling, and
add to that bilingual (translation) dictionaries, and grammatical rules
concerning about structural re-ordering of words and phrases to implement the
machine translation.

In recent years we have also started to develop speech technologies, while this
is not yet production quality for the languages mentioned in this article, we
are hopeful that the successes shown, for example, for Saami languages
by~\cite{hiovain2023developing} will be transferable to Finnic minority
languages as well.

In recent years within natural language processing, the use of large language
models and neural networks has become more popular and widely replaced
rule-based technologies.  While this works for larger languages with plenty of
available language data covering all textual genres and containing largely
grammatically correct and correctly spelled language, this is more challenging
and produces still less optimal results for minority Uralic languages.  For this
reason, the first step for us is usually to get rule-based tools that promote
language revitalisation and writing normative language, that is, creating more
language data that these large language models need as a prerequisite.

There exists some work done in the Uralic neural network model space, especially
within machine translation,~\cite{yankovskaya2023machine} have released systems
for minority Uralic languages, see Table~\ref{neural} below for a discussion.

\section{Grammar models and standardisation}

When making grammatical language models, one always has to make choices: Some
grammatical forms are included in the model, others are not.  When the models
are turned into proofing tools and similar programs, the normative aspects
become central linguistic questions.  On the other hand, when models are used in
search engines or speech technology, a completely different set of questions
over inclusion of words and word-forms arises.

\subsection{How many standard languages?}\label{howmany}

The international standard ISO 639-3, \textit{Codes for the representation of
names of languages}, lists 9 Finnic languages (c.f Table~\ref{lgs}), in addition
to standard Finnish and Estonian. This has profound consequences in a language
technology setting, as the ISO codes are used by the operating systems as
identification of languages for proofing tools, for example, in text editors,
localisation of user interfaces, speech technology, etc. A language without an
ISO 693-9 code is thus invisible to the computer. Any language community in
search of literacy thus needs an ISO language code.

According to~\cite[93f]{laakso2022graphization}, there are literary languages for
Veps,  Livonian, Meänkieli and Kven as well as a common literary language for
Võro and Seto. Laakso and Skribnik do not mention written languages for Ingrian,
Ludic or Votic but for Karelian they report that there exist ``at least three
different written forms for the diverse dialects of Karelian''.

As seen in Table~\ref{lgs}, there is no separate tag for Seto, and \textbf{vro}
is assigned to \textit{Võro}. Glottolog  the ISO standard, aligns the ISO code
\textbf{vro} with Glottolog code \textbf{sout2679} for South Estonian, this node
then contains 13 subnodes, two of them are \textbf{seto1244} for Seto (itself
with 3 subnodes) and \textbf{voro1243} for Võro. If Laakso and Skribnik are
correct, the ISO code \textbf{vro} may be used for identifying the Seto-Võro
written language.

The most problematic part is Karelian. ISO offers the code quadruplet
\textbf{krl, olo, lud, vep}, for Karelian, Livvi, Ludic and Veps, respectively.
The traditional distribution is shown in Figure~\ref{map:karelian}.

\begin{figure}
    \centering
    \includegraphics[width=1.0\linewidth]{2.2a-Karelian-and-Ludic_traditional.png}
    \caption{Karelian and Ludic around 1900~\cite{rantanen2022best}}
    \label{map:karelian}
\end{figure}

According to the corpus data presented in~\cite{boyko2022open}, Chapter 2.1, the
ISO codes are actually quite appropriate for the situation at hand. They present
4 corpora, for the languages ``Veps, Livvi, Ludian and Karelian proper'', i.e.,
an exact match with the existing language codes. As long as no standard is
claimed for South Karelian ((1b) in Figure~\ref{map:karelian}, the ISO code
inventory provides a good tool for making proofing tools for the Finnic
languages of Russia.


\subsection{Meänkieli and Kven: Many norms in one}

Kven and Meänkieli pose a different type of challenge. Here, the ISO codes, are
unambiguous, the problem is rather that some speakers would like to distinguish
between three standardised varieties for both Kven
(c.f.~\cite{soderholm2017kvensk}).
and perhaps also for Meänkieli (incidentally, Glottolog offers 3 codes for
Meänkieli dialects but none for Kven). Obtaining different ISO language codes
for these would probably be problematic, but so is the situation of missing
support for the (emerging) varieties. So far, the problem has been solved in
different ways for these two langauges. For Meänkieli, the analyser includes all
variant forms on an equal footing, thus allowing for (even inconsistent)
variation in writing. For Kven, there is one grammatical model for all three
dialects. We here show a snippet of code for nouns with short vowel stems for
two of the Kven dialects, Porsanki and Varanki. Both share the same genitive
suffix, but the partitive suffix is set to the archiphoneme \texttt{\^{}A} for
the Varanki dialect and \texttt{\^{}V} for the Porsanki dialect. Then TWOL
rules\footnote{for detailed technical description on TWOL refer
to~\cite{koskenniemi1983twolevel}} will spell out the actual forms of \texttt{\^{}A}
(as \textit{a} when the stem contains \textit{aou} or  \textit{ä} elsewhere) and
\texttt{\^{}V} (as a copy of the preceding vowel). During compilation, we build
one transducer for each dialect, by removing the strings containing the other
dialect tags for each dialect, and thereafter the dialect tag of the dialect
desired (but not the string containing it). The genitive case is common to both
dialects (as is most of the morphology), it receives no dialect tags and is kept
throughout compilation.

\begin{small}
\begin{verbatim}
    LEXICON n_11 ! päivä, syksy, kuva, ...
    ...
    +N+Sg+Gen:^WG%>n # ;
    +N+Sg+Par+Dial/Var:%>^A # ;
    +N+Sg+Par+Dial/Por:%>^V # ;
\end{verbatim}
\end{small}

So far, only the Porsanki dialect has been distributed to language users. Having
all three co-existing in the same computer would not be possible, as they must
be referred to by the same ISO code, so if the need should arise we would have
to ask the users to install only one of them.

\subsection{Data-driven and/or rule-based language technology}

A hot topic in NLP of 2020's is, what all can be done with large language models
and chatbots.  Our approach to NLP is based on traditional rule-based systems,
with expert curated dictionaries and hand-written rules.  For languages we talk
about in this article, it can be easy to point out that for data-driven
approaches we simply do not have enough data (c.f. Sable~\ref{subsec:corpora}
for some statistics), while the methods of using little data improve, the
amounts of data available for Baltic Finnic languages is insufficient for large
language modelling.  Another aspect that one has to keep in mind is the quality
of the data: for machine learning to work, the data needs to be representative:
follow the standards that the chatbot-based AI is supposed to use and contain
ample examples of correct usage in various genres.  With limited data and plenty
of non-standard usage, the large language models will not be usable for spell
and grammar checking and correction, while rule-based approaches can be steered
to prefer and suggest current norms if available.



\section{Resources and evaluations}

In this section we list grammatical models in the GiellaLT infrastructure as
well as corpus resources in GiellaLT and elsewhere.  The statistics shown in
this chapter are valid for the time of writing, since the language models are
developed constantly, the figures will be outdated by the time of publication
already.  For this reason, automated generation of resources and evaluations are
evaluated in the \textit{continuous integration / continuous deployment} (CI/CD)
systems and presented as up-to-date online statistics
\footnote{\url{https://giellalt.github.io/CorpusResources.html}}. The relevant
scripts are available in the github
repositories\footnote{\url{https://github.com/giellalt/giella-core} and
\url{https://github.com/divvun/actions}}.


\subsection{Grammatical models}\label{gm}

Within GiellaLT, there are grammatical models for 9 of the Finnic minority
languages, cf. Table~\ref{models}, which gives an overview of the lexical and
morpho-syntactic descriptions of the language models in our infrastructure..
Only two of them are described in publications
(Meänkieli~\cite{trosterud2020sprakteknologi},
Kven~\cite{reino2017morphological}).

The size of morphosyntactic models can be measured in terms of how many lexemes
they contain and the complexity of the morphophonological system can be
approximated by combining the number of affixes used with the number of
morphophonological alteration rules, covering suprasegmental and
non-concatenative morphology as well as sandhi phenomena).

\begin{table}[htb]
    \centering
    \begin{tabular}{|l|c|r|r|r|}
\toprule
Language & \bf ISO & \bf Stems & \bf Affixes & \bf Rules  \\
\midrule
\bf Ingrian	&	izh	&	2,163	&	2,361	&	45			\\
\bf Karelian	&	krl	&	66,096	&	555	&	1			\\
\bf Kven	&	fkv	&	46,354	&	5,096	&	56			\\
\bf Liv	&	liv	&	15,276	&	6,247	&	68			\\
\bf Livvi	&	olo	&	60,008	&	5,456	&	84			\\
\bf Ludic & lud & --- & --- & --- \\
\bf Meänkieli	&	fit	&	65,872	&	3,436	&	63			\\
\bf Veps	&	vep	&	6,280	&	2,011	&	10			\\
\bf Võro	&	vro	&	36,591	&	8,672	&	156			\\
\bf Votic	&	vot	&	1,030	&	190	&	10			\\
\bottomrule
\end{tabular}
    \caption{Grammatical models in the GiellaLT infrastructure
    (\url{https://giellalt.github.io/LanguageModels.html\#uralic})}
    \label{models}
\end{table}

\subsection{Corpora}\label{subsec:corpora}

We have also curated corpora for some of these languages.  The corpora are used
for the development of the language technology tools: we collect spelling and
grammar errors to test and develop writers tools, we collect the words and word
forms to test the morphological implementations and use the sentences to test
the automatic machine translation, to name a few.  The GiellaLT corpora are
summarised in Table~\ref{tab:corpora}.

There are also corpora for minority Finnic languages outside the Giellalt
infrastructure. MetaShare contains a parallel corpus Võro - Estonian containing
171,252 Võro words as well as a monolingual Võro corpus of 350000 words
(\textit{https://metashare.ut.ee}). There are Bible texts available for Viena
Karelian, Livvi and Veps (\textit{https://www.finugorbib.com}), a parallel Bible
corpus (\cite{pabivus-korp_fi}) and an open corpus containing (in total) 2,66
million words for the same languages (cf.~\cite{boyko2022open} for a
presentation).

\subsection{Evaluation}

Using the corpora, it is possible to measure a naïve \textit{coverage} gives an
impression of how much of real world texts can be successfully processed with
the resulting analyser; a näive coverage is measured as a proportion of surface
tokens that gets \textit{any} analysis at all without considering correctness,
this gives a rough estimate of how well the analyser models the language in the
form that is used in real world texts.  It may be noteworthy to remember that,
in the case of minority languages, real world texts can show a variance of
non-standard forms and orthographies wider than established and standardised
majority languages.  In order to perform more thorough evaluation, we would need
to co-operate with a language expert and develop hand-annotated gold standard
corpora, for this article, that is left for future work.  To get a qualitative
insight on the quality of the analysers (or the data), for example the commonest
words that are not analysed for each languages are:\footnote{Both the source
code for analysers and the corpora can be found at
{\url{https://github.com/giellalt}}, in the repositories \textit{lang-xxx} and
\textit{corpus-xxx}, respectively, where \textit{xxx} is the relevant ISO code.
Compilation is docuented at {\url{https://giellalt.github.io}}. Analysis was run
at Oct 18th 2025.}:

\begin{itemize}[noitemsep,parsep=0pt,partopsep=0pt]
    \item Meänkieli: \textit{oova, och, nytten}
    \item Kven: \textit{kirj., muist, đ}
    \item Livvi: \textit{grigorianskoin, kargavusvuon, kalenduaruan}
    \item Veps: \textit{km, Vellest, nell}
    \item Võro: \textit{q, NOTOC, de}
\end{itemize}


\begin{table}
    \centering
    \begin{tabular}{|l|c|r|r||r|}
\toprule
Language & \bf ISO & \bf ktkn & \bf MiB & \bf Cov  \\
\midrule
\bf Meänkieli	&	fit	& 528 & 12  & 90~\% \\
\bf Kven	&	fkv	    & 1,115 & 21  & 92~\%\\
\bf Livvi	&	olo	    & 242 & 4   & 87~\% \\
\bf Veps	&	vep	    & 859 & 9  & 88~\% \\
\bf Võro	&	vro	    & 265 & 4  & 90~\% \\
\midrule
Finnish & fin & 16,694& 382  & ---\\
\bottomrule
\end{tabular}
    \caption{Corpora in the GiellaLT infrastructure. Finnish is listed for its
    relevance to machine translation.\@ \textbf{ktkn} = thousand tokens,
    \textbf{MiB} = million bytes, \textbf{Cov} = coverage, or percentage re
    cognised by the analyser.}
    \label{tab:corpora}
\end{table}

\section{Practical tools}

Several language technology tools and softwares are implemented based on the
morphological analysers and text collection.  These tools are developed to
support the language community, language revitalisation, standardisation, etc.
We provide here experimental results of using these analysers in the context of
these applications and corpora.


\subsection{Keyboards and proofing tools}

Keyboard drivers and tools for checking written language and correcting mistakes
are crucial for literacy development in the digital era.  Each literary language
needs its own keyboard layout, for several reasons. The Finnic languages have
different sets of letters in addition to the basic a-z set, typically around 6
additional ones, but ranging from 3 (Meänkieli) to 21 (Livonian). The optimal
keyboard should be a compromise between keyboard tradition and placement of
letters according to their frequency in running text. Then the keyboard users
will expect non-letter symbols to be in the same positions as they are on the
majority language keyboard. Kven and Meänkieli share the same alphabet (except
for the Kven \texttt{đ}), but in addition, symbols such as \texttt{@, ', §, \$,
€} are placed (and engraved!) on different positions on Norwegian and Swedish
keyboards, and the users of each minority language will expect these symbols to
be in the same positions as they hold on the majority language keyboard.
Finally, in Windows, the language of third-party proofing tools are identified
by sharing ISO code with a keyboard driver. The same goes for mobile phones,
where language support is always linked to the keyboard language.


The GiellaLT infrastructure contains a pipeline for easily setting up keyboard
layouts for all computer and mobile phone operative systems, as well as
keyboards for 8 of the Finnic minority languages~\footnote{For an overview and
links to the keyboards, see
\textit{https://giellalt.github.io/KeyboardLayouts.html\#uralic-languages}}.

Proofing tools include spell-checking and correction as well as grammatical
error correction. The GiellaLT infrastructure is set up so that even a
grammatical model can be turned into a spellchecker. The availability of
proofing tools is thus obviously dependent upon the quality of the language
model. The language models (see Table~\ref{models}) are classified according to
a 4-grade evaluation scale\footnote{For a definition of the various grades, see
\textit{https://giellalt.github.io/MaturityClassification.html}}. In addition,
the spellchecker is dependent upon a suggestion mechanism as well as a text
corpus in order to give precedence to more common words when correcting. A
minimal suggestion mechanism contains approximately 50 rules (one for each
letter or symbol to be suggested). Even a well-developed spellchecker in the
GiellaLT does not contain more than appr. 300 suggestion rules.
Table~\ref{tools} gives an overview of status for the Finnic minority languages.

\begin{table}[htb]
    \centering
    \begin{tabular}{|l|c|r|r|r|c|}
\toprule
Language & \bf ISO & \bf Keyb & \bf Spell & \bf Sugg & \bf W  \\
\midrule
\bf Ingrian	&	izh	&	yes	&	Beta	&	56	& ---		\\
\bf Karelian	&	krl	&	yes	&	Alpha	&	89	& ---	\\
\bf Kven	&	fkv	&	yes	&	Prod.	&	301 & yes 			\\
\bf Liv	&	liv	&	yes	&	Alpha	&	109	& ---		\\
\bf Livvi	&	olo	&	yes	&	Beta	&	88	& ---		\\
\bf Ludic & lud & --- & --- & --- & ---\\
\bf Meänkieli	&	fit	&	yes	&	Beta	&	220	& yes		\\
\bf Veps	&	vep	&	---	&	Alpha	&	68	& ---		\\
\bf Võro	&	vro	&	yes	&	Beta	&	62		& ---	\\
\bf Votic	&	vot	&	yes	&	---	&	---	& ---		\\
\bottomrule
\end{tabular}
    \caption{Proofing tools in the GiellaLT infrastructure.\@ \textbf{Spell} =
    quality level, \textbf{Sugg} = number of suggestion rules, \textbf{W} =
    corpus for weighting of suggestions}
    \label{tools}
\end{table}


\subsection{Rule-based machine translation}

There are 6 Finnic language pairs within the Apertium~\cite{khanna2021recent}
rule-based machine translation system, cf.  Table~\ref{tab:mt}. Each language
pair contains bilingual dictionaries, grammatical language models for analysis
of L1 and generation of L2 as well as grammars for lexical selection and
grammatical differences. As can be seen from the number of lexical entries, the
language pairs range from usable machine translators to early stage projects.


\begin{table}[]
    \centering
    \begin{tabular}{|l|r|}
    \toprule
    Pair     & \bf  Entries  \\
    \midrule
    \bf Finnish---Livvi & 30,212 \\
    \bf Karelian---Livvi & 6,419 \\
    \bf Finnish---Kven & 4,624 \\
    \bf Karelian---Finnish & 2,297 \\
    \midrule
    \bf Vorõ---Estonian & 161 \\
    \bf Livonian---Finnish & 37 \\
    \bottomrule
    \end{tabular}
    \caption{Machine translation models\label{tab:mt}}
\end{table}

\subsection{Neural machine translation}\label{neural}

The neural machine translation project \textit{Neurotõlge}
(\textit{neurotolge.ee}, see~\cite{yankovskaya2023machine}) offers machine
translation between (among other Uralic languages) the Finnic minority languages
Livvi Karelian, Viena Karelian, Lude, Veps, Livonian and Võro and the majority
languages Finnish, Swedish, Norwegian Bokmål and Russian. The monolingual
corpora presented in~\cite[765]{yankovskaya2023machine} range from 5,000
(Ludic) to 115,300 and 162,000 (Veps and Võro) sentences. The amount of parallel
sentences for the languages in Russia with Russian are 10,000 – 27,000, with the
Bible dominating for all languages except Ludic.


Compared to their result for Finnish to Inari Saami and Norwegian to South Saami
(which boast the quite good BLEU scores of 67.34 and 60.79, respectively), their
results for the Finnic languages (op.cit. p. 768) are far worse (BLEU 24.17 for
Estonian to Livonian and 30.63 for Estonian to Võro, the latter even worse than
their previous result of 34.11). As shown by~\cite{yankovskaya2023machine}, the
main reason for this is the paucity of text, and the lack of balance for the
parallel text, for the Finnic languages.

There are some existing critical evaluations of Neurotõlge for Sámi languages,
c.f.~\cite{wiechetek-etal-2024-ethical,wiechetek2023manual}, but these
evaluations concentrate upon key semantic and grammatical elements of the
translated texts rather than the overall closeness between translation and
reference, as~\cite{yankovskaya2023machine} do.

\section{Possibilities and perspectives}


There are grammatical models for most Finnic minority languages, they show a
coverage for running text on around or slightly 90~\% (cf.
Table~\ref{tab:corpora}). This is typical result achieved by rewriting formal
grammars as grammar models. Grammars are seldom comprehensive, they typically
sketch main patterns and obvious exceptions. In order to go the time-consuming
work of getting a coverage of, say, 98~\%, one has to include native speakers
with knowledge of the norm in the team, so that they can add the description not
included in the grammars. It is thus important that language researchers,
teachers and learners are included in the process.

One way that the teachers and learners might help, is to simply provide
paradigmatic information on word inflection. Providing simple information on a
single word \textit{häkki+N+Sg+Ade: häkil}, for example, provides the coder with
information on gradation, and an adjacent plural form \textit{häkki+N+Pl+Ade:
häkkilöil}. These bits of information can be generated in a class environment
where each student is given nouns, verbs or adjectives to describe in paradigms.
The teacher checks to see that the forms are correct and the paradigmatic
information is added to the infrastructure testing.

The GiellaLT infrastructure provides two different kinds of testing: One is
impressionistic testing: Tools that generate parts of the model for the
developer to inspect (e.g. generating all forms of a certain case). Another type
is regression testing. Here, the linguist has set up for example model paradigms
for parts of the morphology, and the model is tested continuously in order to
ensure that it does not get worse.

There are test paradigms for the grammatical models of the Finnich minority
languages to a various degree. Table \ref{paradigm} gives an overview of
paradigm cells in the testing setup for the different languages. The figures
might provide us with a picture of the time allocated to developing the
different models. One could, of course, also add language-form information to
the paradigmatic information, which could help solve problems in Veps, for
example, where the Veps magazine
\textit{Kodima}\footnote{\url{https://omamedia.ru/fi/publication/kodima}} and
the Veps edition of \textit{Wikipedia}\footnote{\url{https://vep.wikipedia.org}}
are written in two different orthographies.


\begin{table}[]
    \centering
    \begin{tabular}{|l|r|}
    \toprule
    Language     & \bf  Paradigm info  \\
    \midrule
    \bf Kven & 10,557 \\
    \bf Livonian & 5,693 \\
    \bf Livvi & 3,538 \\
    \bf Meänkieli & 1,526 \\
    \bf Veps & 392 \\
    \bf Võro & 4,023 \\
    \bottomrule
    \end{tabular}
    \caption{Paradigm info\label{paradigm}}
\end{table}


There is always a continuum of dialects and languages and standards within these
minority languages, one benefit of rule-based approaches is that they offer good
control over the variation: It is possible to implement morphophonological rules
and lexical analyses that concern specific variants.  When this language
technology is combined with a tool like spell-checking and correction, it is a
powerful tool for language normativisation and support of writing culture.
Experience with Kven has shown that the same lexica and morphological tagging
structures can  be used for describing language variants by river valley.
Applied to Karelian languages, this might allow us to share mutual word stems,
on the one hand, but distinguish morphological branches on the other. When it
comes to sharing mutual lexica, it should be noted that the shared lexica are
set off as their own groups. In work with Saami languages, proper noun lexica
are shared. Even here, however, not all proper nouns can be shared. In work with
the Permyak-Komi and Zyrian-Komi, additional sharing of lexica has been included
for 100\% matches in Russian loan words. For the Karelian languages using shared
lexica is dependent on the use of parallel phonematic writing practices.

For future work, there is a lot that can be done in curating more lexical data
and corpora for these languages.  There is also a potential of developing speech
technology applications based on the example of existing systems in Sámi
languages.  All of this requires collaboration, of course, between language
communities and computational linguists.  An important and ever more relevant
issue in collaboration of language communities and computational linguists is
ethical issues related to ownership of the language data and language itself,
there has been a lot of research on this topic by us and others and we want to
point towards~\cite{wiechetek-etal-2024-ethical,wiechetek2022unmasking} for
further references.

\section{Conclusion}

In this article, we have summarised the state of the art in minority Finnic
language technology.  We have shown that there exist some resources and have
compared them to related languages to highlight the potential future
possibilities these languages already have available.

The main part of the language technology work on Finnic so far has been
concentrated on language models and proofing tools. For 5 of the 9 languages, we
have developed grammatical models showing a coverage on running text extending
85 \% (for three of them, 90 \%).

The situation for available corpora is rather limited. Only for Kven and
Meänkieli are there text collections available other than text from (Incubator)
Wikipedias. To what extent the content of the corpora follow established
standards is unclear. The corpora referred to here do not include all published
text, but it is clear that the basis for data-driven language technology is
shaky. In this perspective, we note on the positive side that despite this,
there is neural-based MT for 5 of the languages presented here.

\bibliographystyle{unsrt}
\bibliography{iwclul2025}

\end{document}
