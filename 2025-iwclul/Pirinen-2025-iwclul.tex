\documentclass[free]{flammie}

\usepackage{tcolorbox}

\title{Can advances in NLP lead to worse results for Uralic languages and how
can we fight back? Experiences from the world of automatic spell-checking and
correction for Finnish}

\author{Flammie A Pirinen \\
  Divvun \\
  UiT---Norgga árktalaš universitehta \\
  Tromsø, Norway \\
  \texttt{first.last@uit.no} \\}

\begin{document}

\maketitle
\begin{abstract}
Spell-checking and correction is a ubiquitous application within text input in
    modern technology, and in some ways or another, if you type texts on a
    keyboard or a mobile phone, there will probably be an underlying spelling
    corrector running.  The spell checkers have been around for decades,
    initially based on dictionaries and grammar rules, nowadays increasingly
    based on statistical data or large language models.  In recent years,
    however, there has been a growing concern about the quality of these modern
    spell-checkers.  In this article, we show that the spell-checkers for
    Finnish have gotten significantly worse in their modern implementations
    compared to their traditional knowledge-driven versions.  We propose that
    this can have critical consequences for the quality of texts produced, as
    well as literacy overall.  We furthermore speculate if it would be possible
    to get spell-checking and correction back on track for Uralic languages in
    modern systems.
\end{abstract}


\section{Introduction}

\textit{Spell-checking and correction} is a quintessential \textit{natural
language processing} (NLP) task.  It has been part of the NLP ecosystem for
decades now, from the very early days of processing texts with computers.  It
has become so ubiquitous that it exists in most text editing products without
users even paying much attention to it, and it has been viewed as somewhat of a
solved problem within the scientific study for the last few decades.  While
there has not been much focus on spell-checking and correction in recent years,
we as linguists have noticed something quite problematic with the contemporary
systems.  Namely, we have noticed a drop in quality of writing in our native
languages on the Internet discussion forums.  This is increasingly shown in the
frustrations by the native writers: ``autocorrect wrote it, and it is too hard
to fix it by hand''.  On this basis, we set out to study, if the contemporary
spell-checking and correction systems have become worse in our language.  Our
hypothesis is, that modern autocorrecting spell-checking and correction systems
are based on data-driven methods and lately large language models, which may
work adequately with English---not the least because over 90~\% of the training
data is in
English\footnote{c.f.~e.g.~\url{https://github.com/openai/gpt-3/blob/master/dataset_statistics/languages_by_word_count.csv}}---but
which actually fail to recognise words of non-English languages with potentially
more complicated morphology.

Our \textit{research question} in this short paper is, are data-driven and large
language model based spelling checkers and correctors worse than traditional
knowledge-based ones?  Our initial hypothesis, based on everyday observations,
is that spell-checking tools have gotten significantly worse in the past few
decades, in pace with the introduction of data-driven and `AI'-driven models.
We study the spell-checking and correction results by three popular systems for
\textit{Finnish}.


\section{Background}

There is a long history of spell-checking and correction in language technology,
starting from early days of SPELL, a spell-checker based on a dictionary or a
word-list and few simple rules to modify suffixes.\@ \cite{earnest1976look}
places initial use of their spelling correction to 1969.  This system's
descendants---ispell, aspell and hunspell and so forth---have been in use in
some of the most popular browsers and office suites up to the 2000s.  There have
been several comprehensive scientific surveys of spell-checking and correction,
for example~\cite{kukich1992techniques}.  As of last few decades, office suites
have started using built-in, closed-source, statistical spell-checkers and more
recently, overarching AI assistants which also do spell-checking, a similar
development is happening in browsers, mobile phones and operating systems.  One
of the most influential initial works on data-driven spell-checking and
correction is Google's Norvig's spelling corrector~\cite{norvig2009natural}.
Technical details of the most modern commercial spelling correctors are not
openly documented as far as we know.

When researching on existing studies on LLM-based spelling correctors and
especially comparisons between LLM and traditional methods, one topic that
dominates the results is spell-checking and correction for students / L2 / EFL
users of English~\cite{jaashan2025using,gayed2022exploring}.  A gap in research
we hope to address with this experiment and its followups, is, therefore, a
comparative study, and for L1 users, in non-English.

In our work, we build NLP tools and software, mainly targeting less-resourced,
minority and Indigenous languages, but we also create tools that are language
agnostic and usable for all.  Spell-checking and correction and related software
is a key tool for digital language survival for minority languages, and it is
also increasingly important for ever larger and more majority languages, apart
from the largest few.  This has come to be also, because the contemporary
data-driven language technology is strongly based on big data, that has been
written by humans in correctly spelled and grammatical language.

In this experiment, we have chosen to use Finnish.  Finnish is a national,
majority language in Finland, it is not low-resource by any stretch of the
imagination.  We estimate it is likely to be in the top 50 of the most
resourceful languages in the world.  While we are more interested in
low-resource languages and settings, having moderately resourced Uralic language
works well for our initial experimentation.  We have existing resources such as
corpora and established automatic spell-checkers, which we might not find on
lower resourced languages.  There is also existing research on state of the
Finnish NLP~\cite{hamalainen2021current} including spell-checking
and correction.  Furthermore, Finnish is not an Indo-European language, and has
a slightly more complicated morphology than most IE languages, which makes it
more comparable towards many of the minority and under-resourced languages
relevant to our research.  Finally, we have native speakers of Finnish, which in
our opinion is critical in doing meaningful qualitative studies on language
technology software; without linguistic error analysis and human interpretation
of the results, it is impossible to make meaningful explanation of how useful or
harmful the underlying system is for actual end users.

Linguistically, Finnish is a Uralic language with some 5 million speakers,
mainly in Finland.  Morphologically, Finnish has what we call slightly more
complex morphology, in terms of what matters for spell-checking and correction
this means that there are on average thousands of word-forms per word, instead
of around 5 like in English or few dozens like in most IE languages.  Finnish
also has productive compounding, which means you can put two word-forms together
without a space to create a new word, that does not necessarily exist in the
dictionary, on the fly.  Finnish has had a literary culture for several hundreds
of years and has a strong nationally backed standardisation body, is a primary
language in schools and in public.  It is also a majority language for several
Indigenous and minority languages, which is one of the motivations for us to
work on it as well.

\section{Methods and experimental setup}

In this work, we compare and contrast spell-checking and correction from the
end-user point of view.  We test three different systems: one based on
knowledge-driven paradigm and two based on data-driven approach.  The
knowledge-driven spell-checker is an open source, rule-based product, whereas
the data-driven products we experiment with are commercial and closed-source.

The rule-based spell-checking and correction is a freely available open source
implementation of Finnish spell-checking found on the GitHub called
omorfi\footnote{\url{https://github.com/flammie/omorfi/}}, their implementation
is based on finite-state spell-checking~\cite{pirinen2014state}.  This
spell-checker uses an underlying dictionary and morphological rules to recognise
valid word-forms without context, and uses finite-state error modelling
technology to create suggestions for corrections.

We use Google's spell-checking and correction as a black-box, we have not found
technical documentation detailing it, but we estimate that it is at least in
part based on statistical methods and or large language models based on the
company's recent focuses and public statements.\footnote{Searching online leads
to old posts like:
\url{https://workspace.google.com/blog/productivity-collaboration/everyday-ai-beyond-spell-check-how-google-docs-is-smart-enough-to-correct-grammar},
but we cannot know for sure if this kind of information is up-to-date.} The
function in Google Docs interface is found under spelling and grammar checking,
we have crossed off grammar-checking and only included spelling.

For a product that is certainly using large language models, we test
ChatGPT~\cite{chatgpt4o}, and we use it as a black box with the version
available to us via our university account.  We use the web-based ChatGPT user
interface to query spelling corrections from the language model via its natural
language user interface in the same way an average end user likely would.

The experiments have been performed in May 2025, some details are included in
the Appendix~\ref{sec:appendix}, but since they are closed commercial products,
we do not expect to be able to have reproducible results with them in any case.


\section{Data}

To test the spell-checking and correction we have used a Finnish translation of
\textit{Alice's Adventures in Wonderland} from Project
Gutenberg\footnote{\url{https://www.gutenberg.org/ebooks/46569} for
reproducibility we have the version we used in our GitHub
at~\url{https://github.com/flammie/purplemonkeydishwasher/tree/master/2025-iwclul/reprodata};
this is also for access from within Germany, Italy or other countries with
extreme copyright restrictions where Project Gutenberg may not be available.}
which is in public domain.  This book is a fantasy novel aimed for children, and
contains creative use of language which makes it very suitable for natural
language processing testing.  The translation has been made in early 20th
century which matches the most modern standard written Finnish with almost no
deviations.  In general, proofreading at the times of the publication was highly
valued and efficient, and we expect the manuscript to be mostly error-free
barring potential mistakes in gutenberg's encoding.  The non-word errors we have
found and verified are listed in the error analysis
Section~\ref{subsec:error-analysis}.  The book consists of 18,861
space-separated tokens (after removing project  Gutenberg's licence, preamble
and postamble).

\section{Results}\label{sec:results}

To measure the spelling error correctors, we went through all the words that
were flagged as spelling errors, and categorised them into two categories:
\textit{false positives}, where a correctly spelled word was flagged as
incorrect, and \textit{true positives}, where the flagged word did contain a
spelling error.  This was done by a native speaker who had access to the error
in context, even though the decision was made solely on whether the word is a
valid word in the language at all or not (i.e.\ it can also be decided without
context as traditional non-word spelling corrector does). The breakdown of
errors and flaggings is shown in the Table~\ref{table:errors}, we also provide a
calculations of precision, recall and $F_{0.5}$, the parametre 0.5 for $\beta$
is selected since our starting point is that false positives are more critical
problem in spell-checking than false negatives.

\begin{table}[]
    \centering
    \begin{tabular}{|l|r|r|r|}
    \toprule
    Error \textbackslash{} System & \bf Google & \bf ChatGPT$^*$ & \bf omorfi   \\
    \midrule
    \bf False Positive & 565 & 75 & 59 \\
    \bf False Negative & 22 & 59 & 20 \\
    \bf True Positive & 41 & 4 & 43 \\
    \midrule
    \bf Precision & 0.07 & 0.05 & 0.42 \\
    \bf Recall & 0.65 & 0.06 & 0.68 \\
    \bf F-Score ($F_{0.5}$) & 0.08 & 0.05 & \bf 0.46 \\
    \bottomrule
    \end{tabular}
    \caption{Quantitative evaluation of error types by systems. $^*$ ChatGPT
    results are not proportional due to reasons explained in the chapter. For
    main findings, read the qualitative error analysis.
    \label{table:errors}}
\end{table}

\subsection{Error analysis}\label{subsec:error-analysis}

We have further categorised the errors flagged by the spelling correctors into
error types, based on linguistic insight and world knowledge.  We hypothesise
this will help give an impression of the impact these errors have on the user
experience, this impact is further discussed in the section~\ref{sec:discussion}
below.  The summary of errors is given in table~\ref{tab:my_label}, some of the
error classes are not mutually exclusive and the numbers in the rows do not add
up to the total.

One of the largest groups of false positives in all systems' data is compound
words, particularly the types that do not appear in dictionary: for Google's
spell-checking \textbf{compound} nouns like \textit{pääkallonkuva} (picture of a
skull) or \textit{kyynellammikko} (lake of tears) were consistently underlined,
for ChatGPT we have e.g.\ \textit{herttakuningatar} (queen of hearts) and for
omorfi we saw compound adverbs like \textit{tuulennopeasti} (in wind's speed).
From \textbf{derivational} forms, all systems stumbled on \textit{ruukkusen}
(little jar's$\sim$jarful$^?$).  Some of the false positives found by Google can
also be described as being part of complex morphology that is a bit half-ways
between \textit{inflectional} and derivational morphology, for example
\textit{myöhästynkin} (I will be late too), \textit{elämäniässään} (in their
lifetime), \textit{vaikeroidessaan} (while they were whining), that is, enclitic
particles, possessive suffixes and non-finite verb forms in combinations
that---in all likelihood have not been many times in sufficiently large
corpora---throw Google's spelling checker off the track.  The commonality for
errors in this category is that there are at least two distinct inflectional
suffixes in the word-form.  Perhaps surprisingly, also \textbf{proper nouns}
show up as false positives, even though traditionally maybe it has been common
practice to ignore titlecased words: Google finds \textit{Ellakaan} (Ella too)
and \textit{Vilhelmiä} (of Vilhelm) errors, and omorfi finds \textit{Morcar} and
\textit{Stigand}.  The classes as laid out in the table~\ref{tab:my_label} are
not mutually exclusive, i.e.\ a \textbf{compound} form can also have a
\textbf{derivation} and a \textbf{proper noun} can have a \textbf{inflectional}
possessive suffix, in these cases we have simply counted the error in both
classes.  To illustrate the overlapping between categories, for example
\textit{Irvikissakaan} (Cheshire cat neither, lit.\ grinning-y cat) is a proper
noun compound with inflectional ending.  There are handful of words that do not
seem to fall into any categories; for omorfi we can simply note they are missing
from the dictionary, e.g.\ \textit{satakaunoja} (an old word for some flower) or
\textit{siekailuun} (into scrupulousness) whereas with data-driven models we can
assume the words themselves are so rare that they do not show up enough in the
training materials, e.g.\ \textit{pulppusivat} (bubbled up) or
\textit{pulikoinut} (drudged about), but there are some that are even harder to
diagnose, such as \textit{nurmen} (grass') and \textit{vai} (or).


The true positives in the text fall into following categories: unexpected
hyphenation caused by creative language use (recreation of typeset poems:
\textit{tar-kemmin} (tarkemmin), \textit{päi-villä} (päivillä), and
\textit{veruk-keella} (verukkeella)), lengthening of letters for emphasis
(\textit{li-iemi} (liemi), \textit{ku-ulta} (kulta) and \textit{ihana-ainen}
(ihanainen)), foreign words (\textit{Oú}, \textit{est}, and \textit{chatte}),
dialectal, informal or poetic forms (\textit{teälhän} (täällähän),
\textit{käshän} (käsihän), \textit{näkkyy} (näkyy), \textit{käs} (käsi),
\textit{sittennii} (sitenkin), \textit{pyssyy} (pyssyä), \textit{ruppee}
(rupeaa), \textit{pentus} (pentusi), \textit{juur} (juuri), \textit{loitoll'}
(loitolla), \textit{täss'} (tässä), \textit{kuus} (kuusi), \textit{tavaraks}
(tavaraksi), \textit{niill'} (niillä), and \textit{tuoss}), compounding mistakes
(\textit{mitenpäin} (miten päin), \textit{missäpäin} (missä päin),
\textit{käsikädessä} (käsi kädessä), \textit{sukkajalassa} (sukka jalassa),
\textit{ranskankieltä}, \textit{tipo (tiessään)} (tipotiessään, a non-word error
since tipo by itself is not a dictionary word but a reduplicative form), which
old standard may have allowed), old forms (\textit{sebraa} (seepraa),
\textit{merikilpiö} ($^?$merikilpikonna) again permissible by older standards)
onomatopoeia (\textit{liuskis}, \textit{läyskis}) and two typoes
\textit{antipatiioiksi} (antipatioiksi) and \textit{purstölleni} (pyrstölleni).
We consider all of these nonwords (and eventually true positives) since it is
expected for a typical spell-checker to flag them, even though not all of these
need to be fixed in context of this book.

\begin{table}[]
    \centering
    \begin{tabular}{|l|r|r|r|}
    \toprule
    Error \textbackslash{} System & \bf Google & \bf ChatGPT & \bf omorfi \\
    \midrule
    \bf Compound & 169 & 38 & 18 \\
    \bf Derivation & 21 & 9 & 7 \\
    \bf Inflection & 211 & 6 & 4 \\
    \bf Proper noun & 12 & 0 & 8 \\
    \bf Other & 171 & 24 & 32 \\
    \midrule
    Total & 611 & 70$^\star$ & 57 \\
    \bottomrule
    \end{tabular}
    \caption{Error analysis of false positives in Alice in Wonderland by three
    systems. Classes are not mutually exclusive and may not add up to totals per
    column.  $^\star$ChatGPT started to give empty answers and repeat from the
    beginning after 70 spelling errors.  \label{tab:my_label}}
\end{table}

\section{Discussion}\label{sec:discussion}


While we expected to find some false positives from all the methods, we were
quite surprised indeed to discover how many false positives Google's spelling
error correction flags: over 600 errors in a book of 70 pages means that you see
several wrong red squiggly lines on every page.  This would have been
unacceptable and catastrophical for an office suite in the 1990s, it is alarming
that this is not the case any more.  The fact that this is given to end users
without warnings is starting to be borderline ethically questionable, it has a
real possibility to be destructive to language and culture, as many of the false
positives concern morphologically complexer forms will contribute to make the
language poorer, as language learners and less confident writers will surely
follow the advice of spelling correction program.

ChatGPT's spell-checking is interesting since, despite the fact that we
specifically asked it to only include non-words, kept including real-word
errors.  ChatGPT also includes a helpful explanation for each spelling error it
discovers, this is the opposite of Google doc's system which only provides a
single correction suggestion without any background.  Unfortunately, the
explanation often ends up being nonsensical, for example:

\begin{tcolorbox}
 [colback=white!100,colframe=purple!75!black,width=.48\textwidth,
 righttitle=0.5cm,subtitle style={boxrule=0.4pt,
 colback=yellow!50!purple!25!white},title=ChatGPT]
``torkuksissa - This word does not exist in Finnish. Likely a typo for
    "torkuksissa" (a colloquial form of "torkuksissa").''
\end{tcolorbox}

it reminds us in form the kind of reasonable advice you would get from a helpful
grammar corrector, but content is absolutely mind-boggling and in fact
gas-lighting.

The rule-based spell-checkers also only give very limited feedback to the
end-user, a squiggly red underline to communicate that the word is not in the
dictionary and a list of most common words within a few mistaken keystrokes
away.  Sometimes rule-based spell-checkers are used as a part of a grammatical
error correction system where the grammar-checker can provide context, but it is
typically a very mechanical and limited explanation.  Perhaps an ideal hybrid
system could be to harness ChatGPT's power to create user-friendly descriptions
in addition to rule-based knowledge of actual dictionary and grammar, in style
of this actual example from ChatGPT:\@

\begin{tcolorbox}
 [colback=white!100,colframe=purple!75!black,width=.48\textwidth,
 righttitle=0.5cm,subtitle style={boxrule=0.4pt,
 colback=yellow!50!purple!25!white},title=ChatGPT]
``herttuatar - While valid, it is an older term (archaic) for "duchess."''
\end{tcolorbox}

In this case, ChatGPT had flagged a common word as archaic, but it still gives
the end user information based on which they can more confidently ignore the
suggestion and not left feeling confused or annoyed.  Certainly one could argue
that if it was a modern text about Finnish society and not a translated text of
older times, there would be much less talk about duchesses.

The correction mechanism in Google Docs only gives out one suggestion for
corrections, this leads to many cases where it often ends up actually suggesting
the mistake that users commonly make, exactly the opposite of what we would want
from a spelling corrector. This happens for example for replacing forms of word
\textit{koettaa} (attempt) to word \textit{koittaa} (dawn, verb of sun/morning),
a very common mistake that beginner writers make. It also suggests to split
compound words, and on one occasion it wants to replace \textit{ja pani} (and
put) with \textit{japani} (Japanese).

We are concerned that the lowered quality of spell-checking that is included in
all of our devices and office suites ultimately contributes to lower quality of
texts and literacy, and while the effect is already noticeable for majority
languages like Finnish, the effect will be even greater for less resourced, more
minoritised and Indigenous languages.  Some experts have speculated that the
aggressive push for AI-based writing aids into both office suites and also in
the mobile phone platforms will eventually lead into removal of traditional and
alternative spell-checkers in these contexts; if this happens with the
spell-checkers such as current spell-checker of Google Docs, it will spell a
disaster for Finnish language literacy.

\section{Conclusion}

In this article, we have shown through experimental means that data-driven
spell-checking and correction is much worse for Finnish language than the
traditional rule-based approaches.  Nevertheless, the main systems provided for
spell-checking and correction in many contemporary contexts are using this kind
of spelling correctors for Finnish, without any easy way to change them.

\section*{Limitations}

In this article, we have performed an experiment for one language and one book,
based on limitations of time and human resources: judging and manually analysing
spelling error corrections requires full read-through of the whole text by a
person with native-like language skills who has been trained in proofreading.
There is ample anecdotal evidence that spell-checkers underperform for other
Uralic and minority languages that can be discovered by simple search into
language learning communities in discussion forums like reddit.  More research
on other languages is needed, and we hope our work gives inspiration for other
researchers.

The experiments on large language models have been made on commercial systems,
which makes reproducibility virtually impossible.  Furthermore the version of
ChatGPT we had an access to did not manage to error check the whole text
correctly, for future revisions we will try to find an alternative that can be
more functional; anyways this highlights the problems that average end-user will
face trying to spell-check their texts the way that is available to them.
Training and fine-tuning our own model would not have been a realistic
evaluation setup for the purposes of this article.

\section*{Ethics}

The experiments and analysis have been made by fully paid colleagues, no
underpaid crowd-workers have been hired for this experiment.  The LLMs used in
the experiment waste unethically large amounts of energy and water, while we
have tried to minimise the wastage, our aim for this article is to curb
unnecessary overuse of LLM-based systems through which we hope to achieve a net
positive.

\bibliographystyle{unsrt}
\bibliography{iwclul2025}

\appendix

\section{Versions and parametres}
\label{sec:appendix}

The ChatGPT we use identifies itself as ChatGPT-4o.  ChatGPT prompt is in
figure~\ref{chatgpt-prompt}.\footnote{\url{https:/chat.uit.no}$^\star$, UiT's
safe version of ChatGPT c.f.
\url{https://en.uit.no/om/kunstigintelligens\#collapse_829866}} Omorfi is version
0.9.11\footnote{\url{https://github.com/flammie/omorfi/releases/tag/v0.9.11}}
Google Docs does not seem to have version identification available in the usual
places, we used in 2025.\footnote{\url{https://docs.google.com/}}

\begin{figure}
\begin{tcolorbox}
 [colback=white!100,colframe=green!75!black,width=.48\textwidth,
 righttitle=0.5cm,subtitle style={boxrule=0.4pt,
 colback=yellow!50!green!25!white},title=User]
I'm studying spell-checking in Finnish, can you download Alice's adventures in
    wonderland in Finnish from project gutenberg and produce a list of misspelt
    words in the text?
\end{tcolorbox}
\begin{tcolorbox}
 [colback=white!100,colframe=purple!75!black,width=.48\textwidth,
 righttitle=0.5cm,subtitle style={boxrule=0.4pt,
 colback=yellow!50!purple!25!white},title=ChatGPT]
I cannot directly download or process texts from external websites like Project
    Gutenberg. However, you can download the Finnish version of Alice's
    Adventures in Wonderland (if available) from Project Gutenberg yourself.
    Once you have the text, you can share it with me, and I can help analyze it
    for misspelled words.
\end{tcolorbox}
\begin{tcolorbox}
 [colback=white!100,colframe=green!75!black,width=.48\textwidth,
 righttitle=0.5cm,subtitle style={boxrule=0.4pt,
 colback=yellow!50!green!25!white},title=User]
Ok, I have attached the file here:
liisa-ihmemaassa.text
OTHER file
\end{tcolorbox}
\begin{tcolorbox}
 [colback=white!100,colframe=purple!75!black,width=.48\textwidth,
 righttitle=0.5cm,subtitle style={boxrule=0.4pt,
 colback=yellow!50!purple!25!white},title=ChatGPT]
Thank you for sharing the text. I can now analyze it for misspelled words.
    However, I need to clarify: do you want me to identify words that are not
    valid Finnish words, or are you looking for words that might be contextually
    incorrect?
\end{tcolorbox}
\begin{tcolorbox}
 [colback=white!100,colframe=green!75!black,width=.48\textwidth,
 righttitle=0.5cm,subtitle style={boxrule=0.4pt,
 colback=yellow!50!green!25!white},title=User]
Only words that are not valid Finnish at all, regardless of context
\end{tcolorbox}


    \caption{ChatGPT prompt for spell-checking and correction\label{chatgpt-prompt}}
\end{figure}

\end{document}
